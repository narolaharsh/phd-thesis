It may never cease to amaze one that we can now precisely measure a change in distance that is about 10,000 times smaller than the size of a proton. Such a technological feat was achieved in the quest to detect the gravitational waves, a prediction of Einstein's theory of relativity, which he believed to be too weak to be detectable. He may not have minded being corrected in 2015, about a century years after the theory of relativity was published, when a network of two Laser Interferometric Gravitational wave Observatories (LIGO) detected a gravitational wave signal. 

Since the first detection in 2015, the network of two LIGO and Virgo detectors has detected about 300 gravitational wave signals, leading to several breakthroughs in fundamental physics and astronomy. Upgrades to the current generation (or second-generation) of detectors, the addition of new detectors such as LIGO-India and KAGRA in Japan, make the future even more promising. Going beyond the second-generation detectors, the third generation of gravitational wave detectors, such as the Einstein Telescope, Cosmic Explorer, and the space-based observatory LISA (Laser Interferometric Space Antenna), are anticipated to usher in an era of precision science measurements. A significant effort is invested in developing tools which can extract physics-related insights from the gravitational wave data. To make the best use of the data collected by the next-generation detectors, a next generation of data analysis tools is also required. With this thesis, we attempt to take a step in that direction. 

This thesis is divided into two parts. The first part (chapters 1-3) delves into the pedagogy of gravitational waves. Here, I also lay down the foundation of the tools which are used in the second part of the thesis. The second part (chapters 4-6) contains the published results. Specifically, chapter 4 presents the \texttt{nijntje} -- (null stream-based noise transient elimination) -- framework. The framework can quickly and reliably remove a glitch that may be overlapping with a gravitational wave signal. Chapter 5 presents a fast method to infer the source properties of the gravitational wave signals. Chapter 6 contains simulations to demonstrate how we can use a strongly lensed gravitational wave signal to test a class of beyond general relativistic theories. 
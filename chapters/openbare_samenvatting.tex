\textbf{\textit{Wat zijn zwaartekrachtgolven?}} Stel je voor dat je op een rustige, zonnige dag aan de Oudegracht zit. Als iemand een steen in de gracht gooit, ontstaan er rimpelingen in het water. Zaken als het gewicht en de vorm van de steen en de snelheid waarmee deze is gegooid, kunnen bepalend zijn voor de eigenschappen van de rimpelingen (voornamelijk de amplitude en frequentie). Laten we aannemen dat we om een of andere reden (die later duidelijk zal worden) de steen niet in het water konden zien vallen. We konden alleen de rimpelingen zien. Is het in deze situatie mogelijk om iets te zeggen over de eigenschappen van de steen (gewicht, vorm, snelheid) door alleen naar de rimpelingen te kijken? Als we een natuurkundige theorie hebben die de eigenschappen van de rimpelingen voorspelt op basis van de eigenschappen van de steen, kunnen we inderdaad een aantal dingen afleiden. 

Op dezelfde manier zijn zwaartekrachtgolven (ZG's) rimpelingen in de ruimtetijd die worden veroorzaakt door bewegingen van zware objecten. We kunnen deze objecten niet met het blote oog vanaf de aarde zien, omdat ze te ver weg zijn en/of geen licht uitstralen. We zouden echter toch een aantal dingen over deze objecten kunnen afleiden als we de rimpelingen die ze produceren zouden kunnen detecteren \textit{en} als we een theorie zouden hebben die de eigenschappen van de golven kan voorspellen op basis van de eigenschappen van het object. We hebben inderdaad zo'n theorie, namelijk Einsteins relativiteitstheorie. Maar bijna 100 jaar nadat Einstein de relativiteitstheorie publiceerde, hadden we nog steeds geen instrumenten die gevoelig genoeg waren om zwaartekrachtgolven te detecteren. Einstein geloofde ook dat ze te zwak waren om enige gevolgen te hebben. \\

\textbf{\noindent\textit{Hoe detecteren we ze?}} Wanneer er een zwartekrachtgolf voorbij komt, beweegt deze de objecten in zijn omgeving. De instrumenten om zwaartekrachtgolven te detecteren meten relatieve bewegingen tussen twee testmassa's (spiegels). Op 14 september 2015 hebben twee LIGO-detectoren voor het eerst rechtstreeks een zwaartekrachtgolfsignaal waargenomen, GW150914, afkomstig van de samensmelting van twee zwarte gaten (massieve objecten met een zo sterke zwaartekracht dat zelfs licht er niet aan kan ontsnappen). Ze observeerden met name de laatste paar cycli van twee zwarte gaten die om elkaar heen draaiden, bij elke cyclus dichterbij elkaar kwamen en samensmolten tot een groter zwart gat. Door de samensmelting kwam een enorme hoeveelheid energie vrij in de ruimte, die de aarde bereikte in de vorm van rimpelingen in de ruimtetijd: zwaartekrachtgolven. Het ZG150914-signaal verplaatste de testmassa's met ongeveer $10^{-18}$ meter, en de detectoren hebben inderdaad zulke kleine bewegingen gemeten. Om een voorbeeld te geven: de meetnauwkeurigheid komt hier neer op het waarnemen van de op één na dichtstbijzijnde ster (de dichtstbijzijnde is de zon) vanaf de aarde, die zich verplaatst met de breedte van een menselijke haar. Tussen 2015 en 2025 heeft het netwerk van twee LIGO's, gelegen in Hanford en Louisiana, en de Virgo-detector in Pisa, ongeveer 300 zwaartekrachtsignalen gedetecteerd. Dankzij deze detecties hebben we spannende nieuwe fysica kunnen onderzoeken. \\

\textbf{\noindent\textit{Wat is een storing? Waarom is dat problematisch?}} Als we terugdenken aan de analogie met de rimpelingen in het water, kunnen er naast de bronnen die we willen observeren (stenen) nog veel andere ongewenste verstoringen zijn (zoals een fiets die in het kanaal valt). Op dezelfde manier zijn er naast een astrofysisch zwaartekrachtsignaal verschillende aardse bronnen die de testmassa's van de detector kunnen doen bewegen, waardoor een valse indruk wordt gewekt dat er een ZG voorbijgaat. We noemen dergelijke aardse bronnen \textit{ruisbronnen}. Meer specifiek worden kortstondige (tot enkele seconden of minuten) ruisbronnen \textit{glitches} genoemd. 

Glitches zijn problematisch. Soms kunnen ze zich voordoen als ZG-signalen. Soms treden ze precies op hetzelfde moment op als wanneer een zwaartekrachtgolf de detector bereikt, waardoor het signaal wordt verstoord. Glitches verslechteren de kwaliteit van de gegevens en belemmeren de overgang naar precisiewetenschap. Vooral voor de volgende generatie ZG-detectoren, zoals de Einstein Telescope, is het cruciaal om een methodologie te ontwikkelen om glitches uit de gegevens te verwijderen. In hoofdstuk 4 van mijn proefschrift hebben we een dergelijke methode ontwikkeld waarbij gebruik wordt gemaakt van het driehoekontwerp om glitches te verwijderen die het signaal overlappen, waarmee een belangrijke eerste stap is gezet in de richting van precisiewetenschap. Bovendien kwantificeren dergelijke analyses voor het eerst het voordeel van de driehoekige geometrie van de Einstein Telescopes ten opzichte van de alternatieve geometrieën. \\ 


\textbf{\noindent\textit{Wat kunnen we leren van ZG-signalen?}} Zodra we hebben vastgesteld dat het ZG-signaal “zuiver” is, d.w.z. zonder storingen, kunnen we verschillende eigenschappen van de objecten die ze hebben geproduceerd, schatten. We konden inderdaad de eigenschappen van de fusie van zwarte gaten, GW150914, schatten. We hebben onder andere geschat dat het ZG-signaal 1,3 miljard lichtjaar heeft afgelegd om de aarde te bereiken en dat het zich voornamelijk op het zuidelijk halfrond bevond. We hebben geschat dat de twee zwarte gaten ongeveer 36 en 29 keer de massa van de zon hadden en dat hun fusie een zwart gat heeft gecreëerd dat 62 keer zwaarder is dan de zon. De resterende massa wordt vrijgegeven in de vorm van energie die door ZG's wordt meegevoerd. We hebben ook geschat hoe snel de zwarte gaten draaiden en wat hun oriëntatie was. 

Hoewel het meten van de bronparameters de sleutel is tot verschillende downstream-analyses (zie voorbeeld hieronder), is het een rekenintensief proces. De modernste rekenpakketten hebben ongeveer een paar dagen tot een week nodig om één signaal te analyseren. Om de stijgende rekenkosten van LIGO en Virgo aan te pakken en een springplank te creëren voor de overgang naar het Einstein Telescope-tijdperk, zijn snellere technieken voor parameterraming nodig. Daartoe hebben we een methodologie ontwikkeld om parameterraming te versnellen en deze uitgebreid getest voor verschillende configuraties in hoofdstuk 5. \\

\textbf{\noindent\textit{Kunnen we de voorspellingen van de algemene relativiteitstheorie testen met zwaartekrachtlenzen?}} Het laatste hoofdstuk van mijn proefschrift gaat over het testen van de algemene relativiteitstheorie met behulp van sterk gelenseerde zwaartekrachtgolven – een fenomeen waarbij de ZG's een afbuiging ondergaan en zich kunnen splitsen in meerdere \textit{beelden}, als gevolg van een zwaar object in hun pad. Sterke lensing helpt om hetzelfde ZG-signaal meerdere keren waar te nemen. Als we bijvoorbeeld drie detectoren hebben (twee LIGO's en de Virgo) en we detecteren vier gelenseerde beelden, dan hebben we het signaal in feite twaalf keer waargenomen, wat resulteert in een nauwkeurige meting van de locatie aan de hemel en de afstand. Door middel van simulaties laten we zien dat deze eigenschap van sterk gelenseerde ZG's kan worden benut om de voorspellingen van (meer dan) algemene relativistische theorieën te testen, wat het belang van het zoeken naar sterk gelenseerde zwaartekrachtgolven onderstreept. 
\begin{itemize}
    \item{Lorenz guage and TT guage}
    \item{Linearized field equation}
    \item{Interation of GWs with matter}
\end{itemize}

\iffalse
An acclerating body (due to a gravity) disturbs the medium (spacetime) in which it travels and loses energy. These disturbances (or ripples) are known as the gravitational waves. Any object acclerating under the effect of gravity produces gravitational waves. However, there only a few instances where we can detect them. One of those instances is the collisions of two heavy objects.  

When they travel thorugh space, they interact with matter in the following way. Imagine gravitational waves passing through a ring of bids. The wave will stretch and squeese this bids in + and $\times$ manner. These are two polarizations of Gravitational waves. 

To detect them, we use Michaelson-Morley interferometer. It is an L-shaped infrastructure with two arms of equal length. Laser bounces off of the mirrors located at the end each arms. If detectors encounters a GW signal, the arm length changes which gives rise to a constructive and destructive interference pattern. That helps us measure a GW signal. 
\fi
It is intuitive to think that if this thesis is dropped from the top of a tall building and if the earth's gravity is the only force acting on it, it will travel in a straight line. This trajectory can be determined using Newton's laws of motion. The trajectory is also unique in the sense that it traces the shortest path between the start and end points in the flat space (i.e. where the gravity is weak). The general theory of relativity also known as general relativity generalizes the notion of straight lines using geodesics, which marks the shortest path between any two points in spaces with arbitrary curvature. For example, the geodesic between any two points on a spherical surface will trace the great circle passing through the two points. 
\subsubsection{The geodesic equation}
In practice, the shortest path between two points in curved space is obtained by solving the geodesic equation
\begin{equation}
    \label{eq:geodesic}
    \frac{d^2 x^\mu}{d\tau^2} + \Gamma^\mu_{\alpha\beta} \frac{dx^\alpha}{d\tau} \frac{dx^\beta}{d\tau} = 0,
\end{equation}    
where $\Gamma^\mu_{\alpha\beta}$ are called Christoffel symbols of the second kind and $\tau$ is the scalar parameter of motion also known as proper time. The indices $\mu, \alpha, \beta$ vary from 0...3. If an index repeats on the same side of the equal sign, it is summed over (Einstein's summation convention). 

Intutively, $\Gamma^\mu_{\alpha\beta}$ encodes the notion of force (e.g. gravity) of the Newtonian dynamics\footnote{When writing down axioms of motion, Newton did not use the word force as it did not exist at that time. He struggled to choose from power, efficacy, vigor, strength, or virtue.}. For example, when an inertial frame of an object is described using cartesian coordinate system, $\Gamma^\mu_{\alpha\beta}$ will be zero everywhere. Therefore, Eq.~\eqref{eq:geodesic} reduces to Newton's second law of motion in the absence of any external force, $\ddot{x} = 0$. 

\iffalse
\begin{equation}
    \Gamma^{\mu}_{\alpha\beta} = \frac{1}{2} g^{\mu\sigma} \left( \partial_\alpha g_{\beta\sigma} + \partial_\beta g_{\alpha\sigma} - \partial_\sigma g_{\alpha\beta} \right)
\end{equation}
write metric tensor and what it practically means. 
\fi
\section{Einstein field equation}
Though Newton's theory of gravity attempts to quantify the gravitational force exerted by one body onto another, the explanation on what causes such a force was delibarately left to future work~[cite]. The general relativity refines the notion of gravitational force by establishing it as an effect caused due to a massive object by curving the space around it. In the form of Einstein field equation, general relativity relates the distribution of matter to the curvature the space-time continuum (also referred to as spacetime) aroud it,
\begin{equation}
    \label{eq:field_eq}
    G_{\mu\nu} = 8\pi T_{\mu\nu}
\end{equation}
where the spacetime curvature is encoded in the Einstein tensor $(G_{\mu\nu})$ and distribution of matter is encoded in the stress-energy tensor~$(T_{\mu\nu})$. We use geometric units therefore the speed of light and universal gravitational constant in Eq.~\eqref{eq:field_eq} are set to unity. 

The unknown in Eq.~\eqref{eq:field_eq} is the typically the metric tensor $g_{\mu\nu}$ which can be solved for a given distribution of matter. 
However, The field equation itself represents a set of coupled partial differential equations and there are only a few cases where it has an exact solution, e.g., vacuum solutions also known as gravitational waves, Schwarzschild metric also known as non-spinning black holes, Kerr metric also known as spinning black holes. 

\begin{equation}
G_{\mu\nu} = R_{\mu\nu} - \frac{1}{2}g_{\mu\nu} R
\end{equation}
where $R_{\mu\nu}$ is the Ricci tensor and $g_{\mu\nu}$ is the metric tensor.
\subsubsection{Linearising the Einstein Field equation}
\begin{equation}
    \label{eq:lineq}
    \Box \bar{h}_{\mu\nu} = -16\pi T_{\mu\nu}
\end{equation}
\subsubsection{Gravitational waves: solution to field equation in vacuum}
To solve the linearised field equation far away from the source (which where our GW detectors are), we can assume that the stress-energy is equal to zero. Therefore, Eq.~\eqref{lineq} reduces to 
\begin{equation}
    \label{eq:waveq}
    \Box \bar{h}_{\mu\nu} = 0.
\end{equation}
Eq.~\eqref{eq:waveq} is a wave equation as it admits plane-wave solutions, 
\begin{align}
    \bar{h} &= A_{\mu\nu} \cos{\left(k_{\alpha}x^{\alpha}\right)},\\
    k_{\mu}^{\mu} &= 0.
\end{align}
which are the perturbation of a flat spacetime far away from the source that produced them, i.e. gravitational waves. 

\subsubsection{Plus and cross polarization}
Due to Lorenz gauge, 
\begin{equation}
    k^{\mu}A_{\mu\nu} = 0, 
\end{equation}
which means a gravitational wave does not have a component in the direction of their propagation, i.e., it is a transverse wave. 

The metric perturbation $\bar{h}_{\mu\nu}$ has 16 apparent degrees of freedom due to it being $4\times4$ tensor. From 16, 6 are restricted since $h_{\mu\nu}$ is symmetric in $\mu$ and $\nu$. From the remaining 10, 4 are restricted by Lorenz guage and 4 are restricted by transverse-traceless (TT) guage,
\begin{align}
    \bar{h_{0\mu}} &= 0, \\
    \bar{h^{\mu}_{\mu}} &= 0.
\end{align}
Therefore, $h_{\mu\nu}$ is a tensor with $16-6-4-4 = 2$ degrees of freedom which are known as the plus and cross polarization of the GWs. 
\[
h^{TT}_{\mu\nu} =
\begin{pmatrix}
0 & 0 & 0 & 0 \\
0 & h_{+} & h_{\times} & 0 \\
0 & h_{\times} & -h_{+} & 0 \\
0 & 0 & 0 & 0
\end{pmatrix}
\]

\section{Source frame to detector frame projection}
Gravitational wave detectors are sensitive to the following linear combination of the plus and cross polarization of a GW signal
\begin{equation}
    h(t) = F_+h_+(t) + F_{\times}h_{\times}(t),
\end{equation}
where $F_+$ and $F_{\times}$ are called antenna pattern functions of the detector. They enable the projection of the GW polarizations from source frame to the detector frame. Each $F_i$ is a function of four parameters, 
\begin{equation}
    F_i := F_i(\alpha, \delta, \psi, t_c),
\end{equation}
the right ascension ($\alpha$), declination ($\delta$), polarization angle ($\psi$), and time of merger ($t_c$). 

\begin{figure}
    \centering
    \includegraphics[width=\textwidth]{../figures/joint_pattern.pdf}
    \caption{Antenna pattern functions of the Virgo detector over the whole sky at the time of arrival of GW170817 (blue marker). The dark (bright) color marks the regions of low (high) sensitivity.}
    \label{fig:antenna_pattern}
\end{figure}

Figure~\ref{fig:antenna_pattern} shows the antenna pattern functions of the Virgo detector. We plot the $F_+$ and $F_{\times}$ over the whole sky by varying right ascension from $-\pi$ to $\pi$ and declination to $-\pi/2$ to $\pi/2$, at the time of arrival of GW170817. 
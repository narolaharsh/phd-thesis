Bayesian inference discussed in chapter~\ref{ch:gwpe} is the key to achieving a wide range of scientific insights from gravitational wave signal. The cost of it increases with the complexity of the signal model and the signal's duration. With standard methods, such analyses require at least a few days. Moreover, for third generation detectors, due to the lowered minimum sensitive frequency, the signal duration increases, leading to even longer analysis times. With the increased detection rate, such analyses can then become intractable. In this work, we present a fast parameter estimation method that includes higher order modes and precession using relative binning. Additionally, we extend the method so that it can be used to analyze individual GW as well as two overlapping GW signals in the 3G-era. We adapt our method to perform strong lensing searches. Specifically, we analyze
\begin{enumerate}
    \item A set of simulated BBH signals covering full parameter space to establish validity. 
    \item Two LIGO-Virgo events with significant contribution from higher order modes: GW190412~\cite{LIGOScientific:2020stg} and GW190814~\cite{LIGOScientific:2020zkf}.
    \item A set of simulated GW signals for the 3G-detector; isolated as well as overlapping.
    \item The event pair GW191103-GW191105 as well as a set of simulated pairs, to investigate strong lensing hypothesis
\end{enumerate}

\section{Mode-by-mode relative binning}
\begin{figure}
    \centering
    \includegraphics[width=.4\textwidth]{../figures/phenom_d_xphm_full_waveform_ratio.pdf}\hspace{.5cm}
    \includegraphics[width=.4\textwidth]{../figures/hL_C_ratio.pdf}
    \caption{Ratio (vertical axis) of the real part of the proposal to the fiducial waveform for the plus polarization as a function of frequency (horizontal axis). The brown line is for the IMRPhenomD waveform model with only the dominant mode and without precession, while the blue line is for the IMRPhenomXPHM waveform model containing HOMs and precession. The fiducial parameters are similar to the source parameters of GW190412~\cite{LIGOScientific:2020stg}, a BBH merger with asymmetric masses and detectable HOM content, and the proposal parameters have a 5$\%$ shift from the fiducial chirp mass and mass ratio.}
    \label{fig:dxp-ratio}
\end{figure}

\section{Validation of the method}
\subsection{On simulated data}
\begin{figure}
    \centering
    \includegraphics[width=.75\textwidth]{../figures/pp_3.0_partial_no_railing.pdf}
    \caption{Percentile-percentile showcasing the robustness of relative binning method on BBH events. Each line traces the diagonal, indicating that the parameter corresponding to it is recovered with expected accuracy. The numbers in the brackets of the legend show the p-values of the KS tests. The combined p-value of all parameters is 0.6625, consistent with the hypothesis that individual p-values were derived from a uniform distribution as expected. The shaded regions show 1$\sigma$,2$\sigma$,3$\sigma$ confidence intervals in decreasing order of opacity.}
    \label{fig:pp-plot}
\end{figure}

\subsection{On real data: analysing events detected by LIGO-Virgo}
\begin{figure}
    \centering
    \includegraphics[width=.75\textwidth]{../figures/overlap_production_gw190412_gwtc.pdf}
    %\includegraphics[width=.75\textwidth]{../figures/overlap_production_gw190814_corr_branch_gwtc.pdf}
    \caption{Analysis of GW190412. The blue colour marks the results obtained from relative binning method and grey colour marks the results reported in GWTC-3 data release.}
    \label{fig:gwtc3-events}
\end{figure}

\begin{figure}
    \centering
    \includegraphics[width=.75\textwidth]{../figures/overlap_production_gw190814_corr_branch_gwtc.pdf}
    \caption{Analysis of GW190814. The blue colour marks the results obtained from relative binning method and grey colour marks the results reported in GWTC-3 data release.}
    \label{fig:gw190814}
\end{figure}


\begin{figure}
    \centering
    \includegraphics[width=0.53\textwidth]{../figures/thesis_pair_191103_191105.pdf} \hspace{.5cm}
    \includegraphics[width=0.41\textwidth]{../figures/lens_pair_191103_191105.pdf}
    \caption{Investigating the pair GW191103 (red) and GW191105 (green) under strong lensing hypothesis (blue).}
    \label{fig:lensed_pair}
\end{figure}

\subsection{Third generation events}

\begin{figure}
    \centering
    \includegraphics[width=.75\textwidth]{../figures/prod_third_gen.pdf}
    \caption{Third generation injection example}
    \label{fig:third-gen}
\end{figure}


\begin{figure}
    \centering
    \includegraphics[width=.72\textwidth]{../figures/overlapping_signals_A.pdf}\vspace{.5cm}
    \includegraphics[width=.72\textwidth]{../figures/overlapping_signals_B.pdf}
    \caption{Analysis of overlapping signals using joint parameter estimation}
    \label{fig:overlapping-signal}
\end{figure}

\section{Comparison of speed-up and accuracy}

\begin{figure}
    \centering
    \includegraphics[width=.5\textwidth]{../figures/thesis_speed_up_summary.pdf}
    \caption{Relation between the speed up achieved by the the method in relation to the total mass}
    \label{fig:speed}
\end{figure}




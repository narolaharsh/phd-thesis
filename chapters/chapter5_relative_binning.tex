Bayesian inference discussed in chapter~\ref{ch:gwpe} is the key to achieving a wide range of scientific insights from gravitational wave signal. The cost of it increases with the complexity of the signal model and the signal's duration. With standard methods, such analyses require at least a few days. Moreover, for third generation detectors, due to the lowered minimum sensitive frequency, the signal duration increases, leading to even longer analysis times. With the increased detection rate, such analyses can then become intractable. In this work, we present a fast parameter estimation method that includes higher order modes and precession using relative binning. Additionally, we extend the method so that it can be used to analyze individual GW as well as two overlapping GW signals in the 3G-era. We adapt our method to perform strong lensing searches. Indeed, a fast parameter estimation has a wide range of application. Specifically, we apply our method to
\begin{enumerate}
    \item A set of simulated BBH signals covering full parameter space to establish validity. 
    \item Two LIGO-Virgo events with significant contribution from higher order modes: GW190412~\cite{LIGOScientific:2020stg} and GW190814~\cite{LIGOScientific:2020zkf}.
    \item A set of simulated GW signals for the 3G-detector; isolated as well as overlapping.
    \item The event pair GW191103-GW191105 as well as a set of simulated pairs, to investigate strong lensing hypothesis
\end{enumerate}

\section{Mode-by-mode relative binning}
From chapter~\ref{ch:gwpe}, let us recall the (log) likelihood function
\begin{equation}
\label{eqn:indi-likelihood}
\ln \mathcal{L}(\vec{\theta}) = -\frac{1}{2}\left<\vec{d}-\vec{h}(\vec{\theta})|\vec{d}-\vec{h}(\vec{\theta})\right>\,,
\end{equation}
where $\vec{h}(\vec{\theta})$ represents the GW waveform as a function of the source parameters $\vec{\theta}$ and $\left<\,.\,|\,.\,\right>$ 
denotes the noise-weighted inner product. Expanding the inner product in Eq.~\eqref{eqn:indi-likelihood}, we get
\iffalse
\begin{equation}
(\vec{a}|\vec{b}) = \frac{4}{T}\Re\sum_f \frac{a(f)b^*(f)}{S_n(f)}.
\end{equation}
Here, $f$, $S_n(f)$, and $T$ respectively denote the frequency, one-sided noise power spectral density 
(PSD), and duration, while $\Re$ denotes the real part; the sum over frequencies will be discussed in detail below. 
Expanding $\left<\vec{d}-\vec{h}(\vec{\theta})|\vec{d}-\vec{h}(\vec{\theta}))$ in Eq.~\eqref{eqn:indi-likelihood}, we get
\fi
\begin{equation}
\label{eqn:logL}
\ln \mathcal{L}(\vec{\theta}) = \left<\vec{d}|\vec{h}\right> - \frac{1}{2}\left<\vec{h}|\vec{h}\right> - \frac{1}{2}\left<\vec{d}|\vec{d}\right>.
\end{equation}

Likelihood-based inference involves calculating the innder products for a large number of values for $\vec{\theta}$ to explore the parameter space. Therefore, we want to improve the speed in computing these quantities. Let us introduce the notations
\begin{align}
    \ln \mathcal{L}_{dh} &\equiv \left<\vec{d}|\vec{h}\right>, \\
    \ln \mathcal{L}_{hh} &\equiv \left<\vec{h}|\vec{h}\right>, 
\end{align}

to rewrite Eq.~\eqref{eqn:logL} as 
\begin{equation}
\label{eqn:short-l}
\ln \mathcal{L}(\vec{\theta}) = \ln \mathcal{L}_{dh} - \frac{1}{2} \ln \mathcal{L}_{hh}\,.
\end{equation}
Note that we dropped the last term of Eq.~\eqref{eqn:logL} in the equation above. It is also known as the noise log likelihood. Since it does not depend on the source parameters, it need only be computed once for the data.

\subsection{Dominant mode and aligned spin}
When performing parameter estimation, we are interested in knowing the shape of likelihood around its peak. In other words, we want to evaluate the likelihoods 
for waveforms that are {\it similar} to the waveform at which the likelihood peaks. 
If we choose a fiducial value of $\vec{\theta}$ which is close enough to the peak, 
we can estimate the likelihood by expanding it around the fiducial value. 
Let us denote the fiducial source parameters by $\vec{\theta}'$ and the waveform at the 
fiducial parameters by $\vec{h}'$. For any proposal waveform $\vec{h}$, 
under the assumption that $\vec{h}$ and $\vec{h}'$ are close enough, we can find 
frequency bins $b =  [f_{min}, f_{max}]$ such that the ratio 
$\vec{h}/\vec{h}'$ can be linearly approximated as

\begin{equation}
\label{eqn:ratio}
\frac{h}{h'}(f) = r_1(b) + r_2(b)(f-f_c(b))+\mathcal{O}(f^2), 
\end{equation}

within a given bin, where $r_1(b)$ and $r_2(b)$ are expansion coefficients 
and $f_c$ denotes the central value of the frequency bin $b$. 

The approximation made in Eq.~(\ref{eqn:ratio}) performs better for the IMRPhenomD waveform compared to the IMRPhenomXPHM one because the ratio $\vec{h}/\vec{h}'$ varies more smoothly with $f$ for the former (see left plot of Figure~\ref{fig:dxp-ratio}). Therefore, when using the IMRPhenomD waveform a few bins are sufficient to make the linear expansion from Eq.(\ref{eqn:ratio}) precise in each bin. This is not the case anymore for more complex waveform models. When using the IMPRhenomXPHM waveform, as the ratio is subject to more fluctuation, we need to consider a larger number of bins, which increases waveform generation costs and makes the likelihood evaluation slow. 

\begin{figure}
    \centering
    \includegraphics[width=.4\textwidth]{../figures/phenom_d_xphm_full_waveform_ratio.pdf}\hspace{.5cm}
    \includegraphics[width=.4\textwidth]{../figures/hL_C_ratio.pdf}
    \caption{\textit{Left}: Ratio of the real part of the proposal to the fiducial waveform for the plus polarization as a function of frequency. The IMRPhenomD (brown) waveform model includes only the dominant mode and aligned spins. The IMRPhenomXPHM (black) model includes higher order modes and precessing spins. \textit{Right}: Decomposition of IMRPhenomXPHM waveform into component modes (top right) and precession coefficients (bottom right). The fiducial parameters are similar to the source parameters of GW190412~\cite{LIGOScientific:2020stg}, a BBH merger with asymmetric masses and detectable HOM content, and the proposal parameters have a 5$\%$ shift from the fiducial chirp mass and mass ratio.}
    \label{fig:dxp-ratio}
\end{figure}

\subsection{Inclusion of higher order modes and precessing spins}
In order to make the approximation in Eq.~(\ref{eqn:ratio}) robust for precessing waveforms with HOMs, 
let us express the detector frame waveform for given $\vec{\theta}$ into its 
component modes~\cite{Pratten:2020ceb} 
\begin{equation}
h(f) = \sum_{l, m}(F_+C^{+}_{l, m}(f)+F_\times C^{\times}_{l, m}(f))\,h^L_{l, m}(f)\,.
\label{eqn:comp-mode}
\end{equation} 

Here, $F_+$ and $F_\times$ are the antenna pattern functions of the interferometer, which depend 
only on extrinsic parameters; $\vec{h}^{L}$ denotes the waveform in the co-precessing frame 
(also known as the $L$-frame); and the indices $l$ and $m$ label the different modes. The coefficients $C^{+}_{l, m}$ and $C^{\times}_{l, m}$ account for 
the {\it twisting-up} procedure~\cite{Pratten:2020ceb} which transforms the waveform from a 
co-precessing frame to the inertial (or observer) frame. 
The benefit of handling each mode in the $L$-frame separately is that the ratio of a 
fiducial $\vec{h}'^{L}_{l, m}$ to a proposal $\vec{h}^{L}_{l, m}$ oscillates less compared to the 
ratio of the full waveforms $\vec{h}/\vec{h}'$. This can be seen by comparing the bottom right panel of 
Figure~\ref{fig:dxp-ratio} with the top right panel of Figure~\ref{hlc-plot}. 
Therefore, a linear expansion similar to the one in Eq.~\eqref{eqn:ratio} 
can be made using the ratio $\vec{h}^{L}_{l, m}/\vec{h}'^L_{l, m}$ instead of 
$\vec{h}/\vec{h}'$ and will require a lower number of frequency bins for more complex models:
\begin{equation}
\label{eqn:l-ratio}
\frac{h^{L}_{l, m}}{h'^L_{l, m}} (f) = r_{1, l, m}(b) + r_{2, l, m} (b)(f-f_c(b)) + \mathcal{O}(f^2) \,,
\end{equation}

Now, let us denote the factor $(F_+C^{+}_{l, m}(f)+F_\times C^{\times}_{l, m}(f))$ in 
Eq.~(\ref{eqn:comp-mode}) by $\vec{C}_{l, m}$. We can perform similar expansions on the ratios 
$\vec{C}_{l, m}/\vec{C}'_{l, m}$:
\begin{equation}
\label{eqn:c-ratio}
\frac{C_{l, m}}{C'_{l, m}}(f) = s_{1, l, m}(b) + s_{2, l, m}(b)(f-f_c(b)) + \mathcal{O}(f^2).
\end{equation}
Below, the expansion coefficients will collectively be denoted 
$\mathcal{R} = \{\vec{r}_1, \vec{r}_2, \vec{s}_1, \vec{s}_2\}$. 
For now, we assume we can find frequency bins such that the piece-wise linear interpolations 
in Eqs.~(\ref{eqn:l-ratio}) and (\ref{eqn:c-ratio}) are 
valid. How such bins can be found will be explained in Sec.~\ref{sec:bin-selection}.

Once the frequency bins are made based on the chosen fiducial waveforms, we can pre-compute some 
\textit{summary data}~\cite{Leslie:2021ssu} based only on the reference waveform and useful for the 
computation of the inner products in Eq.~(\ref{eqn:short-l}). These are:
\begin{eqnarray}
&& W_{l, m}(b) = \frac{4}{T} \sum_{f\in b}\frac{d(f)h'^{L*}_{l, m}(f)C'^{*}_{l, m}(f)}{S_n(f)} \, , 
\nonumber \\
&& X_{l, m}(b) = \frac{4}{T} \sum_{f\in b}\frac{d(f)h'^{L*}_{l, m}(f)C'^*_{l, m}(f)}{S_n(f)}(f-f_c(b)) \,, 
\nonumber\\
&& Y_{l, m, \bar{l}, \bar{m}}(b) = \frac{4}{T} \sum_{f\in b}\frac{h'^L_{l, m}(f)C'_{l, m}(f)h'^{L*}_{\bar{l}, \bar{m}}(f)C'^*_{\bar{l}, \bar{m}}(f)}{S_n(f)} \,, 
\nonumber\\
&&Z_{l, m, \bar{l}, \bar{m}}(b)  \nonumber\\
&&= \frac{4}{T} \sum_{f\in b}\frac{h'^L_{l, m}(f)C'_{l, m}(f)h'^{L*}_{\bar{l}, \bar{m}}(f)C'^*_{\bar{l}, \bar{m}}(f)}{S_n(f)}(f-f_c(b))\,.
\nonumber\\
\end{eqnarray}

Note that the summary data 
$\mathcal{D} = \{W_{l, m}, X_{l, m}, Y_{l, m, \bar{l}, \bar{m}}, Z_{l, m, \bar{l}, \bar{m}} \}$ 
need to be computed only once per analysis and they depend only on the choice of the fiducial 
parameters $\vec{\theta}'$. On the other hand, the coefficients $\mathcal{R}$ need to be computed for every 
proposal waveform as they depend on $\vec{\theta}$ and $\vec{\theta}'$.  

Using the summary data $\mathcal{D}$ and the coefficients $\mathcal{R}$, we can estimate $\ln\mathcal{L}_{dh}$ and $\ln\mathcal{L}_{hh}$ in Eq.~\eqref{eqn:short-l} 
for given $\vec{\theta}$ as


\begin{align}
\label{eqn:dh}
\ln\mathcal{L}_{dh} &= \Re \sum_{(l, m)} \sum_{b}  \left\{  W_{l, m}(b)\left[r_{1, l, m}^*(b) s_{1, l, m}^*(b)\right] +X_{l, m}(b)\left[r_{1, l, m}^*(b)s_{2, l, m}^*(b) + r_{2, l, m}^*(b) s_{1, l, m}^*(b) \right]  \right\} \,, \\
\label{eqn:hh}
\ln\mathcal{L}_{hh} &=\Re  \sum_{(l, m), (\bar{l}, \bar{m})} \sum_{b} \left\{  Y_{l, m, \bar{l}, \bar{m}}(b) \left[ r_{1, l, m}(b)s_{1, l, m}(b) r^*_{1, \bar{l}, \bar{m}}(b) s^*_{1, \bar{l}, \bar{m}}(b) \right] \right. \nonumber \\
&\qquad\qquad\qquad\qquad +  \left. Z_{l, m, \bar{l}, \bar{m}}(b) \left[r_{1, l, m}(b) r_{1, \bar{l}, \bar{m}}^*(b) \left( s_{1, \bar{l}, \bar{m}}^*(b) s_{2, l, m}(b) + s_{1, l, m}(b) s_{2, \bar{l}, \bar{m}}^*(b) \right) \right. \right. \nonumber \\ 
&\qquad\qquad\qquad\qquad + \left. \left.  s_{1, l, m}(b)s_{1, \bar{l}, \bar{m}}^*(b)\left(r_{1, \bar{l}, \bar{m}}^*(b) r_{2, l, m}(b) + r_{1, l, m}(b) r_{2, \bar{l}, \bar{m}}^*(b)\right) \right] \right\},
\end{align}

which together with Eq.~(\ref{eqn:short-l}) enables us to approximate the likelihood. Provided the number of bins is reduced compared to the usual method, this approach should offer a rapid likelihood evaluation, leading to a rapid parameter inference.

\subsection{Constructing the relative binning grid}
\label{sec:bin-selection}

The selection of bins is an important step when using the relative binning method, as it determines not only the quality 
of the approximation but also the final speed-up obtained. Previous works focusing on the relative binning method have proposed various bin selection algorithms~\cite{Zackay:2018qdy, Dai:2018dca, Leslie:2021ssu}. The choice of algorithm can vary depending on the waveform model and the scope of applicibality. Indeed, in~\cite{Zackay:2018qdy} and~\cite{Dai:2018dca} the bins are chosen based on the differential phase change of the GW signal whereas in~\cite{Leslie:2021ssu} they are chosen by bisecting the uniform grid until the error in the likelihood in each bin is below desired value.  

The objective of the bin selection step is to identify a minimal set of frequency bins in which the ratios of proposal waveform components 
($\boldsymbol{h}^L_{l, m}$ and $\boldsymbol{C}_{l, m}$) to fiducial waveform components are linear to sufficient approximation.
We refer to a set of frequency bins which satisfy this condition as an RBGrid. 
The RBGrid is used to compute the summary data $\mathcal{D}$, and, once the sampling begins, 
to compute coefficients $\mathcal{R}$ for the points in parameter space proposed by the sampler.  

To initialize the bin selection algorithm, we need the following quantities: 
a uniform frequency grid with spacing $\Delta f = 1/T$, where the time $T$ is sufficiently long so as to easily accommodate the length of the signal; 
a fixed value of ``total error'' on log likelihood, $\epsilon$; 
an initial value $N_0$ for the total number of frequency bins;  
a set of fiducial waveform parameters; and a set of test parameters. For definiteness we set  
$\epsilon = 0.01$ and $N_0 = 200$. In simulations, the fiducial parameters are set to the injection parameters, while for real events they are set to the 
maximum likelihood parameters obtained from earlier analyses. To choose a set of test parameters, we perturb the chirp mass and the 
mass ratio of the fiducial parameters by a random relative value in a $[ -10\%, 10\%]$ interval, and the rest of the parameters are kept the same. 

Using the quantities described in the paragraph above, we construct the RBGrid as follows. 
We start from the first bin of the uniform frequency grid. 
We merge the first bin with the next one and calculate the partial relative binning log likelihoods $\ln\mathcal{L}_{dh}$ and $\ln\mathcal{L}_{hh}$ 
defined in Eqs.~\eqref{eqn:dh}-\eqref{eqn:hh}, 
using as test quantities the edges of the combined frequency bin, and the fiducial waveform parameters.
We also compute the corresponding exact partial likelihoods obtained by using all frequencies within the bins and not just the edges. 
If the absolute difference between the partial relative binning likelihoods and the exact partial likelihoods is less than $\epsilon/\sqrt{N_0}$, 
we add a subsequent bin into it and repeat. 
If the difference is larger than that, we do not add any more bins and move to the next bin and repeat. 
This continues until the maximum frequency of the uniform grid is reached. The value of $N_0$ is then updated to the total number of bins obtained 
through the above process, and the same procedure is started again with the same $\epsilon$ but the updated $N_0$. 
The algorithm terminates when $N_0$ ceases to change, and the edges of the merged bins in the final iteration become the RBGrid. 

The choice of $\epsilon$ controls the trade-off between speed and accuracy, where a larger value increases speed but reduces accuracy. By contrast, we have verified
that the bin selection procedure is not very sensitive to the initial value of $N_0$; it mainly affects the convergence rate of the algorithm. 
Setting $N_0$ equal to the length of the uniform grid would eliminate the need for this additional tuning parameter, but this would slow down the bin selection
algorithm as a whole\footnote{Our implementation of the bin selection and relative binning parameter estimation framework can be found in the git repository at~\cite{janquart:2022}.   
}. 

%The construction of the RBGrid when performing joint parameter estimation on pairs of lensed images is straightforward.
%Besides the fiducial parameters, test parameters, and threshold value $\epsilon$, the RBGrid also depends on the signal-to-noise ratio (SNR) and PSD of each event. 
%As the latter two quantities and the lensing parameters can vary significantly between two events in a pair,  
%we construct the RBGrid for each event separately and use each of them to calculate $\{\mathcal{R}_1, \mathcal{D}_1\}$ and $\{\mathcal{R}_2, \mathcal{D}_2\}$ during the 
%data analysis. 

\section{Extending the relative binning method to}
\subsubsection{Perform parameter estimation on the overlapping signals}
\subsubsection{Perform strong lensing searches}


\section{Validation of the method}
\subsection{On simulated data}
\begin{figure}[h]
    \centering
    \includegraphics[width=.75\textwidth]{../figures/pp_3.0_partial_no_railing.pdf}
    \caption{Percentile-percentile showcasing the robustness of relative binning method on BBH events. Each line traces the diagonal, indicating that the parameter corresponding to it is recovered with expected accuracy. The numbers in the brackets of the legend show the p-values of the KS tests. The combined p-value of all parameters is 0.6625, consistent with the hypothesis that individual p-values were derived from a uniform distribution as expected. The shaded regions show 1$\sigma$,2$\sigma$,3$\sigma$ confidence intervals in decreasing order of opacity.}
    \label{fig:pp-plot}
\end{figure}

\subsection{On real data: analysing events detected by LIGO-Virgo}
\begin{figure}[h]
    \centering
    \includegraphics[width=.75\textwidth]{../figures/overlap_production_gw190412_gwtc.pdf}
    %\includegraphics[width=.75\textwidth]{../figures/overlap_production_gw190814_corr_branch_gwtc.pdf}
    \caption{Analysis of GW190412. The blue colour marks the results obtained from relative binning method and grey colour marks the results reported in GWTC-3 data release.}
    \label{fig:gwtc3-events}
\end{figure}

\begin{figure}[h]
    \centering
    \includegraphics[width=.75\textwidth]{../figures/overlap_production_gw190814_corr_branch_gwtc.pdf}
    \caption{Analysis of GW190814. The blue colour marks the results obtained from relative binning method and grey colour marks the results reported in GWTC-3 data release.}
    \label{fig:gw190814}
\end{figure}


\begin{figure}[h]
    \centering
    \includegraphics[width=0.53\textwidth]{../figures/thesis_pair_191103_191105.pdf} \hspace{.5cm}
    \includegraphics[width=0.41\textwidth]{../figures/lens_pair_191103_191105.pdf}
    \caption{Investigating the pair GW191103 (red) and GW191105 (green) under strong lensing hypothesis (blue).}
    \label{fig:lensed_pair}
\end{figure}

\subsection{Third generation events}

\begin{figure}[h]
    \centering
    \includegraphics[width=.75\textwidth]{../figures/prod_third_gen.pdf}
    \caption{Third generation injection example}
    \label{fig:third-gen}
\end{figure}


\begin{figure}[h]
    \centering
    \includegraphics[width=.72\textwidth]{../figures/overlapping_signals_A.pdf}\vspace{.5cm}
    \includegraphics[width=.72\textwidth]{../figures/overlapping_signals_B.pdf}
    \caption{Analysis of overlapping signals using joint parameter estimation}
    \label{fig:overlapping-signal}
\end{figure}

\section{Comparison of speed-up and accuracy}

\begin{figure}[h]
    \centering
    \includegraphics[width=.5\textwidth]{../figures/thesis_speed_up_summary.pdf}
    \caption{Relation between the speed up achieved by the the method in relation to the total mass}
    \label{fig:speed}
\end{figure}




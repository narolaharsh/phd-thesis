Bayesian inference discussed in chapter~\ref{ch:gwpe} is the key to achieving a wide range of scientific insights from gravitational wave signal. The cost of it increases with the complexity of the signal model and the signal's duration. With standard methods, such analyses require at least a few days. Moreover, for third generation detectors, due to the lowered minimum sensitive frequency, the signal duration increases, leading to even longer analysis times. With the increased detection rate, such analyses can then become intractable. In this work, we present a fast parameter estimation method that includes higher order modes and precession using relative binning. Additionally, we extend the method so that it can be used to analyze individual GW as well as two overlapping GW signals in the 3G-era. We adapt our method to perform strong lensing searches. Indeed, a fast parameter estimation has a wide range of application. Specifically, we apply our method to
\begin{enumerate}
    \item A set of simulated BBH signals covering full parameter space to establish validity. 
    \item Two LIGO-Virgo events with significant contribution from higher order modes: GW190412~\cite{LIGOScientific:2020stg} and GW190814~\cite{LIGOScientific:2020zkf}.
    \item A set of simulated GW signals for the 3G-detector; isolated as well as overlapping.
    \item The event pair GW191103-GW191105 as well as a set of simulated pairs, to investigate strong lensing hypothesis
\end{enumerate}

\section{Mode-by-mode relative binning}
From chapter~\ref{ch:gwpe}, let us recall the (log) likelihood function
\begin{equation}
\label{eqn:indi-likelihood}
\ln \mathcal{L}(\vec{\theta}) = -\frac{1}{2}\left<\vec{d}-\vec{h}(\vec{\theta})|\vec{d}-\vec{h}(\vec{\theta})\right>\,,
\end{equation}
where $\vec{h}(\vec{\theta})$ represents the GW waveform as a function of the source parameters $\vec{\theta}$ and $\left<\,.\,|\,.\,\right>$ 
denotes the noise-weighted inner product. Expanding the inner product in Eq.~\eqref{eqn:indi-likelihood}, we get
\iffalse
\begin{equation}
(\vec{a}|\vec{b}) = \frac{4}{T}\Re\sum_f \frac{a(f)b^*(f)}{S_n(f)}.
\end{equation}
Here, $f$, $S_n(f)$, and $T$ respectively denote the frequency, one-sided noise power spectral density 
(PSD), and duration, while $\Re$ denotes the real part; the sum over frequencies will be discussed in detail below. 
Expanding $\left<\vec{d}-\vec{h}(\vec{\theta})|\vec{d}-\vec{h}(\vec{\theta}))$ in Eq.~\eqref{eqn:indi-likelihood}, we get
\fi
\begin{equation}
\label{eqn:logL}
\ln \mathcal{L}(\vec{\theta}) = \left<\vec{d}|\vec{h}\right> - \frac{1}{2}\left<\vec{h}|\vec{h}\right> - \frac{1}{2}\left<\vec{d}|\vec{d}\right>.
\end{equation}

Likelihood-based inference involves calculating the innder products for a large number of values for $\vec{\theta}$ to explore the parameter space. Therefore, we want to improve the speed in computing these quantities. Let us introduce the notations
\begin{align}
    \ln \mathcal{L}_{dh} &\equiv \left<\vec{d}|\vec{h}\right>, \\
    \ln \mathcal{L}_{hh} &\equiv \left<\vec{h}|\vec{h}\right>, 
\end{align}

to rewrite Eq.~\eqref{eqn:logL} as 
\begin{equation}
\label{eqn:short-l}
\ln \mathcal{L}(\vec{\theta}) = \ln \mathcal{L}_{dh} - \frac{1}{2} \ln \mathcal{L}_{hh}\,.
\end{equation}
Note that we dropped the last term of Eq.~\eqref{eqn:logL} in the equation above. It is also known as the noise log likelihood. Since it does not depend on the source parameters, it need only be computed once for the data.

\subsection{Dominant mode and aligned spin}
When performing parameter estimation, we are interested in knowing the shape of likelihood around its peak. In other words, we want to evaluate the likelihoods 
for waveforms that are {\it similar} to the waveform at which the likelihood peaks. 
If we choose a fiducial value of $\vec{\theta}$ which is close enough to the peak, 
we can estimate the likelihood by expanding it around the fiducial value. 
Let us denote the fiducial source parameters by $\vec{\theta}'$ and the waveform at the 
fiducial parameters by $\vec{h}'$. For any proposal waveform $\vec{h}$, 
under the assumption that $\vec{h}$ and $\vec{h}'$ are close enough, we can find 
frequency bins $b =  [f_{min}, f_{max}]$ such that the ratio 
$\vec{h}/\vec{h}'$ can be linearly approximated as

\begin{equation}
\label{eqn:ratio}
\frac{h}{h'}(f) = r_1(b) + r_2(b)(f-f_c(b))+\mathcal{O}(f^2), 
\end{equation}

within a given bin, where $r_1(b)$ and $r_2(b)$ are expansion coefficients 
and $f_c$ denotes the central value of the frequency bin $b$. 

The approximation made in Eq.~\eqref{eqn:ratio} performs better for the IMRPhenomD waveform compared to the IMRPhenomXPHM one because the ratio $\vec{h}/\vec{h}'$ varies more smoothly with $f$ for the former (see left plot of Figure~\ref{fig:dxp-ratio}). Therefore, when using the IMRPhenomD waveform a few bins are sufficient to make the linear expansion from Eq.(\ref{eqn:ratio}) precise in each bin. This is not the case anymore for more complex waveform models. When using the IMPRhenomXPHM waveform, as the ratio is subject to more fluctuation, we need to consider a larger number of bins, which increases waveform generation costs and makes the likelihood evaluation slow. 

\begin{figure}
    \centering
    \includegraphics[width=.4\textwidth]{../figures/phenom_d_xphm_full_waveform_ratio.pdf}\hspace{.5cm}
    \includegraphics[width=.4\textwidth]{../figures/hL_C_ratio.pdf}
    \caption{\textit{Left}: Ratio of the real part of the proposal to the fiducial waveform for the plus polarization as a function of frequency. The IMRPhenomD (brown) waveform model includes only the dominant mode and aligned spins. The IMRPhenomXPHM (black) model includes higher order modes and precessing spins. \textit{Right}: Decomposition of IMRPhenomXPHM waveform into component modes (top right) and precession coefficients (bottom right). The fiducial parameters are similar to the source parameters of GW190412~\cite{LIGOScientific:2020stg}, a BBH merger with asymmetric masses and detectable HOM content, and the proposal parameters have a 5$\%$ shift from the fiducial chirp mass and mass ratio.}
    \label{fig:dxp-ratio}
\end{figure}

\subsection{Inclusion of higher order modes and precessing spins}
In order to make the approximation in Eq.~\eqref{eqn:ratio} robust for precessing waveforms with HOMs, 
let us express the detector frame waveform for given $\vec{\theta}$ into its 
component modes~\cite{Pratten:2020ceb} 
\begin{equation}
h(f) = \sum_{l, m}(F_+C^{+}_{l, m}(f)+F_\times C^{\times}_{l, m}(f))\,h^L_{l, m}(f)\,.
\label{eqn:comp-mode}
\end{equation} 

Here, $F_+$ and $F_\times$ are the antenna pattern functions of the interferometer, which depend 
only on extrinsic parameters; $\vec{h}^{L}$ denotes the waveform in the co-precessing frame 
(also known as the $L$-frame); and the indices $l$ and $m$ label the different modes. The coefficients $C^{+}_{l, m}$ and $C^{\times}_{l, m}$ account for 
the {\it twisting-up} procedure~\cite{Pratten:2020ceb} which transforms the waveform from a 
co-precessing frame to the inertial (or observer) frame. 
The benefit of handling each mode in the $L$-frame separately is that the ratio of a 
fiducial $\vec{h}'^{L}_{l, m}$ to a proposal $\vec{h}^{L}_{l, m}$ oscillates less compared to the 
ratio of the full waveforms $\vec{h}/\vec{h}'$. This can be seen by comparing the bottom right panel of 
Figure~\ref{fig:dxp-ratio} with the top right panel of Figure~\ref{hlc-plot}. 
Therefore, a linear expansion similar to the one in Eq.~\eqref{eqn:ratio} 
can be made using the ratio $\vec{h}^{L}_{l, m}/\vec{h}'^L_{l, m}$ instead of 
$\vec{h}/\vec{h}'$ and will require a lower number of frequency bins for more complex models:
\begin{equation}
\label{eqn:l-ratio}
\frac{h^{L}_{l, m}}{h'^L_{l, m}} (f) = r_{1, l, m}(b) + r_{2, l, m} (b)(f-f_c(b)) + \mathcal{O}(f^2) \,,
\end{equation}

Now, let us denote the factor $(F_+C^{+}_{l, m}(f)+F_\times C^{\times}_{l, m}(f))$ in 
Eq.~\eqref{eqn:comp-mode} by $\vec{C}_{l, m}$. We can perform similar expansions on the ratios 
$\vec{C}_{l, m}/\vec{C}'_{l, m}$:
\begin{equation}
\label{eqn:c-ratio}
\frac{C_{l, m}}{C'_{l, m}}(f) = s_{1, l, m}(b) + s_{2, l, m}(b)(f-f_c(b)) + \mathcal{O}(f^2).
\end{equation}
Below, the expansion coefficients will collectively be denoted 
$\mathcal{R} = \{\vec{r}_1, \vec{r}_2, \vec{s}_1, \vec{s}_2\}$. 
For now, we assume we can find frequency bins such that the piece-wise linear interpolations 
in Eq.~\eqref{eqn:l-ratio} and Eq.\eqref{eqn:c-ratio} are 
valid. How such bins can be found will be explained in Sec.~\ref{sec:bin-selection}.

Once the frequency bins are made based on the chosen fiducial waveforms, we can pre-compute some 
\textit{summary data}~\cite{Leslie:2021ssu} based only on the reference waveform and useful for the 
computation of the inner products in Eq.~\eqref{eqn:short-l}. These are:
\begin{eqnarray}
&& W_{l, m}(b) = \frac{4}{T} \sum_{f\in b}\frac{d(f)h'^{L*}_{l, m}(f)C'^{*}_{l, m}(f)}{S_n(f)} \, , 
\nonumber \\
&& X_{l, m}(b) = \frac{4}{T} \sum_{f\in b}\frac{d(f)h'^{L*}_{l, m}(f)C'^*_{l, m}(f)}{S_n(f)}(f-f_c(b)) \,, 
\nonumber\\
&& Y_{l, m, \bar{l}, \bar{m}}(b) = \frac{4}{T} \sum_{f\in b}\frac{h'^L_{l, m}(f)C'_{l, m}(f)h'^{L*}_{\bar{l}, \bar{m}}(f)C'^*_{\bar{l}, \bar{m}}(f)}{S_n(f)} \,, 
\nonumber\\
&&Z_{l, m, \bar{l}, \bar{m}}(b)  \nonumber\\
&&= \frac{4}{T} \sum_{f\in b}\frac{h'^L_{l, m}(f)C'_{l, m}(f)h'^{L*}_{\bar{l}, \bar{m}}(f)C'^*_{\bar{l}, \bar{m}}(f)}{S_n(f)}(f-f_c(b))\,.
\nonumber\\
\end{eqnarray}

Note that the summary data 
$\mathcal{D} = \{W_{l, m}, X_{l, m}, Y_{l, m, \bar{l}, \bar{m}}, Z_{l, m, \bar{l}, \bar{m}} \}$ 
need to be computed only once per analysis and they depend only on the choice of the fiducial 
parameters $\vec{\theta}'$. On the other hand, the coefficients $\mathcal{R}$ need to be computed for every 
proposal waveform as they depend on $\vec{\theta}$ and $\vec{\theta}'$.  

Using the summary data $\mathcal{D}$ and the coefficients $\mathcal{R}$, we can estimate $\ln\mathcal{L}_{dh}$ and $\ln\mathcal{L}_{hh}$ in Eq.~\eqref{eqn:short-l} 
for given $\vec{\theta}$ as


\begin{align}
\label{eqn:dh}
\ln\mathcal{L}_{dh} &= \Re \sum_{(l, m)} \sum_{b}  \left\{  W_{l, m}(b)\left[r_{1, l, m}^*(b) s_{1, l, m}^*(b)\right] +X_{l, m}(b)\left[r_{1, l, m}^*(b)s_{2, l, m}^*(b) + r_{2, l, m}^*(b) s_{1, l, m}^*(b) \right]  \right\} \,, \\
\label{eqn:hh}
\ln\mathcal{L}_{hh} &=\Re  \sum_{(l, m), (\bar{l}, \bar{m})} \sum_{b} \left\{  Y_{l, m, \bar{l}, \bar{m}}(b) \left[ r_{1, l, m}(b)s_{1, l, m}(b) r^*_{1, \bar{l}, \bar{m}}(b) s^*_{1, \bar{l}, \bar{m}}(b) \right] \right. \nonumber \\
&\qquad\qquad\qquad\qquad +  \left. Z_{l, m, \bar{l}, \bar{m}}(b) \left[r_{1, l, m}(b) r_{1, \bar{l}, \bar{m}}^*(b) \left( s_{1, \bar{l}, \bar{m}}^*(b) s_{2, l, m}(b) + s_{1, l, m}(b) s_{2, \bar{l}, \bar{m}}^*(b) \right) \right. \right. \nonumber \\ 
&\qquad\qquad\qquad\qquad + \left. \left.  s_{1, l, m}(b)s_{1, \bar{l}, \bar{m}}^*(b)\left(r_{1, \bar{l}, \bar{m}}^*(b) r_{2, l, m}(b) + r_{1, l, m}(b) r_{2, \bar{l}, \bar{m}}^*(b)\right) \right] \right\},
\end{align}

which together with Eq.~\eqref{eqn:short-l} enables us to approximate the likelihood. Provided the number of bins is reduced compared to the usual method, this approach should offer a rapid likelihood evaluation, leading to a rapid parameter inference.

\subsection{Constructing the relative binning grid}
\label{sec:bin-selection}

The objective of the bin selection step is to identify a minimal set of frequency bins in which the ratios of proposal waveform components 
($\vec{h}^L_{l, m}$ and $\vec{C}_{l, m}$) to fiducial waveform components are linear to sufficient approximation.
We refer to a set of frequency bins which satisfy this condition as an RBGrid. The RBGrid is used to compute the summary data $\mathcal{D}$, and, once the sampling begins, 
to compute coefficients $\mathcal{R}$ for the points in parameter space proposed by the sampler.  

To initialize the bin selection algorithm, we need the following quantities: 
a uniform frequency grid with spacing $\Delta f = 1/T$, where the time $T$ is sufficiently long so as to easily accommodate the length of the signal; 
a fixed value of ``total error'' on log likelihood, $\epsilon$; 
an initial value $N_0$ for the total number of frequency bins;  
a set of fiducial waveform parameters; and a set of test parameters. For definiteness we set  
$\epsilon = 0.01$ and $N_0 = 200$. In simulations, the fiducial parameters are set to the injection parameters, while for real events they are set to the 
maximum likelihood parameters obtained from earlier analyses. To choose a set of test parameters, we perturb the chirp mass and the 
mass ratio of the fiducial parameters by a random relative value in a $[ -10\%, 10\%]$ interval, and the rest of the parameters are kept the same. 

Using the quantities described in the paragraph above, we construct the RBGrid as follows. 
We start from the first bin of the uniform frequency grid. 
We merge the first bin with the next one and calculate the partial relative binning log likelihoods $\ln\mathcal{L}_{dh}$ and $\ln\mathcal{L}_{hh}$ 
defined in Eq.~\eqref{eqn:dh} and Eq.~\eqref{eqn:hh}, using as test quantities the edges of the combined frequency bin, and the fiducial waveform parameters. We also compute the corresponding exact partial likelihoods obtained by using all frequencies within the bins and not just the edges. 
If the absolute difference between the partial relative binning likelihoods and the exact partial likelihoods is less than $\epsilon/\sqrt{N_0}$, 
we add a subsequent bin into it and repeat. 
If the difference is larger than that, we do not add any more bins and move to the next bin and repeat. 
This continues until the maximum frequency of the uniform grid is reached. The value of $N_0$ is then updated to the total number of bins obtained 
through the above process, and the same procedure is started again with the same $\epsilon$ but the updated $N_0$. 
The algorithm terminates when $N_0$ ceases to change, and the edges of the merged bins in the final iteration become the RBGrid. 

The choice of $\epsilon$ controls the trade-off between speed and accuracy, where a larger value increases speed but reduces accuracy. By contrast, we have verified
that the bin selection procedure is not very sensitive to the initial value of $N_0$; it mainly affects the convergence rate of the algorithm. 
Setting $N_0$ equal to the length of the uniform grid would eliminate the need for this additional tuning parameter, but this would slow down the bin selection
algorithm as a whole\footnote{Our implementation of the bin selection and relative binning parameter estimation framework can be found in the git repository at~\cite{janquart:2022}.}. 
\begin{figure}[h]
    \centering
    \includegraphics[width=0.75\textwidth]{../figures/vis_overlap_signals.pdf} 
    \hspace*{.5cm}\includegraphics[width=0.75\textwidth]{../figures/lensing_overlap_signals.pdf}
    \caption{Top: An example of two gravitational wave signals merging close to each other. Bottom: An example of a gravitational wave signal that is strongly lensed producing two images. The fiducial time separation between the two signals marked by $\Delta t_{12}$ and relative (de-)magnification marked by $\mu_{12}$.}
    \label{fig:lensing-overlaps}
\end{figure}


\section{Extension of the relative binning method}
A fast parameter estimation method has a wide range of application. Here, we focus on two of them, overlapping signals analysis and strong lensing searches. Specifically, we extend our method to perform parameter estimation on two overlapping signals and to perform strong lensing searches. The two applications are similar in the sense that they both require a framework that can perform \textit{joint} parameter estimation on two siganls. Figure~\ref{fig:lensing-overlaps} gives an illustration of two overlapping signals (top) and strongly lensed signals (bottom). 

When two or more signals merge close to each other, we refer to them as overlapping signals (top panel of Figure~\ref{fig:lensing-overlaps}). It is intutive to think that if two signals are far apart, we can analyze each of them while ignoring the presence of the other signal. However, when they are close, they need to be analyze jointly in order to avoide biases in the measurements. A comprehansive investigation on this topic was done in the Tomek et al [in preparation]. Tomek et al extend the relative binning method developed in this work to perform \textit{joint} parameter estimation on two GW signals. 

When gravitational waves encounter a massive object in their path, they may undergo gravitational lensing, i.e., suffer a deflection from their original path. In the case of strong lening, the GW signals is split into multiple time-separated copies (referred to as images) which maybe observed at the detectors. The imprint of strong lensing on a GW signal can be characterized with three parameters: a relative magnification ratio ($\mu_{12}$), a time delay ($\Delta t_{12}$), and a relative phase shift ($n_{12}$). Strong lensing does not affect the frequency evolution of the source. Therefore, we can search for strong lensing by quantifying a degree of coherence between the parameters of any two GW signals, i.e. by again performing \textit{joint} parameter estimation. 

\subsubsection{To perform parameter estimation on the overlapping signals}
\iffalse
The LIGO-Virgo interferometers have not yet observed any overlapping signals and it is unlikely that we do so at current sensitivities. Therefore, the state-of-the-art parameter estimation pipeline are designed to analyze a single, isolated signal. However, in the 3G-era, due to increase detection rate and increase duration of GW signals, we may often encounter overlapping GW signals. Therefore, it is crucial to develope parameter estimation methods to analyze overlapping signals. 
\fi

We account for two signals in the signal model by replacing $\vec{h}$ by $\vec{h}_1 + \vec{h}_2$ in the expression of the likelihood (cc. Eq.~\eqref{eqn:indi-likelihood}). This leads to
\begin{equation}
\ln \mathcal{L}(\vec{\theta}_1, \vec{\theta}_2) = -\frac{1}{2}\left<\vec{d}-\vec{h}_1(\vec{\theta}_1) - \vec{h}_2(\vec{\theta}_2) |\vec{d}-\vec{h}_1(\vec{\theta}_2) - \vec{h}_2(\vec{\theta}_2)\right>\,,
\end{equation}
where the set $\{\vec{\theta}_1, \vec{\theta}_2\}$ now represent a 30 dimensional parameter space. Follwing the steps from Eq.~\eqref{eqn:indi-likelihood}-Eq.~\eqref{eqn:short-l}, we can factorise the likelihood into
\begin{align}
    \ln \mathcal{L}(\vec{\theta}) &= \left<\vec{d}|\vec{h}_1\right> - \frac{1}{2}\left<\vec{h}_1|\vec{h}_1\right> + \left<\vec{d}|\vec{h}_2\right> - \frac{1}{2}\left<\vec{h}_2|\vec{h}_2\right>, \nonumber \\
    \label{eq:two-likelihood}
    & = \ln \mathcal{L}_{dh_1} - \frac{1}{2} \ln \mathcal{L}_{h_1h_1} + \ln \mathcal{L}_{dh_2} - \frac{1}{2} \ln \mathcal{L}_{h_2h_2} \,.
\end{align}
where we have ignored the overlaps between $\vec{h}_1$ and $\vec{h}_2$ under the assumption that the two signals are sufficiently different, resulting in negligible overlaps. Now, we have all the ingredients to compute the joint likelihood. Each term in factorised likelihod expression of the Eq.~\ref{eq:two-likelihood} can be calculated using the relative binning approximation. To construct the RBGrid, we repeat the bin selection algorithm from section~\ref{sec:bin-selection} two times; once for each signal. 

\subsubsection{To perform strong lensing searches}
For strong lensing searches, we again need to replace $\vec{h}$ by $\vec{h}_1 + \vec{h}_2$ to account for two images. However, since two images are copies of the same GW signals, we expect that the source parameters are identical between the two except for the merger time, luminosity distance, and morse phase. Therefore, we only need to introduce three additional parameters. We denote them by $\vec{\theta}_L\equiv \{\Delta t_{12}, \mu_{12}, \Delta n_{12}\}$. The signal model corresponding to the second image $\vec{h}_2$ can be expressed in terms of the $\vec{h}_1$ and $\vec{\theta}_L$ as

\begin{equation}
\label{eqn:wf-mapping}
\vec{h}_2(\vec{\theta}, \vec{\theta}_{L}) = \sqrt{\mu_{12}}\vec{h}_1(\vec{\theta})e^{2i\pi f \Delta t_{12} - i\Delta n_{12}\pi}
\end{equation}
where the set $\{\vec{\theta}, \vec{\theta}_L\}$ now represent a 18 dimensional parameter space. The factorization of the likelihood expression is slightly different from the case of overlapping signals since we need to incorporate two data stream. Let us denote them by $\vec{d}_1$ and $\vec{d}_2$ for each image. The likelihood function then is,

\begin{align}
    \ln \mathcal{L}(\vec{\theta}, \vec{\theta}_L) &= \left<\vec{d}_1|\vec{h}_1\right> - \frac{1}{2}\left<\vec{h}_1|\vec{h}_1\right> + \left<\vec{d}_2|\vec{h}_2\right> - \frac{1}{2}\left<\vec{h}_2|\vec{h}_2\right>, \nonumber \\
    \label{eq:lesing_small_likelihood}
    & = \ln \mathcal{L}_{d_1h_1} - \frac{1}{2} \ln \mathcal{L}_{h_1h_1} + \ln \mathcal{L}_{d_2h_2} - \frac{1}{2} \ln \mathcal{L}_{h_2h_2} \,.
\end{align}

Note that the $\vec{d}$ is now replaced by $\vec{d}_1$ and $\vec{d}_2$ at relevant places. Next we construct the RBGrid only for image 1. Compute image 2 waveform using the Eq.~\eqref{eqn:wf-mapping}. Use RBGrid to generate the summary data and using the summary data, we can approximate the likelihood. 

Now, we have all the ingredients to perform paramter estimation on the isolated signals, overlapping signals, and strongly lensed signals. 
    

\section{Validation of the method}
\subsection{On simulated events}
\begin{table}
    \centering
    \begin{tabular}{| p{8cm} | p{5cm}|}
      \hline
      \textbf{Parameter} & \textbf{Population Prior} \\
      \hline
      Chirp mass $(\mathcal{M}_c)$ &  Uniform($5M_{\odot},$ $100M_{\odot}$) \\
      \hline
      Mass ratio $(q)$ & Uniform(0.1, 10) \\
      \hline
     Dimensionless spin of the $i^{th}$ black hole $(a_i)$  & Uniform(0, 1) \\
      \hline
      Zenith angle between the spin and orbital angular momentum for 
      $i^{th}$ black hole $(\theta_i )$ &Sine($0, \pi$)\\
      \hline
       Difference between the azimuthal angles of the individual spin vector projections onto the orbital plane $(\Delta \phi)$ & Uniform($0, 2\pi$)\\
      \hline
      Difference between total and orbital angular momentum azimuthal angles $(\phi_{JL})$ &  Uniform($0, 2\pi$) \\
      \hline
      Right Ascension (RA) &  Uniform($0, 2\pi$)\\
      \hline
      Declination (DEC) &  Cosine($-\pi/2,$ $\pi/2$)\\
      \hline
     Angle between the line of sight and total angular momentum$ (\theta_{JN})$ &  Sine($0, \pi$)\\
      \hline
      Polarization angle $(\psi)$ &  Sine($0, \pi$)\\
      \hline
     Phase of coalescence $(\phi_c)$ &  Uniform($0, 2\pi$)\\
      \hline 
      Luminosity distance $(d_L)$ &  UniformSourceFrame(0.1 \text{Gpc}, 5 \text{Gpc}) \\
      \hline
      Time of coalescence $(t_c)$ &  Uniform(0s, 86400s) \\
      \hline
      Morse factor for image 1 $(n_1)$ &  Uniform\{0, 0.5, 1\} \\
      \hline
      Magnification of image 2 w.r.t image 1 $(\mu_{12})$ &  Uniform(0, 10) \\
      \hline
      Delay in arrival time between two images $(\Delta t_{12})$ &  Uniform(0s, 86400s) \\
      \hline
      Difference in Morse factor $(\Delta n_{12})$ &  Uniform\{0, 0.5, 1\} \\
      \hline
       \end{tabular}
      \caption{Parameters and corresponding priors used to simulate individual GW signals and the pairs of lensed GW signals. 
      For a detailed description of each prior probability distribution refer to~\cite{Ashton2018Bilby:Astronomy}.}
       \label{bbh_params}
  \end{table}
  
For a parameter estimation method to be trustworthy, it should recover the injected parameters within a given confidence interval for a corresponding fraction of injections; the 
extent to which it does so can be assessed using a so-called P-P plot~\cite{Sidery:2013zua, Veitch:2014wba, Berry:2014jja, Singer:2015ema, Romero-Shaw:2020owr}. 
To this end we simulate a population of 70 BBH signals by sampling the appropriate source parameters from the prior distributions specified in Table~\ref{bbh_params}. 
We adjust the luminosity distance in such a way that each event has a network SNR above 13. 
The network consists of the two LIGO~\cite{LIGOScientific:2014pky} and the Virgo interferometers~\cite{VIRGO:2014yos} at their design sensitivity~\cite{KAGRA:2013rdx}. 
We perform parameter estimation on each event and show the collective results in Fig.~\ref{fig:pp-plot}, representing a P-P plot for our relative binning framework. Ideally, each coloured 
line should trace the diagonal; however, some fluctuations are expected due to a finite number of signals being analyzed and the presence of noise. 
We perform a Kolmogorov–Smirnov (KS) test~\cite{2020SciPy-NMeth} and quote the p-value (in brackets) as a measure of the consistency between each colored line and the diagonal line.
To further quantify this consistency, we can define a binomial random variable $X$ as the number of times the injected value is recovered within 
a confidence interval corresponding to a value on the horizontal axis. 
For this binomial distribution, the shape parameter $N$ equals 70 and $p$ equals the confidence corresponding to the x-axis.  
The shaded regions in Fig.~\ref{fig:pp-plot} cover 1$\sigma$, 2$\sigma$, and 3$\sigma$ confidence intervals of the binomial probability distribution in decreasing order of opacity. 
As all the plotted curves for the different parameters fall within the 3$\sigma$ boundary and the accompanying p-values indicate good consistency with the diagonal, 
we conclude that our method is robust.
\begin{figure}[h]
    \centering
    \includegraphics[width=.75\textwidth]{../figures/pp_3.0_partial_no_railing.pdf}
    \caption{Percentile-percentile showcasing the robustness of relative binning method on BBH events. Each line traces the diagonal, indicating that the parameter corresponding to it is recovered with expected accuracy. The numbers in the brackets of the legend show the p-values of the KS tests. The combined p-value of all parameters is 0.6625, consistent with the hypothesis that individual p-values were derived from a uniform distribution as expected. The shaded regions show 1$\sigma$, 2$\sigma$, 3$\sigma$ confidence intervals in decreasing order of opacity.}
    \label{fig:pp-plot}
\end{figure}

\subsection{On real data: analysing events detected by LIGO-Virgo}
\begin{figure}[h]
    \centering
    \includegraphics[width=.75\textwidth]{../figures/overlap_production_gw190412_gwtc.pdf}
    %\includegraphics[width=.75\textwidth]{../figures/overlap_production_gw190814_corr_branch_gwtc.pdf}
    \caption{Analysis of GW190412. The blue colour marks the results obtained from relative binning method and grey colour marks the results reported in GWTC-3 data release.}
    \label{fig:gwtc3-events}
\end{figure}

\begin{figure}[h]
    \centering
    \includegraphics[width=.75\textwidth]{../figures/overlap_production_gw190814_corr_branch_gwtc.pdf}
    \caption{Analysis of GW190814. The blue colour marks the results obtained from relative binning method and grey colour marks the results reported in GWTC-3 data release.}
    \label{fig:gw190814}
\end{figure}


\begin{figure}[h]
    \centering
    \includegraphics[width=0.53\textwidth]{../figures/thesis_pair_191103_191105.pdf} \hspace{.5cm}
    \includegraphics[width=0.41\textwidth]{../figures/lens_pair_191103_191105.pdf}
    \caption{Investigating the pair GW191103 (red) and GW191105 (green) under strong lensing hypothesis (blue).}
    \label{fig:lensed_pair}
\end{figure}

\subsection{Third generation events}

Now we turn to parameter estimation for the case of 3G observatories. 
In order to test our method, we analyze a subset of GWTC-3 events as they would be seen by these detectors. 
We arbitrarily choose six events with varying total masses from the GWTC-3 catalog, specified in table~\ref{tab:gwtc-3-table}. 

\begin{table}[h]
    \centering
    \begin{tabular}{|c|c|c|}
        \hline
        \textbf{Event} & \textbf{Total Mass ($M_{\odot}$)} & \textbf{Network SNR} \\
        \hline
        GW191126\_115259 & 26.51 & 192.52 \\
        \hline
        GW191127\_050227 & 138.71 &  260.33 \\
        \hline
        GW191230\_180458 &  144.69 & 383.10 \\
        \hline
        GW200112\_155838 & 78.10 & 1103.53 \\
        \hline
        GW200220\_061928 &  282.50 & 224.91 \\
        \hline
        GW200306\_093714 & 62.10 &  326.02 \\
        \hline
    \end{tabular}
    \caption{Table of events from GWTC-3, their (maximum-likelihood, observed) total masses, and their respective network SNRs as obtained for a network of 3G detectors.}
    \label{tab:gwtc-3-table}
\end{table}

We inject the signal corresponding to the maximum likelihood parameters, as obtained from the GWOSC data release, into Gaussian noise and perform parameter estimation with 
the relative binning framework. 
Our detector network consists of a triangular ET~\cite{Punturo:2010zz, Maggiore2019ScienceTelescope, Branchesi:2023mws} with an arm length of 10 km and one L-shaped CE~\cite{Reitze:2019iox} with an arm length of 40 km.
The noise curves and detector configurations are read from~\cite{threeGdesign}. 
A representative, truncated corner plot for this analysis is shown in Fig.~\ref{3g-event} for the GW200220\_061928 event. 
The orange lines in the plot show the injected parameter values and the green dashed lines indicate $1\sigma$ confidence intervals. 
The $1\sigma$, $2\sigma$, and $3\sigma$ confidence contours are indicated by the green shading, in order of decreasing opacity. 
As can be seen from the plot, all injected parameters have been recovered well. 

To quantify the accuracy of our method for the 3G scenario, we repeat the steps which were performed in subsection~\ref{sss:hlv-sensitivity}.
We again reweight the posterior samples obtained using the relative binning method following the procedure in Appendix~\ref{reweight} and 
calculate the JS divergence between the reweighted samples and the relative binning samples. 
This process is repeated for each of the six events and the results are shown in Fig.~\ref{three-g-js}. 
Each event is represented by a different marker and the dashed black line shows the threshold value, which is calculated in the same way as in~\ref{sss:hlv-sensitivity}. 
Here too the JS divergence for each parameter remains well below the threshold, indicating that the posterior distributions produced by the relative binning method and the 
regular method are statistically indistinguishable.


\begin{figure}[h]
    \centering
    \includegraphics[width=.75\textwidth]{../figures/prod_third_gen.pdf}
    \caption{Third generation injection example}
    \label{fig:third-gen}
\end{figure}


\begin{figure}[h]
    \centering
    \includegraphics[width=.72\textwidth]{../figures/overlapping_signals_A.pdf}\vspace{.5cm}
    \includegraphics[width=.72\textwidth]{../figures/overlapping_signals_B.pdf}
    \caption{Analysis of overlapping signals using joint parameter estimation}
    \label{fig:overlapping-signal}
\end{figure}

\section{Comparison of speed-up and accuracy}

\begin{figure}[h]
    \centering
    \includegraphics[width=.5\textwidth]{../figures/thesis_speed_up_summary.pdf}
    \caption{Relation between the speed up achieved by the the method in relation to the total mass}
    \label{fig:speed}
\end{figure}




Brief introduction on why we need fast PE
\section{Mode-by-mode relative binning}
Description of the method

\section{Application of the method}
\subsection{On simulated data}
\begin{figure}
    \centering
    \includegraphics[width=.75\textwidth]{../figures/pp_3.0_partial_no_railing.pdf}
    \caption{Percentile-percentile showcasing the robustness of relative binning method on BBH events. Each line traces the diagonal, indicating that the parameter corresponding to it is recovered with expected accuracy. The numbers in the brackets of the legend show the p-values of the KS tests. The combined p-value of all parameters is 0.6625, consistent with the hypothesis that individual p-values were derived from a uniform distribution as expected. The shaded regions show 1$\sigma$,2$\sigma$,3$\sigma$ confidence intervals in decreasing order of opacity.}
    \label{fig:pp-plot}
\end{figure}

\subsection{On real data: analysing events detected by LIGO-Virgo}
\begin{figure}
    \centering
    \includegraphics[width=.75\textwidth]{../figures/overlap_production_gw190412_gwtc.pdf}
    %\includegraphics[width=.75\textwidth]{../figures/overlap_production_gw190814_corr_branch_gwtc.pdf}
    \caption{Analysis of GW190412. The blue colour marks the results obtained from relative binning method and grey colour marks the results reported in GWTC-3 data release.}
    \label{fig:gwtc3-events}
\end{figure}

\begin{figure}
    \centering
    \includegraphics[width=.75\textwidth]{../figures/overlap_production_gw190814_corr_branch_gwtc.pdf}
    \caption{Analysis of GW190814. The blue colour marks the results obtained from relative binning method and grey colour marks the results reported in GWTC-3 data release.}
    \label{fig:gw190814}
\end{figure}


\begin{figure}
    \centering
    \includegraphics[width=0.53\textwidth]{../figures/thesis_pair_191103_191105.pdf} \hspace{.5cm}
    \includegraphics[width=0.41\textwidth]{../figures/lens_pair_191103_191105.pdf}
    \caption{Investigating the pair GW191103 (red) and GW191105 (green) under strong lensing hypothesis (blue).}
    \label{fig:lensed_pair}
\end{figure}

\subsection{Third generation events}

\begin{figure}
    \centering
    \includegraphics[width=.75\textwidth]{../figures/prod_third_gen.pdf}
    \caption{Third generation injection example}
    \label{fig:third-gen}
\end{figure}


\begin{figure}
    \centering
    \includegraphics[width=.72\textwidth]{../figures/overlapping_signals_A.pdf}\vspace{.5cm}
    \includegraphics[width=.72\textwidth]{../figures/overlapping_signals_B.pdf}
    \caption{Analysis of overlapping signals using joint parameter estimation}
    \label{fig:overlapping-signal}
\end{figure}

\section{Comparison of speed-up}

\begin{figure}
    \centering
    \includegraphics[width=.75\textwidth]{../figures/thesis_speed_up_summary.pdf}
    \caption{Relation between the speed up achieved by the the method in relation to the total mass}
    \label{fig:speed}
\end{figure}


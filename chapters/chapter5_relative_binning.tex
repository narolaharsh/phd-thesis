Bayesian inference discussed in chapter~\ref{ch:gwpe} is the key to achieving a wide range of scientific insights from gravitational wave signals. The cost of the inference increases with the complexity of the signal model and the signal's duration. Currently, parameter inference on one signal takes $\mathcal{O}(1)$ days. Additionally, for the third-generation detectors, due to the longer duration and increased detection rate of the signal, the cost of parameter inference could quickly become intractable. To address these issues, we present a fast parameter estimation method that includes higher-order modes and precession using relative binning. These effects are important in the context of GW signals from binary black hole mergers, as the inclusion of them provides a more complete description of the phenomenon. Additionally, we extend the method to analyse isolated signals as well as overlapping signals in the 3G era. We also adapt our method to perform strong lensing searches. Indeed, a fast parameter estimation method has a wide range of applications. Specifically, using our method, we analyse

\begin{enumerate}
     \item A set of simulated BBH signals covering the full parameter space to establish validity. 
     \item Two LIGO-Virgo events with significant contribution from higher order modes: GW190412~\cite{LIGOScientific:2020stg} and GW190814~\cite{LIGOScientific:2020zkf}.
     \item A set of simulated GW signals for the 3G-detectors, isolated as well as overlapping.
     \item The event pair GW191103--GW191105 as well as a set of simulated pairs, to investigate the strong lensing hypothesis.
 \end{enumerate}
 
 
\section{Mode-by-mode relative binning}
From chapter~\ref{ch:gwpe}, let us recall the (log) likelihood function,
\begin{equation}
\label{eqn:indi-likelihood}
\ln \mathcal{L}(\vec{\theta}) = -\frac{1}{2}\left<\vec{d}-\vec{h}(\vec{\theta})|\vec{d}-\vec{h}(\vec{\theta})\right>\,,
\end{equation}
where $\vec{h}(\vec{\theta})$ represents the GW waveform as a function of the source parameters $\vec{\theta}$ and $\left<\,.\,|\,.\,\right>$ 
denotes the noise-weighted inner product defined in chpater~\ref{ch:gwpe}~(Eq.~\eqref{eq:nwe}). Expanding the inner product in Eq.~\eqref{eqn:indi-likelihood}, we get:

\begin{equation}
\label{eqn:logL}
\ln \mathcal{L}(\vec{\theta}) = \left<\vec{d}|\vec{h}\right> - \frac{1}{2}\left<\vec{h}|\vec{h}\right> - \frac{1}{2}\left<\vec{d}|\vec{d}\right>.
\end{equation}
Likelihood-based inference involves calculating the inner products for a large number of values for $\vec{\theta}$ to explore the parameter space, which is computationally expensive. A rapid way to compute the these quantities could enable a faster inference. 

Let us introduce the notations:
\begin{align}
 \ln \mathcal{L}_{dh} &\equiv \left<\vec{d}|\vec{h}\right>, \\
 \ln \mathcal{L}_{hh} &\equiv \left<\vec{h}|\vec{h}\right>, 
\end{align}
to rewrite Eq.~\eqref{eqn:logL} as;
\begin{equation}
\label{eqn:short-l}
\ln \mathcal{L}(\vec{\theta}) = \ln \mathcal{L}_{dh} - \frac{1}{2} \ln \mathcal{L}_{hh}\,.
\end{equation}
Note that we dropped the last term of Eq.~\eqref{eqn:logL} in the equation above. It is also known as the noise log likelihood. Since it does not depend on the source parameters, it needs to be computed once for the data.

\subsubsection{Dominant mode and aligned spin}
When performing parameter estimation, we are interested in knowing the shape of the likelihood around its peak. In other words, we want to evaluate the likelihood 
for waveforms that are {\it similar} to the waveform at which the likelihood peaks. 
If we choose a fiducial value of $\vec{\theta}$ which is close enough to the peak, we can approximate the likelihood by expanding it around the fiducial value. 
Let us denote the fiducial source parameters by $\vec{\theta}'$ and the waveform at the 
fiducial parameters by $\vec{h}'$. For any proposal waveform $\vec{h}$, 
under the assumption that $\vec{h}$ and $\vec{h}'$ are close enough, we can find 
frequency bins $b =  [f_{min}, f_{max}]$ such that the ratio 
$\vec{h}/\vec{h}'$ can be linearly approximated as:
\begin{equation}
\label{eqn:ratio}
\frac{h}{h'}(f) = r_1(b) + r_2(b)(f-f_c(b))+\mathcal{O}(f^2), 
\end{equation}
within a given bin, where $r_1(b)$ and $r_2(b)$ are expansion coefficients 
and $f_c$ denotes the central value of the frequency bin $b$. 

The approximation made in Eq.~\eqref{eqn:ratio} performs better for the waveforms that only include the dominant mode (also known as the (2, 2) mode) and aligned spins description. In the phenomenological waveform family, they are known as IMRPhenomD waveforms. For more complete waveforms, i.e, the ones including the effects of higher order modes and precessing spins (known as IMRPhenomXPHM), the ratio $\vec{h}/\vec{h}'$ fluctuates rapidly as a function of frequency (see left plot of Figure~\ref{fig:dxp-ratio}). When using the IMRPhenomD waveform, a few bins are sufficient to make the linear expansion from Eq.~\eqref{eqn:ratio} precise in each bin. When using the IMPRhenomXPHM waveform, as the ratio is subject to more fluctuation, we need to consider a larger number of bins, increasing the waveform generation cost which is the most expensive operation in the calculation of $\ln \mathcal{L}_{dh}$ and $\ln \mathcal{L}_{hh}$.
\begin{figure}
    \centering
    \includegraphics[width=.4\textwidth]{../figures/phenom_d_xphm_full_waveform_ratio.pdf}\hspace{.5cm}
    \includegraphics[width=.4\textwidth]{../figures/hL_C_ratio.pdf}
    \caption{\textit{Left}: Ratio of the real part of the proposal to the fiducial waveform for the plus polarisation as a function of frequency. The waveform model corresponding to the brown curve contains only the dominant mode and the aligned spins description of the system. Whereas, the one corresponding to the black curve presents a more complete description, including contributions from higher-order modes and allowing for precessing spins (referred to as IMRPhenomXPHM). The ratio for the latter fluctuates more due to the increased complexity of the model. \textit{Right}: Decomposition of IMRPhenomXPHM waveform into component modes (top right) and precession coefficients (bottom right). Each mode is marked with a different colour. The decomposition makes the ratios smoother, allowing for an efficient implementation of the relative binning method.}
    \label{fig:dxp-ratio}
\end{figure}


\subsection{Inclusion of higher order modes and precessing spins}
To make the approximation in Eq.~\eqref{eqn:ratio} robust for precessing waveforms with higher order modes, 
let us express the detector frame waveform for given $\vec{\theta}$ into its 
component modes~\cite{Pratten:2020ceb};
\begin{equation}
h(f) = \sum_{l, m}(F_+C^{+}_{l, m}(f)+F_\times C^{\times}_{l, m}(f))\,h^L_{l, m}(f)\,.
\label{eqn:comp-mode}
\end{equation} 

Here, $F_+$ and $F_\times$ are the antenna pattern functions of the interferometer, which depend 
only on extrinsic parameters; $\vec{h}^{L}$ denotes the waveform in the co-precessing frame 
(also known as the $L$-frame); and the indices $l$ and $m$ label the different modes. The coefficients $C^{+}_{l, m}$ and $C^{\times}_{l, m}$ account for 
the {\it twisting-up} procedure, which transforms the waveform from a 
co-precessing frame to the inertial (or observer) frame~\cite{Pratten:2020ceb}. 
The benefit of handling each mode in the $L$-frame separately is that the ratio of a 
fiducial $\vec{h}'^{L}_{l, m}$ to a proposal $\vec{h}^{L}_{l, m}$ oscillates less compared to the 
ratio of the full waveforms $\vec{h}/\vec{h}'$. This can be seen by comparing the left plot of the
Figure~\ref{fig:dxp-ratio} with the two plots in the right panel of it. 
Therefore, a linear expansion similar to the one in Eq.~\eqref{eqn:ratio} 
can be made using the ratio $\vec{h}^{L}_{l, m}/\vec{h}'^L_{l, m}$ instead of 
$\vec{h}/\vec{h}'$ and will require a lower number of frequency bins;
\begin{equation}
\label{eqn:l-ratio}
\frac{h^{L}_{l, m}}{h'^L_{l, m}} (f) = r_{1, l, m}(b) + r_{2, l, m} (b)(f-f_c(b)) + \mathcal{O}(f^2) \,,
\end{equation}

Now, let us denote the factors $(F_+C^{+}_{l, m}(f)+F_\times C^{\times}_{l, m}(f))$ in 
Eq.~\eqref{eqn:comp-mode} by $\vec{C}_{l, m}$. We can perform similar expansions on the ratios 
$\vec{C}_{l, m}/\vec{C}'_{l, m}$:
\begin{equation}
\label{eqn:c-ratio}
\frac{C_{l, m}}{C'_{l, m}}(f) = s_{1, l, m}(b) + s_{2, l, m}(b)(f-f_c(b)) + \mathcal{O}(f^2).
\end{equation}
Below, the expansion coefficients will collectively be denoted 
$\mathcal{R} = \{\vec{r}_1, \vec{r}_2, \vec{s}_1, \vec{s}_2\}$. 
For now, we assume we can find frequency bins such that the piece-wise linear interpolations 
in Eq.~\eqref{eqn:l-ratio} and Eq.~\eqref{eqn:c-ratio} are 
valid. I explain how we construct such bins in section~\ref {sec:bin-selection}.

Once the frequency bins are made based on a fiducial waveform, we can pre-compute some 
\textit{summary data} which is useful for the computation of the inner products in Eq.~\eqref{eqn:short-l}~\cite{Leslie:2021ssu}. Specifically, the expressions for the summary data are:
\begin{eqnarray}
&& W_{l, m}(b) = \frac{4}{T} \sum_{f\in b}\frac{d(f)h'^{L*}_{l, m}(f)C'^{*}_{l, m}(f)}{S_n(f)} \, , 
\nonumber \\
&& X_{l, m}(b) = \frac{4}{T} \sum_{f\in b}\frac{d(f)h'^{L*}_{l, m}(f)C'^*_{l, m}(f)}{S_n(f)}(f-f_c(b)) \,, 
\nonumber\\
&& Y_{l, m, \bar{l}, \bar{m}}(b) = \frac{4}{T} \sum_{f\in b}\frac{h'^L_{l, m}(f)C'_{l, m}(f)h'^{L*}_{\bar{l}, \bar{m}}(f)C'^*_{\bar{l}, \bar{m}}(f)}{S_n(f)} \,, 
\nonumber\\
&&Z_{l, m, \bar{l}, \bar{m}}(b)  \nonumber\\
&&= \frac{4}{T} \sum_{f\in b}\frac{h'^L_{l, m}(f)C'_{l, m}(f)h'^{L*}_{\bar{l}, \bar{m}}(f)C'^*_{\bar{l}, \bar{m}}(f)}{S_n(f)}(f-f_c(b))\,.
\nonumber\\
\end{eqnarray}
Note that the summary data, collectively denoted by  $\mathcal{D} \equiv \{W_{l, m}, X_{l, m}, Y_{l, m, \bar{l}, \bar{m}}, Z_{l, m, \bar{l}, \bar{m}} \}$ 
need to be computed only once per analysis, and they depend only on the choice of the fiducial 
parameters $\vec{\theta}'$. On the other hand, the coefficients $\mathcal{R}$ need to be computed for every 
proposal waveform as they depend on $\vec{\theta}$ and $\vec{\theta}'$.  

Using the summary data $\mathcal{D}$ and the coefficients $\mathcal{R}$, we can estimate $\ln\mathcal{L}_{dh}$ and $\ln\mathcal{L}_{hh}$ in Eq.~\eqref{eqn:short-l} 
for given $\vec{\theta}$ as,
\begin{align}
\label{eqn:dh}
\ln\mathcal{L}_{dh} &= \Re \sum_{(l, m)} \sum_{b}  \left\{  W_{l, m}(b)\left[r_{1, l, m}^*(b) s_{1, l, m}^*(b)\right] +X_{l, m}(b)\left[r_{1, l, m}^*(b)s_{2, l, m}^*(b) + r_{2, l, m}^*(b) s_{1, l, m}^*(b) \right]  \right\} \,, \\
\label{eqn:hh}
\ln\mathcal{L}_{hh} &=\Re  \sum_{(l, m), (\bar{l}, \bar{m})} \sum_{b} \left\{  Y_{l, m, \bar{l}, \bar{m}}(b) \left[ r_{1, l, m}(b)s_{1, l, m}(b) r^*_{1, \bar{l}, \bar{m}}(b) s^*_{1, \bar{l}, \bar{m}}(b) \right] \right. \nonumber \\
&\qquad\qquad\qquad\qquad +  \left. Z_{l, m, \bar{l}, \bar{m}}(b) \left[r_{1, l, m}(b) r_{1, \bar{l}, \bar{m}}^*(b) \left( s_{1, \bar{l}, \bar{m}}^*(b) s_{2, l, m}(b) + s_{1, l, m}(b) s_{2, \bar{l}, \bar{m}}^*(b) \right) \right. \right. \nonumber \\ 
&\qquad\qquad\qquad\qquad + \left. \left.  s_{1, l, m}(b)s_{1, \bar{l}, \bar{m}}^*(b)\left(r_{1, \bar{l}, \bar{m}}^*(b) r_{2, l, m}(b) + r_{1, l, m}(b) r_{2, \bar{l}, \bar{m}}^*(b)\right) \right] \right\},
\end{align}
which together with Eq.~\eqref{eqn:short-l} enables us to approximate the likelihood. Provided the number of bins is reduced compared to the exact method, this approach should offer a rapid likelihood evaluation, leading to a faster parameter inference.

\subsection{Constructing the relative binning grid}
\label{sec:bin-selection}

The objective of the bin selection step is to identify a minimal set of frequency bins in which the ratios of proposal waveform components. 
($\vec{h}^L_{l, m}$ and $\vec{C}_{l, m}$) to fiducial waveform components are linear to a sufficient approximation.
We refer to a set of frequency bins which satisfy this condition as an RBGrid. The RBGrid is used to compute the summary data $\mathcal{D}$, and, once the sampling begins, 
to compute coefficients $\mathcal{R}$ for the points in parameter space proposed by the sampler.  

To initialise the bin selection algorithm, we need the following quantities: 
a uniform frequency grid with spacing $\Delta f = 1/T$, where the time $T$ is sufficiently long to accommodate the length of the signal easily; 
a fixed value of ``total error'' on log likelihood, $\epsilon$; 
an initial value $N_0$ for the total number of frequency bins;  
a set of fiducial waveform parameters; and a set of test parameters. For definiteness, we set  
$\epsilon = 0.01$ and $N_0 = 200$. In simulations, the fiducial parameters are set to the injection parameters, while for real events, they are set to the 
maximum likelihood parameters obtained from earlier analyses. To choose a set of test parameters, we perturb the chirp mass and the 
mass ratio of the fiducial parameters by a random relative value in a $[ -10\%, 10\%]$ interval, and the rest of the parameters are kept the same. 

Using the quantities described in the paragraph above, we construct the RBGrid as follows. 
We start from the first bin of the uniform frequency grid. 
We merge the first bin with the next one and calculate the partial relative binning log likelihoods $\ln\mathcal{L}_{dh}$ and $\ln\mathcal{L}_{hh}$ 
defined in Eq.~\eqref{eqn:dh} and Eq.~\eqref{eqn:hh}, using as test quantities the edges of the combined frequency bin, and the fiducial waveform parameters. We also compute the corresponding exact partial likelihoods obtained by using all frequencies within the bins and not just the edges. 
If the absolute difference between the partial relative binning likelihoods and the exact partial likelihoods is less than $\epsilon/\sqrt{N_0}$, 
we add a subsequent bin into it and repeat. 
If the difference is larger than that, we do not add any more bins and move to the next bin and repeat. 
This continues until the maximum frequency of the uniform grid is reached. The value of $N_0$ is then updated to the total number of bins obtained 
through the above process, and the same procedure is started again with the same $\epsilon$ but with the updated $N_0$. 
The algorithm terminates when $N_0$ ceases to change, and the edges of the merged bins in the final iteration become the RBGrid. 

The choice of $\epsilon$ controls the trade-off between speed and accuracy, where a larger value increases speed but reduces accuracy. By contrast, we have verified that the bin selection procedure is not very sensitive to the initial value of $N_0$; it mainly affects the convergence rate of the algorithm. 
Setting $N_0$ equal to the length of the uniform grid would eliminate the need for this additional tuning parameter, but this would slow down the bin selection
algorithm as a whole. 
\begin{figure}[h]
    \centering
    \includegraphics[width=0.75\textwidth]{.././figures/thesis_whitebg_graphics_overlapping_signals.pdf} 
    %\hspace*{.5cm}\includegraphics[width=0.75\textwidth]{../figures/lensing_overlap_signals.pdf}
    \caption{An illustration of two gravitational wave signals (blue and orange) arriving at the Earth at nearly the same time. At the Earth based GW detector, we can only measure the sum of the two (black line).}
    \label{fig:schem_overlaps}
\end{figure}

\section{Extension of the relative binning method}
A fast parameter estimation method has a wide range of applications. Here, we focus on two of them, overlapping signals analysis and strong lensing searches. Specifically, we extend the relative binning method to perform parameter estimation on overlapping signals and to perform strong lensing searches. The two applications are similar in the sense that they both require a framework that can perform \textit{joint} parameter estimation on two signals. 

\subsection{Extension to perform parameter estimation on the overlapping signals}
\iffalse
The LIGO-Virgo interferometers have not yet observed any overlapping signals, and it is unlikely that we will do so at current sensitivities. Therefore, the state-of-the-art parameter estimation pipelines are designed to analyse a single, isolated signal. However, in the 3G era, due to the increased detection rate and increased duration of GW signals, we may often encounter overlapping GW signals. Therefore, it is crucial to develop parameter estimation methods to analyse overlapping signals. 
\fi

Overlapping signals refer to a phenomenon where the arrival times of two (or more) GW signals are close to each other. Figure~\ref{fig:schem_overlaps} shows an illustration of two overlapping GW signals. While such scenarios are less likely for the LIGO-Virgo interferometers, they may occur frequently for 3G detectors. Due to their increased baseline sensitivity, the 3G detectors are expected to detect $\approx 1$ signal per minute, with longer in-band duration of the signals. These two factors, increased detection rate and increased in-band duration, may cause GW signals to overlap with each other. Current rate estimates suggest that we may see up to 20,000 cases per year where two GW signals are overlapping within $\approx 1$ second of each other~\cite{Samajdar:2021egv}. If the two signals are close to each other, we may need to analyse them jointly to measure their source parameters, i.e., we may need to perform \textit{joint} parameter estimation on them. Indeed, the joint parameter estimation on two signals can be computationally more expensive compared to analysing a single signal since the dimension of the space is doubled. Here, we demonstrate how relative binning can be used to speed up joint parameter estimation. 

We account for two signals in the signal model by replacing $\vec{h}$ by $\vec{h}_1 + \vec{h}_2$ in the expression of the likelihood (see Eq.~\eqref{eqn:indi-likelihood}). This leads to;
\begin{equation}
\ln \mathcal{L}(\vec{\theta}_1, \vec{\theta}_2) = -\frac{1}{2}\left<\vec{d}-\vec{h}_1(\vec{\theta}_1) - \vec{h}_2(\vec{\theta}_2) |\vec{d}-\vec{h}_1(\vec{\theta}_2) - \vec{h}_2(\vec{\theta}_2)\right>\,,
\end{equation}
where the set $\{\vec{\theta}_1, \vec{\theta}_2\}$ now represent a 30 dimensional parameter space. Following the steps from Eq.~\eqref{eqn:indi-likelihood}-\eqref{eqn:short-l}, we can factorise the likelihood into;
\begin{align}
 \ln \mathcal{L}(\vec{\theta}) &\approx \left<\vec{d}|\vec{h}_1\right> - \frac{1}{2}\left<\vec{h}_1|\vec{h}_1\right> + \left<\vec{d}|\vec{h}_2\right> - \frac{1}{2}\left<\vec{h}_2|\vec{h}_2\right>, \nonumber \\
    \label{eq:two-likelihood}
    & = \ln \mathcal{L}_{dh_1} - \frac{1}{2} \ln \mathcal{L}_{h_1h_1} + \ln \mathcal{L}_{dh_2} - \frac{1}{2} \ln \mathcal{L}_{h_2h_2} \,.
\end{align}
where we have ignored the overlaps between $\vec{h}_1$ and $\vec{h}_2$ under the assumption that the two signals are sufficiently different, resulting in negligible overlaps. Each term in the factorised likelihood expression of Eq.~\eqref{eq:two-likelihood} can be calculated using the relative binning approximation. To construct the RBGrid, we repeat the bin selection algorithm from section~\ref{sec:bin-selection} two times; once for each signal. 

\subsection{Extension to perform strong lensing searches}

\begin{figure}[h]
     \centering
     \includegraphics[width=0.75\textwidth]{.././figures/schem_lensing_graphics.pdf} 
     %\hspace*{.5cm}\includegraphics[width=0.75\textwidth]{../figures/lensing_overlap_signals.pdf}
     \caption{An illustration of a gravitational wave signal that is strongly lensed by a heavy object in its path and producing two images. Each image may be (de-)magnified and arrive at the Earth at different times. The fiducial time separation between the two images is marked by $\Delta t_{12}$, and relative (de-)magnification is marked by $\mu_{12}$. The precise values of them depend on the relative configuration of the GW source and the lens.}
     \label{fig:schem_lensing}
 \end{figure}

When gravitational waves encounter a massive object in their path, they may undergo gravitational lensing, i.e., suffer a deflection from their original path. In the case of strong lensing, the GW signal is split into multiple time-separated copies (referred to as images), which may be observed at the detectors. Figure~\ref{fig:schem_lensing} gives an illustration of a GW signal being strongly lensed by a massive object and producing two images. Each image produced by strong lensing may be (de-)magnified, may suffer a constant phase shift (also known as the Morse phase~\cite{Ezquiaga:2020gdt}), and may arrive at Earth at different times. Strong lensing does not affect the frequency evolution of the source. Therefore, we can perform strong lensing searches by quantifying how similar any two signals are, i.e., by jointly analysing two signals to measure the coherence between their source parameters. 

Similar to the joint parameters estimation of the overlapping signals, we need to account for two images, $\vec{h}_1$ and $\vec{h}_2$. However, we do not simply replace $\vec{h}$ by $\vec{h}_1+\vec{h}_2$. The two images are copies of the same GW signal; we expect that the source parameters are identical between the two, except for the arrival time at Earth, apparent luminosity distance, and Morse phase. Therefore, we only need to introduce three additional parameters. We denote them by $\vec{\theta}_L\equiv \{\Delta t_{12}, \mu_{12}, \Delta n_{12}\}$. The signal model corresponding to the second image $\vec{h}_2$ can be expressed in terms of the $\vec{h}_1$ and $\vec{\theta}_L$ as
\begin{equation}
\label{eqn:wf-mapping}
\vec{h}_2(\vec{\theta}, \vec{\theta}_{L}) = \sqrt{\mu_{12}}\vec{h}_1(\vec{\theta})e^{2i\pi f \Delta t_{12} - i\Delta n_{12}\pi},
\end{equation}
where the set $\{\vec{\theta}, \vec{\theta}_L\}$ now represent a 18 dimensional parameter space. The factorisation of the likelihood expression is slightly different from the case of overlapping signals since we need to incorporate two data segments, $\vec{d}_1$ and $\vec{d}_2$, one for each image. The full likelihood expression becomes~\cite{Lo:2021nae}; 
\begin{equation}
 p(\vec{d}_1, \vec{d}_2 | \vec{\theta}, \vec{\theta}_L) = p(\vec{d}_1 | \vec{\theta}) ~ p(\vec{d}_2 | \vec{\theta}, \vec{\theta}_L),
\end{equation}
which holds since two data segments are disjoint and $\vec{d}_1$ do not depend on the relative lensing parameters $\vec{\theta}_L$. We can factorise it into
\begin{align}
 \ln \mathcal{L}(\vec{\theta}, \vec{\theta}_L) &= \left<\vec{d}_1|\vec{h}_1\right> - \frac{1}{2}\left<\vec{h}_1|\vec{h}_1\right> + \left<\vec{d}_2|\vec{h}_2\right> - \frac{1}{2}\left<\vec{h}_2|\vec{h}_2\right>, \nonumber \\
    \label{eq:lesing_small_likelihood}
    & = \ln \mathcal{L}_{d_1h_1} - \frac{1}{2} \ln \mathcal{L}_{h_1h_1} + \ln \mathcal{L}_{d_2h_2} - \frac{1}{2} \ln \mathcal{L}_{h_2h_2} \,.
\end{align}
The overlaps between $\vec{h}_1$ and $\vec{h}_2$ do not appear here, unlike the overlapping signals analysis. Next, we construct the RBGrid only for image 1 and use the same grid to generate image 2 waveform (see Eq.~\eqref{eqn:wf-mapping}). We use the RBGrid to generate the summary data, and then, using the summary data, we can approximate the likelihood. 

Now, we have all the ingredients to perform parameter estimation on the isolated signals, overlapping signals, and strongly lensed signals using the relative binning method. Our aim here is to demonstrate that the relative binning approximation significantly decreases the computation cost without compromising on the accuracy. Before we perform the analysis and present the results let us discuss the quantities used to compare the accuracy of the exact method with the relative binning method.

\subsubsection{Checking the efficiency}
\label{reweight}
\textit{Reweighing the relative binning posterior samples:} Ideally, we want to compare the posterior samples obtained using exact likelihood and the relative binning likelihood. However, there may be cases when the exact computation too expensive to carry out and we can only peform the analysis with relative binning likelihood. For such cases, we reweight the posterior samples produced using relative binning to produce the samples which may be obtained with the exact likelihood computations. This step helps test the method's efficiency. Let us denote the likelihood evaluated for the posterior samples produced by the relative binning method by $\mathcal{L}^{\text{RB}}(\vec{\theta})$, where $\vec{\theta}$ stands for the posterior samples; and the probability for $\vec{\theta}$ according to the relative binning 
samples by $p^{\text{RB}}(\vec{\theta})$. To perform the reweighting, for each posterior sample $\vec{\theta}$, we compute the likelihood according to the exact method, which we denote by $\mathcal{L}^{\text{X}}(\vec{\theta})$. Using $\mathcal{L}^{\text{RB}}(\vec{\theta})$, $\mathcal{L}^{\text{X}}(\vec{\theta})$, 
we reweigh $p^{\text{RB}}(\vec{\theta})$ to obtain the probability for $\vec{\theta}$ according to the exact method:
\begin{equation}
p^{\text{X}}(\vec{\theta})  = \frac{\mathcal{L}^{\text{X}}(\vec{\theta})}{\mathcal{L}^{\text{RB}}(\vec{\theta})}p^{\text{RB}}(\vec{\theta})\,.
\end{equation}

\textit{Jensen-Shannon divergence:} Once we have the posterior samples from the exact method (either by performing exact analysis or by reweighting), we quantify the similarities between the two using the Jensen-Shannon (JS) divergence~\cite{JSD:61115}. The JS divergence yields a finite value between 0nat and 1nat. A value closer to 0nat correspond to greater similarity between two distributions. If we denote the two distributions we want compare by $p$ and $q$, then the JS divergence is given by
\begin{equation}
     \mathrm{JS}(p||q) = \frac{1}{2}\left(D(p||m) + D(q||m)\right),
\end{equation}
where $D$ is the Kullback-Leibler divergence~\cite{Kullback:1951zyt} and $m$ is the pointwise mean of $p$ and $q$. The JS divergence is symmetric in $p$ and $q$.

\textit{Kolmogorov–Smirnov test:} Alternatively, we also use the Kolmogorov–Smirnov (KS) test to perform sanity checks~\cite{2020SciPy-NMeth}. This test is useful to determine if the parameter estimation on a large set of simulated GW signals is not biased in any particular direction. If the parameter estimation on a large set of simulated signal is not biased, the percentile at which the true value is found should follow a uniform distribution over the population. In this regard, the KS test helps determine if the distribution of percentiles follows a uniform distribution. The KS test returns a $p$-value which can be used to reject the null hypothesis, i.e., the percentiles follow a uniform distribution. We use the standard routines from \texttt{SciPy} codebase to calculate the JS divergence and KS statistic.  


\section{Validation of the method by performing parameter estimation}
Next, we demonstrate the speed and accuracy with which our method can perform parameter inference. We analyse a large set of GW signals consisting of simulated signals, signals detected by LIGO-Virgo, lensed signals, and overlapping signals. 
\subsection{Validation on simulated events: analysing GW signals in Gaussian noise}
\begin{table}
    \centering
    \begin{tabular}{| p{8cm} | p{5cm}|}
      \hline
      \textbf{Parameter} & \textbf{Population Prior} \\
      \hline
 Chirp mass $(\mathcal{M}_c)$ & Uniform($5M_{\odot},$ $100M_{\odot}$) \\
      \hline
 Mass ratio $(q)$ & Uniform(0.1, 10) \\
      \hline
 Dimensionless spin of the $i^{th}$ black hole $(a_i)$  & Uniform(0, 1) \\
      \hline
 Zenith angle between the spin and orbital angular momentum for 
      $i^{th}$ black hole $(\theta_i )$ &Sine($0, \pi$)\\
      \hline
 Difference between the azimuthal angles of the individual spin vector projections onto the orbital plane $(\Delta \phi)$ & Uniform($0, 2\pi$)\\
      \hline
 Difference between total and orbital angular momentum azimuthal angles $(\phi_{JL})$ & Uniform($0, 2\pi$) \\
      \hline
 Right Ascension (RA) & Uniform($0, 2\pi$)\\
      \hline
 Declination (DEC) & Cosine($-\pi/2,$ $\pi/2$)\\
      \hline
 Angle between the line of sight and total angular momentum$ (\theta_{JN})$ & Sine($0, \pi$)\\
      \hline
 Polarization angle $(\psi)$ & Sine($0, \pi$)\\
      \hline
 Phase of coalescence $(\phi_c)$ & Uniform($0, 2\pi$)\\
      \hline 
 Luminosity distance $(d_L)$ & UniformSourceFrame(0.1 \text{Gpc}, 5 \text{Gpc}) \\
      \hline
 Time of coalescence $(t_c)$ & Uniform(0s, 86400s) \\
      \hline
 Morse factor for image 1 $(n_1)$ & Uniform\{0, 0.5, 1\} \\
      \hline
 Magnification of image 2 w.r.t image 1 $(\mu_{12})$ & Uniform(0, 10) \\
      \hline
 Delay in arrival time between two images $(\Delta t_{12})$ & Uniform(0s, 86400s) \\
      \hline
 Difference in Morse factor $(\Delta n_{12})$ & Uniform\{0, 0.5, 1\} \\
      \hline
       \end{tabular}
      \caption{Parameters and corresponding priors used to simulate individual GW signals and the pairs of lensed GW signals.}
       \label{bbh_params}
  \end{table}
  
For a parameter estimation method to be trustworthy, it should recover the injected parameters within a given confidence interval for a corresponding fraction of injections; the extent to which it does so can be assessed using a so-called P-P plot~\cite{Veitch:2014wba, Romero-Shaw:2020owr}. 
To this end, we simulate a population of 70 BBH signals by sampling the appropriate source parameters from the prior distributions specified in Table~\ref{bbh_params}. 
We adjust the luminosity distance in such a way that each event has a network SNR above 13. 
The network consists of the two LIGO and the Virgo interferometers at their design sensitivity~\cite{KAGRA:2013rdx}. 
We perform parameter estimation on each event and show the collective results in Figure~\ref{fig:pp-plot}, representing a P-P plot for our relative binning framework. Ideally, each coloured 
line should trace the diagonal; however, some fluctuations are expected due to a finite number of signals being analysed and the presence of noise. 
We perform a KS test as a measure of the consistency between each coloured line and the diagonal line. We quote the p-value of this test for each parameter in the bracket. Additionally, we also test if the distribution of all p-values quoted in the brackets follows a uniform distribution. The combined p-value for this test is 0.6625, which is expected.
To further quantify this consistency, we can define a binomial random variable $X$ as the number of times the injected value is recovered within 
a confidence interval corresponding to a value on the horizontal axis. 
For this binomial distribution, the shape parameter $N$ equals 70 and $p$ equals the confidence corresponding to the x-axis.  
The shaded regions in Figure~\ref{fig:pp-plot} cover 1$\sigma$, 2$\sigma$, and 3$\sigma$ confidence intervals of the binomial probability distribution in decreasing order of opacity. 
As all the plotted curves for the different parameters fall within the 3$\sigma$ boundary and the accompanying p-values indicate good consistency with the diagonal, 
we conclude that our method is robust.
\begin{figure}[h]
    \centering
    \includegraphics[width=.75\textwidth]{../figures/pp_3.0_partial_no_railing.pdf}
    \caption{Percentile-percentile showcasing the robustness of the relative binning method on simulated BBH signals. Each line traces the diagonal, indicating that the parameter corresponding to it is recovered with expected accuracy. The numbers in the brackets of the legend show the p-values of the KS tests. The combined p-value of all parameters is 0.6625, consistent with the hypothesis that individual p-values were derived from a uniform distribution as expected. The shaded regions show 1$\sigma$, 2$\sigma$, and 3$\sigma$ confidence intervals in decreasing order of opacity.}
    \label{fig:pp-plot}
\end{figure}

\subsection{Validation on real data: analysing GW signals detected by LIGO-Virgo interferometers}

\begin{figure}[h]
    \centering
    \includegraphics[width=.75\textwidth]{../figures/overlap_production_gw190412_gwtc.pdf}
    %\includegraphics[width=.75\textwidth]{../figures/overlap_production_gw190814_corr_branch_gwtc.pdf}
    \caption{The parameter measurements obtained using the relative binning framework (blue curves) are consistent with the ones reproduced from the public data release of the LVK collaboration (grey) for GW190412.}
    \label{fig:gw190412}
\end{figure}

Next, we turn to parameter estimation on the events detected by LIGO-Virgo. 
We choose GW190412, GW190814, GW191103, and GW191105. 
The first two events are chosen as they consist of BBHs with highly asymmetric masses. This feature allows for a significant contribution from the higher 
harmonics~\cite{LIGOScientific:2020stg}, enabling us to stress-test our framework.  
The remaining two events were chosen because they were part of a candidate event pair from the strong lensing searches~\cite{LIGOScientific:2023bwz, Janquart:2023mvf}. 

\begin{figure}[h]
    \centering
    \includegraphics[width=.75\textwidth]{../figures/overlap_production_gw190814_corr_branch_gwtc.pdf}
    \caption{The parameter measurements obtained using the relative binning framework (blue curves) are consistent with the ones reproduced from the public data release of the LVK collaboration (grey curves) for GW190814.}
    \label{fig:gw190814}
\end{figure}

For GW190412 and GW190814, a truncated corner plot showing posterior distributions for selected parameters is presented in Figures~\ref{fig:gw190412} and Figure~\ref{fig:gw190814}, respectively.
The blue (grey) curves represent the results of parameter estimation with (without) relative binning. 
The grey curves are generated by using the posterior samples provided in the public data release of the LVK collaboration~\cite{LIGOScientific:GWTC3DR, LIGOScientific:GWTC2.1DR}. 
Figures~\ref{fig:gw190412} and \ref{fig:gw190814} show a qualitative agreement between the posterior distributions obtained by the exact method (without making any approximation) and the relative binning method. Similarly, we analysed GW191103 and GW191105 with the relative binning method, which led to a comparable agreement with the exact method. 

\begin{figure}[h]
    \centering
    \includegraphics[width=\textwidth]{../figures/real_threeg_lensed_jsd.pdf}
    \caption{JS divergence values to quantify if the measurements made using the relative binning method are in agreement with the ones made using exact computation. The black line marks the JS divergence below which the two measurements are indistinguishable. The plot contains the JS divergence values for all parameters for all signals, simulated as well as real, analysed in this chapter. For compactness, we use the same marker for the set of 3G signals (circles) and for the set of lensed signals (crosses). We use different coloured markers for LIGO-Virgo events. All JS divergence values remain below the threshold, indicating that measurements obtained using the relative binning method are reliable.}
    \label{fig:master-jsd}
\end{figure}

To quantify the mismatch between the posterior samples obtained using the relative binning method and the ones obtained with the exact method, we use the JS divergence. 
We do not perform an exact parameter estimation run due to the computational cost. Instead, we reweigh the posterior samples obtained using the relative binning 
method to obtain the posterior samples for the exact method as discussed in the next paragraph.
We repeat this step for all four events, GW190412, GW190814, GW191103, and GW191105, and plot the JS divergence between the relative 
binning samples and the exact samples for each parameter in Figure~\ref{fig:master-jsd}. Each event is represented by a different coloured marker. 
To determine a heuristic threshold value for statistical indistinguishability (represented by the dashed line in Figure~\ref{fig:master-jsd}), 
we compute the JS statistic for 2000 pairs of sets of $10^5$ samples taken from Gaussian distributions. 
We then use the 99th percentile of the obtained values as the heuristic threshold for indistinguishability. The JS divergence for each parameter remains well below the threshold, 
indicating that the posterior distributions produced by the relative binning method and the exact method are statistically indistinguishable. As an additional validation test, we use the posterior samples obtained through the relative binning method on the GW190412, GW190814, GW191103, and GW191105 events to compute the exact likelihoods (i.e. Eq.~\eqref{eqn:short-l}). The standard deviation of the absolute difference between the exact and the relative binning likelihoods turns out to be $\leq 0.1$, which indicates that the error incurred by the relative binning likelihood is sufficiently small to not impact the posterior distribution~\cite{Leslie:2021ssu}. 


\subsection{On the parameter estimation of 3G-era events}

To further test our method, we analyse a subset of LIGO-Virgo events as they would appear in 3G interferometers. We also analyse an example of overlapping signals.  

Specifically, for the isolated signals, we use the maximum likelihood parameters obtained from the posterior samples of the LIGO-Virgo events. We then simulate a GW signal using the maximum likelihood parameters and add it to the simulated Gaussian noise of the 3G interferometers. The high sensitivities of the 3G interferometers lead to significantly louder SNRs. We then analyse the simulated data with the relative binning framework. We arbitrarily choose six events with varying total masses, specified in Table~\ref{tab:gwtc-3-table}. 

\begin{table}[h]
    \centering
    \begin{tabular}{|c|c|c|}
        \hline
        \textbf{Event} & \textbf{Total Mass ($M_{\odot}$)} & \textbf{Network SNR} \\
        \hline
 GW191126\_115259 & 26.51 & 192.52 \\
        \hline
GW200306\_093714 & 62.10 & 326.02 \\
        \hline
GW200112\_155838 & 78.10 & 1103.53 \\
        \hline
 GW191127\_050227 & 138.71 & 260.33 \\
        \hline
 GW191230\_180458 & 144.69 & 383.10 \\
        \hline
 GW200220\_061928 & 282.50 & 224.91 \\
        \hline

    \end{tabular}
    \caption{Table of events analysed in the context of 3G detectors. The second column shows the maximum likelihood values of the total mass obtained from the public data release of LVK. The third column shows the respective network SNRs if they were observed by a network of 3G detectors. The events were detected by LIGO-Virgo interferometers during their third observation runs, and the naming convention follows the pattern GWYYMMDD\_HHMMSS. }
    \label{tab:gwtc-3-table}
\end{table}

\begin{figure}[t]
    \centering
    \includegraphics[width=.75\textwidth]{../figures/prod_third_gen.pdf}
    \caption{A representative corner plot showing the results of parameter estimation on GW200220\_061928 using relative binning. The event was observed by LIGO-Virgo interferometers in the third observation run. Here, we analyse a simulated equivalent of it as if it were observed by a network of 3G detectors. The black lines show the true values of the parameters. The blue contours correspond to $1\sigma$, $2\sigma$, and $3\sigma$ confidence, in order of decreasing opacity, indicating that all parameters are recovered with expected accuracy.}
    \label{fig:third-gen}
\end{figure}

Our detector network consists of a triangular ET with an arm length of 10 km~\cite{Branchesi:2023mws} and one L-shaped CE with an arm length of 40 km~\cite{Reitze:2019iox}.
The noise curves and detector configurations are obtained from~\cite{threeGdesign}. The starting frequency for our analysis is 5 Hz. 
A representative, truncated corner plot for this analysis is shown in Figure~\ref{fig:third-gen} for the GW200220\_061928 event. 
The black lines in the plot show the injected parameter values. 
The $1\sigma$, $2\sigma$, and $3\sigma$ confidence contours are indicated by the blue shading, in order of decreasing opacity. 
As can be seen from the plot, all injected parameters have been recovered well. 

To quantify the accuracy of our method for the 3G scenario, we again reweigh the posterior samples obtained using the relative binning method following the reweighting procedure discussed in Section~\ref{reweight}. Then, we calculate the JS divergence between the reweighted samples and the relative binning samples. 
This process is repeated for each of the six events and the results are shown in Figure~\ref{fig:master-jsd}. For compactness, we use the circle makers for all events and all parameters since our aim here is to test if JS divergences for any of them corsses the threshold. Again, the JS divergence for each parameter remains well below the threshold, indicating that the posterior distributions produced by the relative binning method and the exact method are statistically indistinguishable.

\subsubsection{Overlapping signals example}

\begin{figure}
    \centering
    \includegraphics[width=.68\textwidth]{../figures/overlapping_signals_A.pdf}\vspace{.25cm}
    \includegraphics[width=.68\textwidth]{../figures/overlapping_signals_B.pdf}
    \caption{Comparison of parameter estimation results when two signals are in isolation and when two signals are overlapping. Measurements in isolation establish the benchmark for accuracy. The agreement between the overlapping signal analysis and isolated signals analysis indicates that the relative binning framework to perform joint parameter estimation can be extended to a wider parameter space.}
    \label{fig:overlapping-signal}
\end{figure}
We collectively present the results for overlapping signals analysis in Figure~\ref{fig:overlapping-signal}. Let us refer to the two signals by signal A and signal B. The red (green) lines in the top (bottom) plot show how accurately we can recover signal A (signal B) when they are in isolation. This establishes the benchmark for the accuracy. The blue line in both plots shows the results of joint parameter estimation using relative binning framework, i.e., how well the parameters of the two signals are recovered when they are overlapping. The agreement of the blue curves with the green and red curves suggests that our relative binning framework can be reliably extended to perform joint parameter estimation when two (or more) GW signals are overlapping. 

\subsection{On strongly lensed events}

\begin{figure}
    \centering
    \includegraphics[width=0.53\textwidth]{../figures/thesis_pair_191103_191105.pdf} \hspace{.5cm}
    \includegraphics[width=0.41\textwidth]{../figures/lens_pair_191103_191105.pdf}
    \caption{Analysing the pair, GW191103 (red) and GW191105 (green), under strong lensing hypothesis (blue) using the relative binning method. The left plot shows the measurements of the common parameter between the two events, when analysed individually and when analysed under strong lensing hypothesis. The right plot shows the measurements of the lensing parameters (only available when analysed under strong lensing hypothesis). The event pair was of great interest for strong lensing searches due to significant overlaps in the posterior distributions of individual events, though the analyses turned out to be inconclusive.}
    \label{fig:lensed_pair}
\end{figure}

We analyse eight simulated lensed pairs and, additionally, the GW191103--GW191105 event pair to test the relative binning framework for joint parameter estimation. 
The latter event pair is of interest as it was followed up by strong lensing searches because it showed some prototypical features of galaxy lensing, such as a mild posterior overlap and a short time delay between two images~\cite{LIGOScientific:2023bwz, Janquart:2023mvf}. 
On the other hand, the simulated pairs were generated by sampling the relevant source parameters from the priors shown in Table~\ref{bbh_params}. 
We adjust the luminosity distance and the relative magnification of each lensed pair to ensure that the network SNR of each image is above 13.

Again, we compute the JS divergence between the posterior distributions obtained using the exact and relative binning methods;  
see Figure~\ref{fig:master-jsd}. Note that, for the lensed injections, we do not use reweighting as was done for the individual parameter estimation cases. 
Instead, we carry out joint parameter estimation using the exact likelihood~\cite{Janquart:2023osz}. All crosses are below the threshold value, suggesting that the posterior distributions obtained using the exact and relative binning frameworks are statistically indistinguishable. For compactness, we use the cross marker for all 8 pairs and all parameters, since our aim here is to check if the JS divergences remain below the threshold. 

Now let us look at joint parameter estimation results for the GW191103--GW191105 pair. 
In this analysis, GW191103 is treated as image 1, and GW191105 as image 2. 
The left plot of Figure~\ref{fig:lensed_pair} shows a truncated corner plot of the common parameters between the two images, while the right one presents the corner plot of the relative lensing parameters, connecting images 1 and 2.
To quantify the accuracy of our results, we follow the same reweighting steps mentioned towards the end of Section~\ref{reweight} to calculate the JS divergence between the relative binning samples and the exact samples for each parameter. 
The JS divergences are marked with a hexagon marker in Figure~\ref{fig:master-jsd}, which remain all below the threshold. 

Finally, we report that the log of coherence ratio -- a statistic to quantify the overlap between the posterior distribution of two images -- obtained for GW191103--GW191105 is 4.5~\cite{Janquart:2023osz, Lo:2021nae, Janquart:2021qov}. This is in favour of the lensing hypothesis and comparable to the one reported by the lensing follow-up searches~\cite{Janquart:2023mvf}. 

\section{Comparison of speed-up and accuracy}

\begin{figure}
    \centering
    \includegraphics[width=.5\textwidth]{../figures/thesis_speed_up_summary.pdf}
    \caption{The speed-up achieved by the relative binning framework in relation to the total mass. The colourbar shows the reduction in the frequency points on which one needs to generate the waveform. The plot contains the comparison of run time for all events analysed using the relative binning framework in this chapter. The cross markers represent the P-P plot injections, the plus markers represent the 3G-injections, and the hexagon markers represent the lensing analyses. The remaining markers are for LIGO-Virgo events. The median value of the factor of speed-up is $\sim 10$ for total masses below $50 M_{\odot}$. The low total mass signals typically have more cycles in-band, so they are more expensive to analyse with standard methods. Hence, the speed-up obtained by the relative binning method is precisely in the region where it is highly desirable.}
    \label{fig:speed}
\end{figure}

The speed-up achieved by the relative binning method can be broadly attributed to two factors. 
First, every proposal waveform is generated on fewer frequency nodes -- provided by the RBGrid -- compared to the exact method, reducing the time required to generate the waveforms during sampling. 
Second, as the summary data are pre-computed, the summation in Eqs.~\eqref{eqn:dh}-~\eqref{eqn:hh} needs to be computed at fewer frequency points, reducing the number of operations.
The first point is the more significant contributor to the overall speed-up.
With this, the total speed-up for relative binning mainly depends on how coarse the RBGrid is compared to the corresponding uniform grid. The reduction in frequency nodes depends on the signal characteristics such as the duration, SNR, fiducial parameters, and also the total error chosen to generate the RBGrid.


Here, we use the comprehensive parameter estimation analyses performed in previous sections to assess the speed-up obtained by the relative binning compared to the exact method.
Figure~\ref{fig:speed} shows the speed-up achieved by the relative binning method over the exact method in relatio to the total mass of the systems. The colourbar shows the ratio of the length of the uniform frequency grid to the length of the corresponding RBGrid. We denote the ratio by $\Delta$. For the instances where we performed parameter estimation runs with exact likelihood and relative binning approximation, we report the ratio of the run times of the two. For the instances where we only performed relative binning runs, we show the ratio of the average single likelihood evaluation time taken between the two methods calculated over a set of $10^4$ points. The latter is an approximation, but it allows us to 
assess the speed-up factor given by relative binning without having to do the exact parameter estimation runs. Still, for our various tests, 
the number of likelihood evaluations done for the relative binning and exact methods is in the same ballpark, so that this is a reasonable approximation to make.

We observe that the relative binning runs are up to a factor of $\sim 34$ faster for the LIGO-Virgo interferometers and up to a factor of $\sim 47$ faster for the 3G interferometers. 
The speed-up increases as the total mass of the system decreases. This is a highly desirable feature of the relative binning method since the low mass systems are typically more computationally expensive. The cost can be attributed to the large number of inspiral cycles in-band. Relative binning helps place a coarse frequency grid for such signals, which speeds up the analysis. Thus, the speed up achieved by relative binning is precisely in the region of the parameter space where it is needed, i.e., the low total mass region. 

Additionally, the 3G detectors have a lower starting frequency, meaning they can observe a longer inspiral of the GW compared to the LIGO-Virgo interferometers. Consequently, we find that the relative binning framework provides a larger speed-up in the 3G scenario compared to similar systems as seen in LIGO-Virgo. This trend is also visible in Figure~\ref{fig:speed}. %However, the 3G injections considered for the analysis have a considerably higher SNR than the corresponding P-P injections for the LIGO-Virgo interferometers. 

%A higher SNR will increase the size of the coarse grid and decrease the speed of the relative binning method. 
%In addition, the median of the SNR for the set of P-P injections is $59.44$, which is $\simeq 2.2$ times larger compared to the loudest GW event observed to date \cite{LIGOScientific:GWTC3DR}. Since a large value of SNR increases the size of the coarse grid, the factor of speed-up reported by the P-P injections is on the conservative side.  

Joint parameter estimation reports a similar speed-up as individual parameter estimation. Here, we note that the lensing joint parameter estimation framework used in this work generates the first waveform and the subsequent images are obtained by scaling the first one. The scaling helps bypass the additional waveform generation steps, which saves computational cost. However, this is not the case in all strong lensing search methods, i.e., when the method is designed to generate all waveforms independently. In comparison to such methods, we expect a larger speed-up factor for relative binning framework.

\section{Conclusion and outlook}
In this chapter, we have presented a relative binning framework to perform fast parameter estimation. We have shown the robustness of the method by performing parameter estimation on a large set of simulated as well as real events. We have also extended the method to perform parameter estimation on overlapping signals and to perform strong lensing searches. This marks the first instance where the relative binning method has been tested in such a comprehensive manner. Our implementation of the relative binning framework can be found in the Github repository at~\cite{janquart:2022}.

To streamline the relative binning framework, we still need to address a couple of challenges. The performance of our method depends on the choice of the fiducial waveform. In this work, we have assumed that the fiducial waveform is either the same as the injected waveform for simulations or the maximum likelihood waveform for real events. However, the method can be seeded more realistically by using the best-fitting template reported by the template-based searches or maximum likelihood estimator routines can be used to determine the fiducial waveforms' parameters. We plan to work on the choice of the fiducial waveform and understanding how it can impact the performance of our method in the future. We also leave it to future work to implement the detector calibration parameters in the relative binning framework.  

While the relative binning method is shown to work well in the joint parameter estimation of the overlapping signals, it is crucial to devise a reciepe to determine when such a joint parameter estimation is required, i.e., if the signals are sufficiently apart, they can be analysed separately, which may help reduce the cost of the analysis. This question was addressed in the comprehensive analysis performed by Baka \textit{et al}~\cite{Baka:2025yqx}. The codebase presented in this chapter served as a building block for it. 

The detection rates of gravitational wave signals and consequently the amount of computational resources needed to perform parameter estimation will increase with the upgrades to the current detectors. There is also a growing concern about the challenges posed by the increasing cost of parameter inference and data analysis tools in general in the 3G era. The relative binning framework presented in this chapter marks a step towards addressing these challenges.


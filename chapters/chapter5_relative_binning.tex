Bayesian inference discussed in chapter~\ref{ch:gwpe} is the key to achieving a wide range of scientific insights from gravitational wave signal. The cost of it increases with the complexity of the signal model and the signal's duration. With standard methods, such analyses require at least a few days. Moreover, for third generation detectors, due to the lowered minimum sensitive frequency, the signal duration increases, leading to even longer analysis times. With the increased detection rate, such analyses can then become intractable. In this work, we present a fast parameter estimation method that includes higher order modes and precession using relative binning. Additionally, we extend the method so that it can be used to analyze individual GW as well as two overlapping GW signals in the 3G-era. We adapt our method to perform strong lensing searches. Indeed, a fast parameter estimation has a wide range of application. Specifically, we apply our method to
\begin{enumerate}
    \item A set of simulated BBH signals covering full parameter space to establish validity. 
    \item Two LIGO-Virgo events with significant contribution from higher order modes: GW190412~\cite{LIGOScientific:2020stg} and GW190814~\cite{LIGOScientific:2020zkf}.
    \item A set of simulated GW signals for the 3G-detector; isolated as well as overlapping.
    \item The event pair GW191103-GW191105 as well as a set of simulated pairs, to investigate strong lensing hypothesis
\end{enumerate}

\section{Mode-by-mode relative binning}
From chapter~\ref{ch:gwpe}, let us recall the (log) likelihood function
\begin{equation}
\label{eqn:indi-likelihood}
\ln \mathcal{L}(\vec{\theta}) = -\frac{1}{2}\left<\vec{d}-\vec{h}(\vec{\theta})|\vec{d}-\vec{h}(\vec{\theta})\right>\,,
\end{equation}
where $\vec{h}(\vec{\theta})$ represents the GW waveform as a function of the source parameters $\vec{\theta}$ and $\left<\,.\,|\,.\,\right>$ 
denotes the noise-weighted inner product. Expanding the inner product in Eq.~\eqref{eqn:indi-likelihood}, we get
\iffalse
\begin{equation}
(\vec{a}|\vec{b}) = \frac{4}{T}\Re\sum_f \frac{a(f)b^*(f)}{S_n(f)}.
\end{equation}
Here, $f$, $S_n(f)$, and $T$ respectively denote the frequency, one-sided noise power spectral density 
(PSD), and duration, while $\Re$ denotes the real part; the sum over frequencies will be discussed in detail below. 
Expanding $\left<\vec{d}-\vec{h}(\vec{\theta})|\vec{d}-\vec{h}(\vec{\theta}))$ in Eq.~\eqref{eqn:indi-likelihood}, we get
\fi
\begin{equation}
\label{eqn:logL}
\ln \mathcal{L}(\vec{\theta}) = \left<\vec{d}|\vec{h}\right> - \frac{1}{2}\left<\vec{h}|\vec{h}\right> - \frac{1}{2}\left<\vec{d}|\vec{d}\right>.
\end{equation}

Likelihood-based inference involves calculating the innder products for a large number of values for $\vec{\theta}$ to explore the parameter space. Therefore, we want to improve the speed in computing these quantities. Let us introduce the notations
\begin{align}
    \ln \mathcal{L}_{dh} &\equiv \left<\vec{d}|\vec{h}\right>, \\
    \ln \mathcal{L}_{hh} &\equiv \left<\vec{h}|\vec{h}\right>, 
\end{align}

to rewrite Eq.~\eqref{eqn:logL} as 
\begin{equation}
\label{eqn:short-l}
\ln \mathcal{L}(\vec{\theta}) = \ln \mathcal{L}_{dh} - \frac{1}{2} \ln \mathcal{L}_{hh}\,.
\end{equation}
Note that we dropped the last term of Eq.~\eqref{eqn:logL} in the equation above. It is also known as the noise log likelihood. Since it does not depend on the source parameters, it need only be computed once for the data.

\subsection{Dominant mode and aligned spin}
When performing parameter estimation, we are interested in knowing the shape of likelihood around its peak. In other words, we want to evaluate the likelihoods 
for waveforms that are {\it similar} to the waveform at which the likelihood peaks. 
If we choose a fiducial value of $\vec{\theta}$ which is close enough to the peak, 
we can estimate the likelihood by expanding it around the fiducial value. 
Let us denote the fiducial source parameters by $\vec{\theta}'$ and the waveform at the 
fiducial parameters by $\vec{h}'$. For any proposal waveform $\vec{h}$, 
under the assumption that $\vec{h}$ and $\vec{h}'$ are close enough, we can find 
frequency bins $b =  [f_{min}, f_{max}]$ such that the ratio 
$\vec{h}/\vec{h}'$ can be linearly approximated as

\begin{equation}
\label{eqn:ratio}
\frac{h}{h'}(f) = r_1(b) + r_2(b)(f-f_c(b))+\mathcal{O}(f^2), 
\end{equation}

within a given bin, where $r_1(b)$ and $r_2(b)$ are expansion coefficients 
and $f_c$ denotes the central value of the frequency bin $b$. 

The approximation made in Eq.~(\ref{eqn:ratio}) performs better for the IMRPhenomD waveform compared to the IMRPhenomXPHM one because the ratio $\vec{h}/\vec{h}'$ varies more smoothly with $f$ for the former (see left plot of Figure~\ref{fig:dxp-ratio}). Therefore, when using the IMRPhenomD waveform a few bins are sufficient to make the linear expansion from Eq.(\ref{eqn:ratio}) precise in each bin. This is not the case anymore for more complex waveform models. When using the IMPRhenomXPHM waveform, as the ratio is subject to more fluctuation, we need to consider a larger number of bins, which increases waveform generation costs and makes the likelihood evaluation slow. 

\begin{figure}
    \centering
    \includegraphics[width=.4\textwidth]{../figures/phenom_d_xphm_full_waveform_ratio.pdf}\hspace{.5cm}
    \includegraphics[width=.4\textwidth]{../figures/hL_C_ratio.pdf}
    \caption{\textit{Left}: Ratio of the real part of the proposal to the fiducial waveform for the plus polarization as a function of frequency. The IMRPhenomD (brown) waveform model includes only the dominant mode and aligned spins. The IMRPhenomXPHM (black) model includes higher order modes and precessing spins. \textit{Right}: Decomposition of IMRPhenomXPHM waveform into component modes (top right) and precession coefficients (bottom right). The fiducial parameters are similar to the source parameters of GW190412~\cite{LIGOScientific:2020stg}, a BBH merger with asymmetric masses and detectable HOM content, and the proposal parameters have a 5$\%$ shift from the fiducial chirp mass and mass ratio.}
    \label{fig:dxp-ratio}
\end{figure}

\subsection{Inclusion of higher order modes and precessing spins}
In order to make the approximation in Eq.~(\ref{eqn:ratio}) robust for precessing waveforms with HOMs, 
let us express the detector frame waveform for given $\vec{\theta}$ into its 
component modes~\cite{Pratten:2020ceb} 
\begin{equation}
h(f) = \sum_{l, m}(F_+C^{+}_{l, m}(f)+F_\times C^{\times}_{l, m}(f))\,h^L_{l, m}(f)\,.
\label{eqn:comp-mode}
\end{equation} 

Here, $F_+$ and $F_\times$ are the antenna pattern functions of the interferometer, which depend 
only on extrinsic parameters; $\vec{h}^{L}$ denotes the waveform in the co-precessing frame 
(also known as the $L$-frame); and the indices $l$ and $m$ label the different modes. The coefficients $C^{+}_{l, m}$ and $C^{\times}_{l, m}$ account for 
the {\it twisting-up} procedure~\cite{Pratten:2020ceb} which transforms the waveform from a 
co-precessing frame to the inertial (or observer) frame. 
The benefit of handling each mode in the $L$-frame separately is that the ratio of a 
fiducial $\vec{h}'^{L}_{l, m}$ to a proposal $\vec{h}^{L}_{l, m}$ oscillates less compared to the 
ratio of the full waveforms $\vec{h}/\vec{h}'$. This can be seen by comparing the bottom right panel of 
Figure~\ref{fig:dxp-ratio} with the top right panel of Figure~\ref{hlc-plot}. 
Therefore, a linear expansion similar to the one in Eq.~\eqref{eqn:ratio} 
can be made using the ratio $\vec{h}^{L}_{l, m}/\vec{h}'^L_{l, m}$ instead of 
$\vec{h}/\vec{h}'$ and will require a lower number of frequency bins for more complex models:
\begin{equation}
\label{eqn:l-ratio}
\frac{h^{L}_{l, m}}{h'^L_{l, m}} (f) = r_{1, l, m}(b) + r_{2, l, m} (b)(f-f_c(b)) + \mathcal{O}(f^2) \,,
\end{equation}

Now, let us denote the factor $(F_+C^{+}_{l, m}(f)+F_\times C^{\times}_{l, m}(f))$ in 
Eq.~(\ref{eqn:comp-mode}) by $\vec{C}_{l, m}$. We can perform similar expansions on the ratios 
$\vec{C}_{l, m}/\vec{C}'_{l, m}$:
\begin{equation}
\label{eqn:c-ratio}
\frac{C_{l, m}}{C'_{l, m}}(f) = s_{1, l, m}(b) + s_{2, l, m}(b)(f-f_c(b)) + \mathcal{O}(f^2).
\end{equation}
Below, the expansion coefficients will collectively be denoted 
$\mathcal{R} = \{\vec{r}_1, \vec{r}_2, \vec{s}_1, \vec{s}_2\}$. 
For now, we assume we can find frequency bins such that the piece-wise linear interpolations 
in Eqs.~(\ref{eqn:l-ratio}) and (\ref{eqn:c-ratio}) are 
valid. How such bins can be found will be explained in Sec.~\ref{sec:bin-selection}.

Once the frequency bins are made based on the chosen fiducial waveforms, we can pre-compute some 
\textit{summary data}~\cite{Leslie:2021ssu} based only on the reference waveform and useful for the 
computation of the inner products in Eq.~(\ref{eqn:short-l}). These are:
\begin{eqnarray}
&& W_{l, m}(b) = \frac{4}{T} \sum_{f\in b}\frac{d(f)h'^{L*}_{l, m}(f)C'^{*}_{l, m}(f)}{S_n(f)} \, , 
\nonumber \\
&& X_{l, m}(b) = \frac{4}{T} \sum_{f\in b}\frac{d(f)h'^{L*}_{l, m}(f)C'^*_{l, m}(f)}{S_n(f)}(f-f_c(b)) \,, 
\nonumber\\
&& Y_{l, m, \bar{l}, \bar{m}}(b) = \frac{4}{T} \sum_{f\in b}\frac{h'^L_{l, m}(f)C'_{l, m}(f)h'^{L*}_{\bar{l}, \bar{m}}(f)C'^*_{\bar{l}, \bar{m}}(f)}{S_n(f)} \,, 
\nonumber\\
&&Z_{l, m, \bar{l}, \bar{m}}(b)  \nonumber\\
&&= \frac{4}{T} \sum_{f\in b}\frac{h'^L_{l, m}(f)C'_{l, m}(f)h'^{L*}_{\bar{l}, \bar{m}}(f)C'^*_{\bar{l}, \bar{m}}(f)}{S_n(f)}(f-f_c(b))\,.
\nonumber\\
\end{eqnarray}

Note that the summary data 
$\mathcal{D} = \{W_{l, m}, X_{l, m}, Y_{l, m, \bar{l}, \bar{m}}, Z_{l, m, \bar{l}, \bar{m}} \}$ 
need to be computed only once per analysis and they depend only on the choice of the fiducial 
parameters $\vec{\theta}'$. On the other hand, the coefficients $\mathcal{R}$ need to be computed for every 
proposal waveform as they depend on $\vec{\theta}$ and $\vec{\theta}'$.  

Using the summary data $\mathcal{D}$ and the coefficients $\mathcal{R}$, we can estimate $\ln\mathcal{L}_{dh}$ and $\ln\mathcal{L}_{hh}$ in Eq.~\eqref{eqn:short-l} 
for given $\vec{\theta}$ as


\begin{align}
\label{eqn:dh}
\ln\mathcal{L}_{dh} &= \Re \sum_{(l, m)} \sum_{b}  \left\{  W_{l, m}(b)\left[r_{1, l, m}^*(b) s_{1, l, m}^*(b)\right] +X_{l, m}(b)\left[r_{1, l, m}^*(b)s_{2, l, m}^*(b) + r_{2, l, m}^*(b) s_{1, l, m}^*(b) \right]  \right\} \,, \\
\label{eqn:hh}
\ln\mathcal{L}_{hh} &=\Re  \sum_{(l, m), (\bar{l}, \bar{m})} \sum_{b} \left\{  Y_{l, m, \bar{l}, \bar{m}}(b) \left[ r_{1, l, m}(b)s_{1, l, m}(b) r^*_{1, \bar{l}, \bar{m}}(b) s^*_{1, \bar{l}, \bar{m}}(b) \right] \right. \nonumber \\
&\qquad\qquad\qquad\qquad +  \left. Z_{l, m, \bar{l}, \bar{m}}(b) \left[r_{1, l, m}(b) r_{1, \bar{l}, \bar{m}}^*(b) \left( s_{1, \bar{l}, \bar{m}}^*(b) s_{2, l, m}(b) + s_{1, l, m}(b) s_{2, \bar{l}, \bar{m}}^*(b) \right) \right. \right. \nonumber \\ 
&\qquad\qquad\qquad\qquad + \left. \left.  s_{1, l, m}(b)s_{1, \bar{l}, \bar{m}}^*(b)\left(r_{1, \bar{l}, \bar{m}}^*(b) r_{2, l, m}(b) + r_{1, l, m}(b) r_{2, \bar{l}, \bar{m}}^*(b)\right) \right] \right\},
\end{align}

which together with Eq.~(\ref{eqn:short-l}) enables us to approximate the likelihood. Provided the number of bins is reduced compared to the usual method, this approach should offer a rapid likelihood evaluation, leading to a rapid parameter inference.

\section{P\lowercase{arameter Estimation for Strong Lensing}}
\label{sec:rb2}
When analyzing a strongly-lensed gravitational wave signal that has been split into several images, one can perform joint parameter estimation, i.e.~one can analyze the images at the same time 
taking into account the common parameters in addition to the ones introduced by lensing~\cite{Janquart:2023osz, Liu:2020par, Lo:2021nae, Janquart:2021qov}. 
For two observable images, the joint analysis requires 19 parameters (see Table~\ref{bbh_params}), hence 4 more than single unlensed signal parameter 
estimation. If more than two images are present, each additional image will introduce three more parameters, thereby further increasing the computational time.
In this section, we recall the main aspects of strong lensing and describe the extension of the relative binning method to perform joint parameter estimation. For 
simplicity we focus on the case of two images, but the extension to $N > 2$ images is straightforward.

\subsection{Effect of strong lensing on gravitational-wave signals}

When a GW signal is strongly lensed, it is split into multiple images, each undergoing a (de-)magnification, time delay and overall phase shift~\cite{Dai2020SearchO2}. 
The $j^{\mathrm{th}}$ lensed image ($\bm{h}^l_j$) is related to the unlensed waveform ($\bm{h}^u$) through 
\begin{equation}
h^l_j (f; \boldsymbol{\Theta}, \mu_j, t_j, n_j) = \sqrt{\mu_j} h^{u}(f;\boldsymbol{\Theta})e^{2i\pi f t_j -in_j\pi} \,,
\end{equation}
where $\mu_j$, $t_j$ and $n_j$ denote the magnification factor, arrival time, and Morse factor for the $j^{\text{th}}$ image, respectively. We refer to them as the \textit{lensing parameters} of the $j^{\text{th}}$ image.
 
For a given image, these lensing parameters can not all be separately measured. Indeed, the magnification and the time delay are degenerate with the luminosity distance and the time of coalescence, 
respectively~\cite{Dai:2017huk, Lo:2021nae, Janquart:2021qov}. Therefore, we can absorb $\sqrt{\mu_j}$ into the luminosity distance $D_L$, and $t_j$ into the coalescence time $t_c$ 
to define the following effective quantities for the $j^{th}$ image:
\begin{align}
D_{L, j}^{\text{eff}} &= \frac{D_L}{\sqrt{\mu_j}} \,, \\
t_{c, j}^{\text{eff}} &= t_c+t_j \,.
\end{align}
These quantities can be measured from the GW data. 
On the other hand, the Morse phase is not necessarily degenerate with the phase of coalescence, as the presence of HOMs can break the possible phase-Morse factor 
degeneracy~\cite{Dai:2017huk, Ezquiaga2020PhaseWaves, Janquart:2021nus}

As mentioned above, for simplicity we focus on the scenario where two strongly-lensed images are analyzed. We refer to the image that arrives first at the detector as image 1, 
and denote its observed parameters by $\boldsymbol{\Theta}_1$\footnote{Note that in the context of strong lensing $\boldsymbol{\Theta}_1$ includes the Morse factor, and the effective luminosity and time delay for the first image.}. Since some of the lensing parameters are not measurable on their own, let us define \emph{relative lensing parameters} linking the lensing parameters of the two images:
\begin{align}
\label{eqn:mu21}
\mu_{21} = \mu_2/\mu_1  &= \left(D_{L, 1}^{\text{eff}} / D_{L, 2}^{\text{eff}}\right)^2  \\\
\Delta t_{21} &= t^{\text{eff}}_{c, 2}-t^{\text{eff}}_{c, 1} \\
\label{eqn:n21}
\Delta n_{21} &= n_2 - n_1 
\end{align}
We collectively denote the relative lensing parameters by $\boldsymbol{\Phi}_{21} = \{\mu_{21}, \Delta t_{21}, \Delta n_{21}\}$. 
Table \ref{tab:notation-table} summarizes all the parameters defined in the context of strong lensing. 
The relative lensing parameters can be used to express the GW from the second image in terms of the waveform for the first image:
\begin{equation}
\label{eqn:wf-mapping}
h^l_2(f; \boldsymbol{\Theta}_2) = \sqrt{\mu_{21}}h^{l}_1(f; \boldsymbol{\Theta}_1)e^{2i\pi f \Delta t_{21} - i\Delta n_{21}\pi}.
\end{equation}

\begin{table}
\begin{center}
\begin{tabular}{| c | p{1 in} | p{1.7 in} |}
\hline 
 Notation & Description & Parameters  \\ \hline 
 $\boldsymbol{\Lambda}$ & Source parameters common between the two images & \{$\mathcal{M}_c, \eta, a_1, a_2, \theta_1, \theta_2, \Delta \phi, \phi_{JL}$, $\text{RA}, \text{DEC}, \theta_{JN}, \psi, \phi \}$ \\ \hline
$\boldsymbol{\Theta}_1$ & Observed parameters of image 1 & $ \boldsymbol{\Lambda} \cup \{D^{\text{eff}}_{L, 1}, t^{\text{eff}}_{c, 1}, n_1\}  $ \\ \hline 
$\boldsymbol{\Theta}_2$ & Observed parameters of image 2 & $\boldsymbol{\Lambda} \cup \{D^{\text{eff}}_{L, 2}, t^{\text{eff}}_{c, 2}, n_2 $\} \\ \hline 
$\boldsymbol{\Phi}_{21}$ & Lensing parameters of image 2 relative to image 1 & ${\mu_{21}, \Delta t_{21}, \Delta n_{21}}$ \\ \hline 
\end{tabular}
\end{center}
\caption{Lensing parameters and source parameters defined in context of a pair of strongly lensed GW signals.}
\label{tab:notation-table}
\end{table}

\subsection{Joint parameter estimation for strong lensing}

Under the hypothesis that two GW events are strongly lensed, the two data streams can be written as $\bm{d}_1$ and $\bm{d}_2$, and the joint parameter estimation aims to measure 
\begin{equation}
p(\boldsymbol{\Theta}_1, \boldsymbol{\Theta}_2 | \bm{d}_1, \bm{d}_2, \mathcal{H}_l)\, ,
\end{equation}
where $\mathcal{H}_l$ denotes the strong lensing hypothesis. 
However, given $\boldsymbol{\Theta}_1$ and $\Phi_{21}$, we can calculate $\boldsymbol{\Theta}_2$ using Eq.(\ref{eqn:mu21})-(\ref{eqn:n21}). 
Therefore, we can equivalently infer
\begin{equation}
p(\boldsymbol{\Theta}_1, \Phi_{21} | \bm{d}_1, \bm{d}_2, \mathcal{H}_l)\, .
\end{equation}
Using Bayes' theorem we can write 
\begin{eqnarray}
\label{eqn:jpe-eqn}
&& p(\boldsymbol{\Theta}_1, \boldsymbol{\Phi}_{21} | \bm{d}_1, \bm{d}_2, \mathcal{H}_l) \nonumber\\
&& = \frac{\pi(\boldsymbol{\Theta}_{1}, \boldsymbol{\Phi}_{21} | \mathcal{H}_l) p( \bm{d}_1, \bm{d}_2 | \boldsymbol{\Theta}_1, \boldsymbol{\Phi}_{21}, \mathcal{H}_l)}{\mathcal{Z}_l} \,,
\end{eqnarray}
where $\pi(\boldsymbol{\Theta}_1, \boldsymbol{\Phi}_{21} | \mathcal{H}_l)$ is the prior on $\boldsymbol{\Theta}_1$ and $\boldsymbol{\Phi}_{21}$, 
$p( \bm{d}_1, \bm{d}_2 | \boldsymbol{\Theta}_1, \boldsymbol{\Phi}_{21}, \mathcal{H}_l)$ is the likelihood, and $\mathcal{Z}_l$ is the evidence under the strong lensing 
hypothesis\footnote{Excluding selection and population effects; see Ref.~\cite{Lo:2021nae} for more details.}: 
\begin{equation}
\mathcal{Z}_l = \int \mathrm{d}\boldsymbol{\Theta}_1 \mathrm{d}\boldsymbol{\Phi}_{21} \,  p(\boldsymbol{\Theta}_1, \boldsymbol{\Phi}_{21} | \mathcal{H}_l) \,  p(\bm{d}_1, \bm{d}_2 | \boldsymbol{\Theta}_1, \boldsymbol{\Phi}_{21}, \mathcal{H}_l) \,. 
\label{eqn:zL}
\end{equation}

Additionally, the joint likelihood in Eq.(\ref{eqn:jpe-eqn}) can be decomposed as
\begin{align}
\label{eqn:jpe-small-eqn}
p(\bm{d}_1, \bm{d}_2 | \boldsymbol{\Theta}_1, \boldsymbol{\Phi}_{21}) &= p(\bm{d}_1|\boldsymbol{\Theta}_1, \boldsymbol{\Phi}_{21}) \, p(\bm{d}_2 | \boldsymbol{\Theta}_1, \boldsymbol{\Phi}_{21}) \nonumber \\ 
& = p(\bm{d}_1 | \boldsymbol{\Theta}_1) \,  p(\bm{d}_2 | \boldsymbol{\Theta}_1, \boldsymbol{\Phi}_{21}) \,,
\end{align}
hinting at a possible way to perform relative binning analyses, where each of the terms above can be computed using the relative binning approximation before being combined.

\subsection{Relative binning for strong lensing}

The likelihood $p(\bm{d}_1| \boldsymbol{\Theta}_1)$ in Eq.~\eqref{eqn:jpe-small-eqn} is similar to the one expressed in Eq.(\ref{eqn:indi-likelihood}), except for the Morse factor, 
which is a simple constant phase shift to add to the waveform. So, the framework in Eqs.~(\ref{eqn:l-ratio})-(\ref{eqn:hh}) of Sec.~\ref{sec:rb1} to estimate the likelihood 
for individual events is also applicable here.
First, we choose appropriate fiducial parameters $\boldsymbol{\Theta}'_1$ and calculate the summary data $\mathcal{D}_1$ and the coefficients $\mathcal{R}_1$ for event 1.
Using $\mathcal{D}_1$ and $\mathcal{R}_1$, we can compute $\ln\mathcal{L}_{d_1h_1}$ and $\ln\mathcal{L}_{h_1h_1}$ as shown in Eqs.~(\ref{eqn:dh}) and 
(\ref{eqn:hh}). 
The likelihood $p(\bm{d}_1|\boldsymbol{\Theta}_1)$ is then obtained by
\begin{equation}
\label{eqn:ll-one}
\ln p(\bm{d}_1|\boldsymbol{\Theta}_1) = \ln\mathcal{L}_{d_1h_1} - \frac{1}{2} \ln\mathcal{L}_{h_1h_1} - \frac{1}{2} (\bm{d}_1 | \bm{d}_1) \,,
\end{equation}
where $(\bm{d}_1 | \bm{d}_1)$ again needs to be computed only once for the given GW data.  

The waveform $\bm{h}_1$ used to calculate the likelihood of $\bm{d}_1$ given $\boldsymbol{\Theta}_1$ can be rescaled 
to obtain $\bm{h}_2$ using Eq.(\ref{eqn:wf-mapping}) and $\boldsymbol{\Phi}_{21}$. 
The rescaled waveforms $\bm{h}_2$ can be used when sampling the likelihood of $\bm{d}_2$ given $\boldsymbol{\Theta}_2$;  
this approach saves additional computational costs that would be incurred in generating completely new waveforms. 
The rest of the steps are similar to the ones performed when calculating $p(\bm{d}_1|\boldsymbol{\Theta}_1)$, and we obtain 
\begin{equation}
\label{eqn:ll-two}
\ln p(\bm{d}_2|\boldsymbol{\Theta}_1, \boldsymbol{\Phi}_{21}) = \ln\mathcal{L}_{d_2h_2}-\frac{1}{2} \ln\mathcal{L}_{h_2h_2} - \frac{1}{2} (\bm{d}_2 | \bm{d}_2) \,.
\end{equation}

The joint likelihood defined in Eq.(\ref{eqn:jpe-small-eqn}) can be then calculated from Eqs.~(\ref{eqn:ll-one}) and (\ref{eqn:ll-two}) as
\begin{eqnarray}
&& \ln p(\bm{d}_1, \bm{d}_2 | \boldsymbol{\Theta}_1, \boldsymbol{\Phi}_{21}) \nonumber\\
&& = \ln\mathcal{L}_{d_1h_1} - \frac{1}{2} \ln\mathcal{L}_{h_1h_1} - \frac{1}{2} (\bm{d}_1 | \bm{d}_1) \nonumber \\
&& \quad + \ln\mathcal{L}_{d_2h_2}-\frac{1}{2} \ln\mathcal{L}_{h_2h_2} - \frac{1}{2} (\bm{d}_2 | \bm{d}_2) \,.
\end{eqnarray}
The combination of the relative binning for the first image and the waveform recycling to produce waveforms for the second image leads 
to a faster joint parameter estimation for strongly-lensed GW signals.



\section{Validation of the method}
\subsection{On simulated data}
\begin{figure}[h]
    \centering
    \includegraphics[width=.75\textwidth]{../figures/pp_3.0_partial_no_railing.pdf}
    \caption{Percentile-percentile showcasing the robustness of relative binning method on BBH events. Each line traces the diagonal, indicating that the parameter corresponding to it is recovered with expected accuracy. The numbers in the brackets of the legend show the p-values of the KS tests. The combined p-value of all parameters is 0.6625, consistent with the hypothesis that individual p-values were derived from a uniform distribution as expected. The shaded regions show 1$\sigma$,2$\sigma$,3$\sigma$ confidence intervals in decreasing order of opacity.}
    \label{fig:pp-plot}
\end{figure}

\subsection{On real data: analysing events detected by LIGO-Virgo}
\begin{figure}[h]
    \centering
    \includegraphics[width=.75\textwidth]{../figures/overlap_production_gw190412_gwtc.pdf}
    %\includegraphics[width=.75\textwidth]{../figures/overlap_production_gw190814_corr_branch_gwtc.pdf}
    \caption{Analysis of GW190412. The blue colour marks the results obtained from relative binning method and grey colour marks the results reported in GWTC-3 data release.}
    \label{fig:gwtc3-events}
\end{figure}

\begin{figure}[h]
    \centering
    \includegraphics[width=.75\textwidth]{../figures/overlap_production_gw190814_corr_branch_gwtc.pdf}
    \caption{Analysis of GW190814. The blue colour marks the results obtained from relative binning method and grey colour marks the results reported in GWTC-3 data release.}
    \label{fig:gw190814}
\end{figure}


\begin{figure}[h]
    \centering
    \includegraphics[width=0.53\textwidth]{../figures/thesis_pair_191103_191105.pdf} \hspace{.5cm}
    \includegraphics[width=0.41\textwidth]{../figures/lens_pair_191103_191105.pdf}
    \caption{Investigating the pair GW191103 (red) and GW191105 (green) under strong lensing hypothesis (blue).}
    \label{fig:lensed_pair}
\end{figure}

\subsection{Third generation events}

\begin{figure}[h]
    \centering
    \includegraphics[width=.75\textwidth]{../figures/prod_third_gen.pdf}
    \caption{Third generation injection example}
    \label{fig:third-gen}
\end{figure}


\begin{figure}[h]
    \centering
    \includegraphics[width=.72\textwidth]{../figures/overlapping_signals_A.pdf}\vspace{.5cm}
    \includegraphics[width=.72\textwidth]{../figures/overlapping_signals_B.pdf}
    \caption{Analysis of overlapping signals using joint parameter estimation}
    \label{fig:overlapping-signal}
\end{figure}

\section{Comparison of speed-up and accuracy}

\begin{figure}[h]
    \centering
    \includegraphics[width=.5\textwidth]{../figures/thesis_speed_up_summary.pdf}
    \caption{Relation between the speed up achieved by the the method in relation to the total mass}
    \label{fig:speed}
\end{figure}




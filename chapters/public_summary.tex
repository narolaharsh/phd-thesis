\textbf{\textit{What are gravitational waves?}} Imagine sitting by the Oudegracht (old canal in Utrecht) on a quiet sunny day. If someone throws a stone in the canal, it will generate ripples in the water. Things like the stone's weight, shape, and how fast it was thrown may determine the properties of the ripples (amplitude and frequency mainly). Let's assume, for some reason (which will become clear later), we could not see the stone falling in the water. We could only see the ripples. In this situation, is it possible to tell something about the properties of the stone (weight, shape, speed) simply by looking at the ripples? Indeed, if we have a physical theory that predicts the properties of the ripples given the properties of the stone, we may be able to deduce a few things. 

Similarly, the gravitational waves (GWs) are ripples in spacetime caused by movements of heavy objects. We cannot see these objects with the naked eye from Earth since they are too far and/or do not emit any light. However, we could still deduce a few things about these objects if we could detect the ripples they produce \textit{and} if we have a theory that can predict the properties of the waves given on the properties of the object. In fact, we do have such a theory, and it is Einstein's theory of relativity. However, nearly a 100 years after Einstein published the theory of relativity, we still did not have instruments sensitive enough to detect the gravitational waves. Einstein also believed them to be too weak to have any consequence. \\ 

\textbf{\noindent\textit{How do we detect them?}} When a gravitational wave passes by, it moves the objects in its surroundings. The instruments to detect gravitational waves measure relative movements between two test masses (mirrors). On 14th September 2015, two LIGO detectors made the first direct observation of a gravitational wave signal, GW150914, from the merger of two black holes (massive objects with gravity so strong that even light cannot escape). Specifically, they observed the last few cycles of two black holes orbiting around each other, coming closer with each cycle, and merging to create a bigger black hole. The merger released a tremendous amount of energy into the space, which reached the Earth in the form of ripples in spacetime -- gravitational waves. The GW150914 signal moved the test masses by around $10^{-18}$ meters, and the detectors indeed measured such tiny movements. To give an example, the measurement accuracy here amounts to observing the second nearest star (nearest being the sun) from Earth move by the width of a human hair. Between September 2015 and August 2025, the network of two LIGOs, located in Hanford and Louisiana, and the Virgo detector located in Pisa, has detected around 300 gravitational wave signals. Such detections have enabled us to probe exciting new physics. \\

\textbf{\noindent\textit{What is a glitch? Why is it problematic?}} Recalling the water-ripple analogy, besides the sources we want to observe (stones), there may be many other unwanted disturbances (like a bike falling in the canal). Similarly, besides an astrophysical gravitational wave signal, there are several terrestrial sources which may make the test masses of the detector move, creating a false impression of a GW passing by. We refer to such terrestrial sources as the \textit{noise} sources. Specifically, the short-lived (up to a few seconds or minutes) noise sources are known as \textit{glitches}. 

Glitches are problematic. Sometimes they can masquerade as GW signals. Sometimes they occur exactly at the same time as when a gravitational wave reaches the detector, which corrupts the signal. Glitches deteriorate the quality of the data, hindering the transition to precision science. Especially for the next generation of GW detectors, such as the Einstein Telescope, it is crucial to develop a methodology to remove glitches from the data. In chapter 4 of my thesis, we developed such a method exploiting the triangle design to remove glitches overlapping with the signal, taking an important first step towards doing precision science. Also, such analyses quantify the advantage of the triangular geometry of the Einstein Telescopes over the alternative geometries for the first time. \\ 

\textbf{\noindent\textit{What can we learn from GW signals?}} Once we establish that the GW signal is ``clean'', i.e. without any glitches, we can estimate several properties of the objects that produced them. Indeed, we could estimate the properties of the black hole merger, GW150914. Among other things, we estimated that the GW signal travelled 1.3 billion light-years to reach the Earth, and it was primarily located in the southern hemisphere. We estimated that the two black holes were around 36 and 29 times the mass of the sun, and their merger created a black hole 62 times heavier than the sun. The remaining mass is released in the form of energy carried away by GWs. We also estimated how fast the black holes were spinning and their orientations. 

While measuring the source parameters is the key to several downstream analyses (an example below), it is a computationally intensive process. The state-of-the-art computational packages require about a few days to a week to analyse one signal. To address the growing computational cost of LIGO and Virgo and to create a stepping stone for the transition to the Einstein Telescope era, faster parameter estimation techniques are required. To that end, we developed a methodology to make parameter estimation faster and extensively tested it for various configurations in Chapter 5. \\

\textbf{\noindent\textit{Can we test the predictions of general relativity with gravitational lensing?}} The last chapter of my thesis concerns testing the general relativity using strongly lensed gravitational waves -- a phenomenon where the GWs suffer a deflection and may split into multiple \textit{images}, due to a heavy object in their path. Strong lensing helps observe the same GW signal multiple times, i.e., if we have 3 detectors (two LIGOs and the Virgo) and if we detect 4 lensed images, we virtually have observed the signal 12 times, resulting in accurate sky-location and distance measurement. Through simulations, we show that this feature of strongly lensed GWs can be exploited to test the predictions of (beyond) general relativistic theories, highlighting the importance of the searches for strongly lensed gravitational waves. 
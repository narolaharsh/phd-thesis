\textit{What are gravitational waves?} -- Imagine sitting by the Oudegracht (old canal in Utrecht) on a sunny day. If someone throws a stone in the canal or if a boat passes by, it disturbs the water. The disturbance propagates through ripples. Different sources of disturbances (stones, boats) may generate different types of ripples (frequency, amplitude, etc.). If we have a physical theory that can predict the properties of the ripples given the source of disturbance, we may also be able to deduce the source of disturbance. More concretely, just by examining the ripples (let us assume we cannot directly see the source; the assumption will become clear later), we may be able to tell what kind of source caused them. Depending on how carefully we can examine the ripples, i.e., with naked eye or with a sophisticated device, we could tell more about the source, e.g., how big the boat was or how fast the stone was thrown. 

Similarly, the gravitational waves (GWs) are ripples in spacetime caused by movements of massive objects. If we have a theory that can predict the properties of the waves (and also show such waves exist in the first place) depending on the type of the source, we may be able backtrack like the example above. In fact, we do have such a theory and it is Einstein's theory of relativity. Though Einstein's theory predicts the existence of GWs, he believed them to be too weak to be of any consequence, i.e., we might not have instruments in the near future which are sensitive enough to observe gravitational waves. We definitely do not expect to see the sources of GW or GWs themselves with the naked eye from Earth. The beliefs remained true for nearly a 100 years after Einstein published the theory of relativity in 1915. 

\noindent\textit{How do we detect them?} -- When a gravitational wave passes by, it indeed moves the objects in the surroundings. The GW detectors specifically measure relative movements between two test masses (mirrors) to detect GW signals. On 15th September 2015, two LIGO detectors made the first direct observation of a gravitational wave signal, GW150914, from the merger of two black holes (massive objects with gravity so strong that even light cannot escape). Specifically, they observed the last few cycles of two black holes orbiting around each other, coming closer with each cycle, and merging to create a bigger black hole. The merger released a tremendous amount of energy into the space, which reached the Earth in the form of ripples in spacetime -- gravitational waves. The GW150914 signal moved the test masses by around $10^{-21}$ meters and the detectors were indeed able to measure such tiny movements. To give an example, the measurement accuracy here amounts to observing the second nearest star (nearest being the sun) from Earth move by the width of a human hair. Between September 2015 and August 2025, the network of two LIGOs, located in Hanford and Louisiana, and the Virgo detector located in Pisa, has detected around 300 gravitational wave signals.

\noindent\textit{What can we learn from them?} -- Unlike the sources of water ripples, we cannot see the source of gravitational waves with the naked eye from Earth. However, we can infer their properties, i.e., infer the properties of black holes and neutron star mergers producing GWs. Indeed, we estimated the properties of the black holes merger, GW150914, using the data collected by LIGO and the predictions of the general relativity. Among other things, we estimated that the GW signal travelled 1.3 billion lightyears to reach the Earth and it was primarily located in the southern hemisphere. We estimated that the two black holes were around 36 and 29 times the mass of the sun, and their merger created a black hole with 62 times heavier than the sun. The remaining mass released in the form of energy carried away by GWs. We also estimated how fast the black holes were spinning and their orientations. These fundamental measurements of black hole (and sometimes neutron stars) properties are then used to probe exciting new physics. 

\noindent\textit{What is glitch mitigation and why do we need it?}
Recalling the water-ripple analogy, besides the sources we want to observe (boats and stones), there may be many other unwanted disturbances (someone falling in the canal or the current of the water itself). Similarly, besided an astrophysical gravitational wave signal, there are several terrestrial sources which may make the test masses of the GW detector move, creating a false impression of a gravitational wave passing by. We refer to such terrestrial sources as the \textit{noise} sources. The noise sources can be broadly divided into two: long-lived and short-lived. The long-lived noise sources contribute to the constant hum of the detectors also known as the background noise (like the water current). The short-lived or transient noise source are known as \textit{glitches} (like someone jumping in the water).

Glitches are problematic. They can masquarade as GW signals. Sometimes they occur exactly at the same time as when a GW signal the detector which corrupts the signal. Especially for the next generation of GW detectors, such as Einstein Telescope, it is crucial to develope methodology to remove glitches from the data. In chapter 4 of this thesis, we have demonstrated such a method exploiting the triangle design to remove glitches overlapping with the signal, taking an important first step towards doing precision science. 


\textit{What are gravitational waves?}
Imagine sitting by the Vaartsche Rijn canal on a sunny day. When a boat or a flock of ducks passes, it disturbs the water. The disturbance propagates through ripples. Similarly, if you throw a stone in the canal, it also creates ripples. We can speculate that different sources of disturbances (ducks, boats, stones) may generate different types of ripples (frequency, amplitude, etc.) in the water. If we have a physical theory that can predict the properties of the ripples given the type of source, we might be able to backtrack the problem. Precisely, just by examining the ripples (let us assume we cannot directly see the source; the assumption will become clear later), we might be able to tell what kind of source caused them. Depending on how carefully we can examine the ripples, i.e. by naked eye or with a sophisticated device, we might be able to tell more about the source, e.g. how big was the ship, how big was the flock, or how fast the stone was thrown. 

Similarly, we can think of gravitational waves (GWs) as ripples in spacetime caused by movements of massive objects. Following the analogy in the paragraph above, if we have a theory can predict the properties of the waves (and also show such waves exist in the first place) depending on the type of the source, we may be able to learn something about the source. In fact, we do have such a theory and it is the Einstein's theory of relativity. Though Einstein's theory predicts the existence of GWs, he believed them to be too weak to be of any consequence, i.e., we might not have instruments in near future which are sensitive enough to observe gravitational waves. We definitely do not expect to see them with naked eye from Earth. The beliefs remained true for nearly a 100 years after Einstein published the theory of relativity in 1915. 

On 15th September 2015, two LIGO detectors made the first direct observation of a gravitational wave signal (GW150914) from the merger of two black holes (massive objects with gravity so strong that even the light cannot escape). Specifically, we observed last few cycles of two black holes orbiting around each other, coming closer with each cycle, and merging to create a bigger black hole. The merger released trumenduous amount of energy released into the space which reached the Earth in the form of ripples in spacetime -- gravitational waves. Between September 2015 and August 2025, the network of two LIGO located in Hanford and Louisiana and the Virgo detector located in Pisa have detected around 300 gravitational wave signals.

Unlike the water ripples, we cannot see the source of gravitational waves (binary black holes or other compact objects such as neutron stars) with naked eye from Earth. However, we can carefully examine the gravitational waves to infer properties of the black holes and neutron stars -- some of the most exotic astrophysical objects. Comparing the data collected by LIGO with the predictions of general theory of relativity, we were able to infer many properties of the black holes that generated the GW signal. Among other things, we estimated that the GW signal travelled 1.3 billion lightyears to reach the Earth and the source was primarily located in the southern hemisphere. We estimated that the two black holes were around 36 and 29 times the mass of the sun and their merger created a black hole with mass 62 times mass of the sun. The remaining mass was released in the form of energy carried away by GWs. When a GW passes by, it indeed moves the objects in the surrounding. The detectors specifically measure relative movements between two test masses (mirrors) to detect GWs. GW150914 signal moved the masses by around $10^{-21}$ meters, which were indeed able to measure. It amounts to observing the motion of the second nearest star (nearest being the sun) from earth move by the width of a human hair.  

\textit{What is glitch mitigation and why do we need it?}

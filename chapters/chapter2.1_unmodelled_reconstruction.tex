\begin{itemize}
    \item What is meant by generic excess power? What can cause it? unmodelled signals, glitches, signals overlapping glitches
    \item Transdimensional sampling and RJMCMC
    \item BayesWave: Morlet Gabour wavelet. Q-scan and bayeswave reconstruction
\end{itemize}

\begin{figure}
    \centering
    \includegraphics[width=12cm]{../figures/morlet_gabour.pdf}
    \caption{GW signal}
    \label{fig:gw-qscan}
\end{figure}
When reconstructing GW signals to measure source parameters, we typically use a modelled approach, i.e. relying on GW signal models predicted by general relativity [Phenom, SEOBNR, NRSur]. However, there are cases when such models either do not exist or are expensive to simulate or are not mature enough. For such instance, it becomes important to be able to perform \textit{unmodelled} reconstruction of the source. In this chapter, we review the instance when such unmodelled reconstruction is necessary and how it is done. 




Glitches can masquarade as GW signal, e.g. blip glitches have similar time-frequency evolutions as a GW signal from a high mass binary black hole merger. They can increase false alarm of GW searches. In addition, when overlapping with a GW signal, they can corrupt the measurement of the source parameters or can be mistaken for signs of new physics. 

When reconstruction various GW signal to measure source parameters, we rely on a theoretical model. However, there are cases where such theoretical models do not exist, e.g. wide varities of glitches, GW signal from echoes, or are expensive to generate, e.g. core-collapse supernovae. For such scenarios, it become important to devise tools that can perform unmodelled reconstruction of any excess power above the Gaussian noise background. 

Paragraph 1: What is generic excess power? What can cause it (glitches mainly)?
Paragraph 2: Why it is important to be able to reconstruct generic excess power? Removing glitches, glitches overlapping signal, searches for echoes, supernovae, test of GR, estimating Power Spectral Density

\section{Transdimensional Sampling Algorithms}
\section{BayesWave}

\begin{figure}
    \centering
    \includegraphics[width=12cm]{../figures/morlet_gabour.pdf}
    \caption{Morlet-Gabour wavelet in time (left) and frequency (right) domain}
    \label{fig:mg_wav}
\end{figure}

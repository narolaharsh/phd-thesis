

General relativity predicts that gravitational waves are non-dispersive (all frequency components travel at the same speed), travel at the speed of light, their amplitude is inversely proportional to the distance to the source. The non-dispersive nature has been a part of a suite of testing GR analyses carried out by LVK. The multimessenger detection of GW170817 helped place stringent constraints on the difference between the speed of GWs and the speed of light. Specifically, it was measured that the difference between the speed of GWs and light is between $-3\times10^{-15}$ and $6\times10^{-16}$ times the speed of light. The identification of the EM counterpart of GW170817 also helped constrain any modifications to the distance-amplitude relation. The relation in GR is:

\begin{equation}
\text{Amplitude} \propto \frac{1}{\text{Distance to the source}}. 
\end{equation}

However, the constraints obtaind on modified propagation theories using GW170817 remain relatively poor as will become clear later. When discussing modified propagation theories, we specifically refer to the ones that predict a different distance-amplitude relation than GR. In this chapter, we demonstrate how the detection of a strongly lensed GW signal can significantly improve upon beyond general relativistic theories that allow for modified propagation of GWs. This chapter contains results from our recent work on this topic~\cite{Narola:2023viz}.


\section{Gravitational lensing and distance measurements}
First, I briefly recap the discussion on the gravitational lensing of gravitational waves. 


\subsubsection{Electromagnetic distance}
%A strongly lensed GW source will have an improved sky localization compared to a non-lensed source, as we can observe the former multiple times with different detector orientations~\cite{Janquart:2021qov, Lo:2021nae, %Hannuksela2020LocalizingLensing}. Especially with four detectable images, we may be able to localize the source within $\mathcal O(1) $ square degrees~\cite{Hannuksela:2020xor, Wempe:2022zlk}. %Janquart2021AEvents
%Various lensing rate estimates show that such a scenario where strong lensing produces four images (a quadruplet) of GW signal is fairly likely. More concretely, about $30\%$ of the strong lensing scenarios are %expected to produce quadruplets.

When a GW source is lensed, we can expect that the electromagnetic (EM) radiation coming from its host 
galaxy is also lensed, as is widely assumed in cosmography 
studies~\cite{Chen:2017rfc, LIGOScientific:2018gmd, Gray:2019ksv, LIGOScientific:2019zcs, DES:2019ccw}. 
A joint GW+EM analysis can help locate the source's host galaxy once its location is narrowed down to a 
few square degrees using only GW data.
In this step, one reconstructs all the lenses in the region provided by the GW data to find which lens 
could best produce a GW quadruplet with 
properties similar to the ones observed; the galaxy that is undergoing lensing by this particular lens is then likely to be the host galaxy of the GW event.
This methodology was first proposed in~\cite{Hannuksela:2020xor, Wempe:2022zlk}. For the scenarios where strong lensing produces four images (a quadruplet), we can narrow down the host of the GWs source to one or a few galaxies. Various lensing rate estimates show that $\approx 30\%$ of all strong lensing scenarios will produce a quadruplet, making this a fairly likely scenario~\cite{Li:2018prc, Wierda:2021upe}.

Once the host galaxy is known, a dedicated spectroscopic or photometric follow-up can lead us to the redshift of the source.
By combining the source's redshift with a cosmological model, we can estimate the source's luminosity distance in a way 
that is unaffected by the anomalous GW propagation~\cite{Hogg:1999ad}. Let us refer to this distance as the electromagnetic distance $\DEM$ to the source. 

\subsubsection{Gravitational wave distance}
Besides the electromagnetic distance ($\DEM$), we can have another independent measurement of the source's luminosity distance from the GW data as explained in chapter~\ref{ch:gwpe}. Let us refer to this distance as the gravitational wave distance $\DGW$ to the source. The $\DGW$ distance measurement could be affected by the anomalous propagation of gravitational waves while $\DEM$ may not. Therefore, by comparing the two distances, $\DGW$ and $\DEM$, the anomaly in the gravitational wave propagation can be discovered or bounded.

\subsubsection{Strongly lensed signals: A probe complementary to the BNS signals}
Studying modified propagation theories in the context of strongly lensed and localized GW events, especially from 
binary black hole (BBH) coalescences, is attractive, because such events can help probe higher redshift regimes compared to binary neutron star (BNS) events. 
In the past, modified propagation theories have been tested using 
GW170817 \cite{Mastrogiovanni:2020mvm, Lagos:2019kds, LIGOScientific:2018dkp, Pardo:2018ipy}, 
a signal from a BNS inspiral with an identifiable EM counterpart 
\cite{LIGOScientific:2017vwq, LIGOScientific:2017ync}. However, by cosmological standards, the GW170817 signal 
travelled only a small distance before it reached the detectors, and 
in modified propagation theories, the imprint of the deviation tends to accumulate with distance. 
Other methods have been proposed that exploit the population properties of BBH coalescences observed with GWs 
\cite{Ezquiaga:2021ayr,Leyde:2022orh,MaganaHernandez:2021zyc}; since BBHs can be detected out to larger 
distances, this enables considerably improved bounds over the ones from GW170817. 
Due to magnification, GWs from \emph{lensed} BBH events can potentially be seen out to redshifts 
$z \sim 6$~\cite{Wierda:2021upe}, this feature leads to furhter improvements in our measurements. 
%This paper aims to quantify the gain from GW lensing for the different anomalous propagation scenarios considered.

\section{Modified propagation theories}
\label{sec:mpt}

As explained above, our tests of modified theories of gravity will be based on a comparison between
the reconstructed $\DGW$ and the luminosity distance $\DEM$ obtained by electromagnetic 
means.\footnote{Here we will focus exclusively on anomalous propagation affecting the amplitude of GWs, but 
for models that lead to dispersion, the effect on the GW \emph{phasing} of BBH signals has been used to place  
very stringent constraints 
\cite{LIGOScientific:2016aoc,LIGOScientific:2016lio,LIGOScientific:2017bnn,LIGOScientific:2019fpa,LIGOScientific:2020tif,LIGOScientific:2021sio}.  
In addition, the difference between the times of arrival of GW170817 and the associated gamma ray burst has 
enabled strong constraints on differences between the speed of gravitational waves and the speed of light 
\cite{LIGOScientific:2017zic}.} 
In the specific modified gravity models we consider -- large extra dimensions, $\Xi$-paramaterization, and varying Planck mass -- there is a non-trivial relationship between these 
two quantities, which will depend on the parameter(s) related to the deviation from GR and on the cosmological
parameters. Let us briefly recall what these relationships look like for our three models.




\subsubsection{Large extra spatial dimensions}
In theories of gravity with large extra dimensions, there is the possibility of some energy of the GWs
leaking into them \cite{Dvali:2000hr, Calcagni:2019kzo, LIGOScientific:2018dkp}, while
EM radiation is confined to the usual three spatial dimensions. This would make the detected signal 
appear weaker, leading to larger measured values for $\DGW$ than would otherwise be the case. For definiteness, we
will work with the following simple phenomenological ansatz for the relation between $\DGW$ and $\DEM$, 
based on conservation of integrated flux \cite{Deffayet:2007kf}:
\begin{equation}
	\label{eqn:D-model}
	\DGW = (\DEM(z_s, H_0))^{\frac{D-2}{2}},
\end{equation}
where $D$ is the number of spacetime dimensions and $z_s$ is the source redshift.
We will allow $D$ be a real number, with the GR value $D = 4$ as a fiducial value.  
An illustration of the effect of extra dimensions on a GW waveform is given in the top 
panel of Fig.~\ref{fig:wf-effects}.


\subsubsection{$\Xi-$parameterization}
Another parameterization 
was proposed in~\cite{Belgacem:2018lbp}, where the link between $\DGW$ and $\DEM$ is expressed as
\begin{equation}
	\label{eqn:xi-model}
	\DGW = \DEM(z_s, H_0)\left[\Xi_0+\frac{1-\Xi_0}{(1+z_s)^n}\right]\,.
\end{equation}
The free parameters of the model are $(\Xi_0, n)$. This parameterization is phenomenological in 
nature, but as shown in \cite{LISACosmologyWorkingGroup:2019mwx} it can be related to a large class of 
modified gravity theories, 
including Horndeski~\cite{Horndeski:1974wa} theories, 
Degenerate Higher Order Scalar-Tensor theories (DHOST)~\cite{Langlois:2018dxi}, 
and theories with nonlocally modified gravity~\cite{Maggiore:2013mea, Maggiore:2014sia,Belgacem:2017cqo}. 
When $z \ll 1$, $\DGW \simeq \DEM$. 
Therefore, similar to the extra dimension theories, we expect to observe a departure from 
GR only at large distances ($z \gtrsim 1$). For GR, 
$\Xi_0 = 1$ and $n$ is degenerate. 
In Fig.~\ref{fig:wf-effects}, middle panel, one can see an illustration of 
the effect of this modified propagation theory on the observed GW signal.

\subsubsection{Time-varying Planck mass}
A time-varying Planck mass is another possible cause for 
modified GW propagation. Following~\cite{Lagos:2019kds}, 
the relation between $\DGW$ and $\DEM$ can be expressed as
\begin{equation}
	\label{eqn:cm-model}
	\DGW(z) = \DEM(z_s, H_0) \times \\
	\exp{\left(\frac{c_M}{2\Omega_{\Lambda}}\ln{\frac{1+z_s}{(\Omega_{m}(1+z_s)^3 + \Omega_{\Lambda})^{1/3}}}\right)} \,,
\end{equation}
where $c_M$ is a constant that relates the rate of change of the Planck mass with the
fractional dark energy density in the Universe; for details, see \cite{Lagos:2019kds} and 
references therein. For GR, $c_M = 0$. The bottom panel of Fig.~\ref{fig:wf-effects} illustrates the change in 
a GW signal in the non-GR case.

\begin{table}
    \centering
	\begin{tabular}{ c | c | c }
		\hline\hline
		Theory                 & Parameter & Priors                 \\
		\hline
		Large extra dimension        & $D$         & Uniform(3, 5)          \\
		\hline
		$\Xi$-parameterization & $\Xi_0$   & Log Uniform(0.01, 100) \\
		                       & $n$       & Uniform(0, 10)         \\
		\hline
		Running Planck mass    & $c_M$     & Uniform(-150, 150)     \\
		\hline
		
	\end{tabular}
	\caption{Deviation parameter(s) for each theory and the corresponding prior probability
	distributions used in our analyses.}
	\label{tab:priors}
\end{table}




\section{Method}


\section{Results}

\begin{figure}
    \centering
    \includegraphics[width=\textwidth]{../figures/tgr_wf_effects.pdf}
    \caption{The effect on the time domain GW signal in each of the modified propagation models. For each model, the GR waveform is shown in black colour and deviation from it by different amounts is shown in different colours. 
 In these examples, the GW source is assumed to be at $\sim 5$ Gpc and the masses of the source are similar to GW150914~\cite{LIGOScientific:2016aoc}. For the running 
 Planck mass model, the deviation is absolute since one has $c_M = 0$ in GR.  
 For large extra dimensions and $\Xi$-parameterization, we consider percentage deviation in the 
 parameters $D$ and $\Xi_0$, taking the fiducial values to be $D = 4$ and $\Xi = 1$, respectively. 
 For the $\Xi$-parameterization, we arbitrarily choose $n = 1$ here, though in our subsequent analyses
 it will be a free parameter.}
    \label{fig:wf-effects}
\end{figure}

\begin{figure}
    \centering
    \includegraphics[width=\textwidth]{../figures/thesis-2D-deviation-plots.pdf}
    \caption{Qualitative estimates}
    \label{fig:qual-res}
\end{figure}


\begin{figure}
    \centering
    \includegraphics[width=\textwidth]{../figures/presentation_t_perc_deviation_parameter_column.pdf}
    \caption{Parameter estimation results}
    \label{fig:lensing-pe-results}
\end{figure}
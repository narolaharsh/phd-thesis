

General relativity predicts that gravitational waves are non-dispersive (all frequency components travel at the same speed), travel at the speed of light, their amplitude is inversely proportional to the distance to the source. The non-dispersive nature has been a part of a suite of testing GR analyses carried out by LVK~\cite{LIGOScientific:2016aoc,LIGOScientific:2016lio,LIGOScientific:2017bnn,LIGOScientific:2019fpa,LIGOScientific:2020tif,LIGOScientific:2021sio}. The multimessenger detection of GW170817 helped place stringent constraints on the difference between the speed of GWs and the speed of light. Specifically, it was measured that the difference between the speed of GWs and light is between $-3\times10^{-15}$ and $6\times10^{-16}$ times the speed of light~\cite{LIGOScientific:2017zic}. The identification of the EM counterpart of GW170817 also helped constrain any modifications to the distance-amplitude relation. The relation in GR is:

\begin{equation}
\text{Amplitude} \propto \frac{1}{\text{Distance to the source}}. 
\end{equation}

However, the constraints obtaind on modified propagation theories using GW170817 remain relatively poor as will become clear later. When discussing modified propagation theories, we specifically refer to the ones that predict a different distance-amplitude relation than GR. In this chapter, we demonstrate how the detection of a strongly lensed GW signal can significantly improve upon beyond general relativistic theories that allow for modified propagation of GWs. This chapter contains results from our recent work on this topic~\cite{Narola:2023viz}.


\section{Gravitational lensing and distance measurements}
First, I briefly recap the discussion on the gravitational lensing of gravitational waves. 

\subsubsection{Electromagnetic distance}

When a GW source is lensed, we can expect that the electromagnetic (EM) radiation coming from its host 
galaxy is also lensed, as is widely assumed in cosmography 
studies~\cite{Chen:2017rfc, LIGOScientific:2018gmd, Gray:2019ksv, LIGOScientific:2019zcs, DES:2019ccw}. 
A joint GW+EM analysis can help locate the source's host galaxy once its location is narrowed down to a 
few square degrees using only GW data.
In this step, one reconstructs all the lenses in the region provided by the GW data to find which lens 
could best produce a GW quadruplet with 
properties similar to the ones observed; the galaxy that is undergoing lensing by this particular lens is then likely to be the host galaxy of the GW event.
This methodology was first proposed in~\cite{Hannuksela:2020xor, Wempe:2022zlk}. For the scenarios where strong lensing produces four images (a quadruplet), we can narrow down the host of the GWs source to one or a few galaxies. Various lensing rate estimates show that $\approx 30\%$ of all strong lensing scenarios will produce a quadruplet, making this a fairly likely scenario~\cite{Li:2018prc, Wierda:2021upe}.

Once the host galaxy is known, a dedicated spectroscopic or photometric follow-up can lead us to the redshift of the source.
By combining the source's redshift with a cosmological model, we can estimate the source's luminosity distance in a way 
that is unaffected by the anomalous GW propagation~\cite{Hogg:1999ad}. Let us refer to this distance as the electromagnetic distance $\DEM$ to the source. 

\subsubsection{Gravitational wave distance}
Besides the electromagnetic distance ($\DEM$), we can have another independent measurement of the source's luminosity distance from the GW data as explained in chapter~\ref{ch:gwpe}. Let us refer to this distance as the gravitational wave distance $\DGW$ to the source. The $\DGW$ distance measurement could be affected by the anomalous propagation of gravitational waves while $\DEM$ may not. Therefore, by comparing the two distances, $\DGW$ and $\DEM$, the anomaly in the gravitational wave propagation can be discovered or bounded.

\section{Modified propagation theories}
\label{sec:mpt}

As explained above, our tests of modified propagation theories of gravity will be based on a comparison between
the reconstructed $\DGW$ and $\DEM$.
means. In the specific modified gravity models we consider -- large extra dimensions, $\Xi$-paramaterization, and varying Planck mass -- there is a non-trivial relationship between these 
two quantities, which will depend on the parameter(s) related to the deviation from GR and on the cosmological
parameters. Let us briefly discuss what these relationships look like for our three models.


\begin{figure}
    \centering
    \includegraphics[width=\textwidth]{../figures/tgr_wf_effects.pdf}
    \caption{The effect on the time domain GW signal in each of the modified propagation models. For each model, the GR waveform is shown in black colour and deviation from it by different amounts is shown in different colours. 
 In these examples, the GW source is assumed to be at $\sim 5$ Gpc and the masses of the source are similar to GW150914~\cite{LIGOScientific:2016aoc}. For the running 
 Planck mass model, the deviation is absolute since one has $c_M = 0$ in GR.  
 For large extra dimensions and $\Xi$-parameterization, we consider percentage deviation in the 
 parameters $D$ and $\Xi_0$, taking the fiducial values to be $D = 4$ and $\Xi = 1$, respectively. 
 For the $\Xi$-parameterization, we arbitrarily choose $n = 1$ here, though in our subsequent analyses
 it will be a free parameter.}
    \label{fig:wf-effects}
\end{figure}

\subsubsection{Large extra dimensions}
In theories of gravity with large extra dimensions, there is the possibility of some energy of the GWs
leaking into them \cite{Dvali:2000hr, Calcagni:2019kzo, LIGOScientific:2018dkp}, while
EM radiation is confined to the usual three spatial dimensions. This would make the detected signal 
appear weaker, leading to larger measured values for $\DGW$ than would otherwise be the case. For definiteness, we
will work with the following simple phenomenological ansatz for the relation between $\DGW$ and $\DEM$, 
based on conservation of integrated flux \cite{Deffayet:2007kf}:
\begin{equation}
	\label{eqn:D-model}
	\DGW = (\DEM(z_s, H_0))^{\frac{D-2}{2}},
\end{equation}
where $D$ is the number of spacetime dimensions and $z_s$ is the source redshift.
We will allow $D$ be a real number, with the GR value $D = 4$ as a fiducial value.  
An illustration of the effect of extra dimensions on a GW signal is given in the left plot of Figure~\ref{fig:wf-effects}.

\subsubsection{The $\Xi-$parameterization}
Another parameterization 
was proposed in~\cite{Belgacem:2018lbp}, where the link between $\DGW$ and $\DEM$ is expressed as
\begin{equation}
	\label{eqn:xi-model}
	\DGW = \DEM(z_s, H_0)\left[\Xi_0+\frac{1-\Xi_0}{(1+z_s)^n}\right]\,.
\end{equation}
The free parameters of the model are $(\Xi_0, n)$. This parameterization is phenomenological in 
nature, but as shown in \cite{LISACosmologyWorkingGroup:2019mwx} it can be related to a large class of 
modified gravity theories, 
including Horndeski~\cite{Horndeski:1974wa} theories, 
Degenerate Higher Order Scalar-Tensor theories (DHOST)~\cite{Langlois:2018dxi}, 
and theories with nonlocally modified gravity~\cite{Maggiore:2013mea, Maggiore:2014sia,Belgacem:2017cqo}. 
When $z \ll 1$, $\DGW \simeq \DEM$. 
Therefore, similar to the extra dimension theories, we expect to observe a departure from 
GR only at large distances ($z \gtrsim 1$). For GR, 
$\Xi_0 = 1$ and $n$ is degenerate. 
In Figure~\ref{fig:wf-effects}, middle plot, one can see an illustration of 
the effect of this modified propagation theory on a GW signal.

\subsubsection{Time-varying Planck mass}
A time-varying Planck mass is another possible cause for 
modified GW propagation. Following~\cite{Lagos:2019kds}, 
the relation between $\DGW$ and $\DEM$ can be expressed as
\begin{equation}
	\label{eqn:cm-model}
	\DGW(z) = \DEM(z_s, H_0) \times \\
	\exp{\left(\frac{c_M}{2\Omega_{\Lambda}}\ln{\frac{1+z_s}{(\Omega_{m}(1+z_s)^3 + \Omega_{\Lambda})^{1/3}}}\right)} \,,
\end{equation}
where $c_M$ is a constant that relates the rate of change of the Planck mass with the
fractional dark energy density in the Universe; for details, see \cite{Lagos:2019kds} and 
references therein. For GR, $c_M = 0$. The right plot of the Figure~\ref{fig:wf-effects} illustrates the effect on a GW signal.

\begin{table}
    \centering
	\begin{tabular}{ c | c | c }
		\hline\hline
		Theory                 & Parameter & Priors  \\
		\hline
		Large extra dimension        & $D$         & Uniform(3, 5)          \\
		\hline
		$\Xi$-parameterization & $\Xi_0$   & Log Uniform(0.01, 100) \\
		                       & $n$       & Uniform(0, 10)         \\
		\hline
		Running Planck mass    & $c_M$     & Uniform(-150, 150)     \\
		\hline
		
	\end{tabular}
	\caption{Deviation parameter(s) for each theory and the corresponding prior probability
	distributions used in our analyses. The ranges of the prior are chosen following the previous studies on modified propagation \cite{Mastrogiovanni:2020mvm, Lagos:2019kds, Pardo:2018ipy}.}
	\label{tab:priors}
\end{table}

\section{Strongly lensed signals: A probe complementary to the BNS signals}

\begin{figure}[h]
    \centering
    \includegraphics[width=\textwidth]{../figures/thesis-2D-deviation-plots.pdf}
    \caption{Fractional difference ($\Delta$) between the electromagnetic distance and gravitational wave distance for three modified propagation models as a function of source redshift (x-axis) and scale of deviation (y-axis). For the large extra dimension model and $\Xi$-parameterization, the y-axis shows percentage deviation with respect to the general relativistic value. For time-varying Plank mass, the y-axis shows the absolute deviation since the general relativistic value is 0 for the model parameter. Blue vertical line marks the redshift measurement of GW170817 signal. For the $\Xi$-parametrization and varying Planck mass, at the redshift of GW170817, the $\Delta$ is quite small even for large amounts of deviations from GR (grey region), suggesting that these models may be poorly constrained. Access to the high-redshift regime are likely to lead to better constraints on these deviation. On the other hand, the effect of extra dimensions is less sensitive to redshift, and measurements of $D$ are already quite stringent at the redshift of GW170817 and
    are not expected to improve as much as for the other two cases when going to a higher redshift.}
    \label{fig:qual-res}
\end{figure}

Studying modified propagation theories in the context of strongly lensed and localized GW events, especially from 
binary black hole (BBH) coalescences, is attractive, because such events can help probe higher redshift regimes compared to binary neutron star (BNS) events. 
In the past, modified propagation theories have been tested using 
GW170817 \cite{Mastrogiovanni:2020mvm, Lagos:2019kds, LIGOScientific:2018dkp, Pardo:2018ipy}, 
a signal from a BNS inspiral with an identifiable EM counterpart 
\cite{LIGOScientific:2017vwq, LIGOScientific:2017ync}. However, by cosmological standards, the GW170817 signal 
travelled only a small distance before it reached the detectors, and 
in modified propagation theories, the imprint of the deviation tends to accumulate with distance. 
Other methods have been proposed that exploit the population properties of BBH coalescences observed with GWs 
\cite{Ezquiaga:2021ayr,Leyde:2022orh,MaganaHernandez:2021zyc}; since BBHs can be detected out to larger 
distances, this enables considerably improved bounds over the ones from GW170817. 
Due to magnification, GWs from \emph{lensed} BBH events can potentially be seen out to redshifts 
$z \sim 6$~\cite{Wierda:2021upe}, this feature leads to furhter improvements in our measurements. 
%This paper aims to quantify the gain from GW lensing for the different anomalous propagation scenarios considered.

\section{Method}
We want to estimate the posterior probability distribution on deviation parameters given the GW data and EM data, i.e.,
\begin{equation}
p(\betaMGR, H_0|\vec{d}_{\mathrm{GW}}, \vec{d}_{\mathrm{EM}}),
\end{equation}
where we use $\betaMGR$ to denote the model parameters in all generality, the notations $\vec{d}_{GW}$ and $\vec{d}_{EM}$ represent the GW and EM data of a quadruply lensed event whose host galaxy has been identified. The parameter $H_0$ denotes the Hubble constant. 

Using Bayes' theorem, we can write 
\begin{equation}
\label{eq:pe1}
p(\betaMGR, H_0 | \vec{d}_\mathrm{GW},  \vec{d}_{\mathrm{EM}}) = \frac{p(\betaMGR, H_0)\, p(\vec{d}_{\mathrm{GW}}, \vec{d}_{\mathrm{EM}} | \betaMGR, H_0)}{Z},
\end{equation}
where $p(\betaMGR, H_0)$ is the prior probability distribution on $\betaMGR$ and $H_0$; the term $p(\vec{d}_{\mathrm{GW}}, \vec{d}_{\mathrm{EM}} | \betaMGR, H_0)$ represents the 
likelihood function; and $Z$ the evidence, whose value follows from the requirement that 
the posterior probability distribution be normalized. 
The prior distributions for $\betaMGR$ are specified in Table~\ref{tab:priors}. 
We have chosen uninformative priors on each of the $\betaMGR$ parameters in order to not favour any specific value. 

For $H_0$, we could in principle choose a relatively narrow prior range based on the Planck \cite{Planck:2006aa}, SHoES \cite{Riess:2021jrx}, or other existing measurements \cite{Wong:2019kwg}. 
Instead, we make a more conservative choice. Specifically, we use a measurement of $H_0$ that is obtained from the differences in time of arrival of the GW images together with lens reconstruction through electromagnetic means. We use the measurement of $H_0$ obtaind in such a way as the prior of our analysis. For details we refer to the work by Hannuksela \textit{et al}~\cite{Hannuksela:2020xor}; here we confine ourselves 
to recalling the so-called time delay distance $D_{\Delta t}$, which is related to $H_0$ through

\begin{equation}
\label{eqn:time-delay-dist}
D_{\Delta t}(z_l, z_s, H_0) = \frac{\int_0^{z_s} dz'/E(z')}{\int_{z_l}^{z_s} dz'/E(z')}\DEM(z_s, H_0) \, .
\end{equation} 
Here $z_l$ and $z_s$ are respectively the lens and the source redshift, and 
$E(z) \equiv \sqrt{\Omega_m (1+z)^3 + \Omega_\Lambda}$. If $D_{\Delta t}$ 
is measured, we can estimate $\DEM$ since we assume that $z_l$ and $z_s$ are known from 
the EM follow-up observations. Using the $\DEM$ measurement, $H_0$ can be estimated through
Eq.~\eqref{eqn:dl-em}. $D_{\Delta t}$ can be measured by performing lens reconstruction; 
however, owing to the computational complexity and cost, we skip the lens construction step. 
We assume that the observed value of $D_{\Delta t}$ follows a Gaussian distribution with a 10\% standard deviation. 
To allow for an offset in the observation, we pick the mean of this distribution from another Gaussian distribution with a 10\% standard deviation which is 
centered at the true value. The 10\% error assumed in the measurement of $D_{\Delta t}$ 
is motivated by the results of~\cite{Hannuksela:2020xor}. We translate the measurement of $D_{\Delta t}$ into the measurement of $\DEM$ using Eq.~\ref{eqn:time-delay-dist}. 
Using the value of $\DEM$ together with  
Eq.~\eqref{eqn:dl-em}, we construct the prior for $H_0$. 



To calculate the likelihood $p(\vec{d}_{\mathrm{GW}}, \vec{d}_{\mathrm{EM}} | \betaMGR, H_0)$,  
we first express it as
\begin{eqnarray}
\label{eq:likeli}
&&p(\vec{d}_{\mathrm{GW}}, \vec{d}_{\mathrm{EM}} | \betaMGR, H_0) \nonumber\\
&&= \int d\vec{\theta}\, dz_s\, p(\vec{d}_{\mathrm{GW}} | \vec{\theta})\, p(\vec{d}_{\mathrm{EM}} | z_s) \nonumber\\ 
&& \;\;\;\;\;\;\;\;\; \times \, p(\vec{\theta} | z_s, \betaMGR, H_0) \, p(z_s| \betaMGR, H_0) \,,
\end{eqnarray}
where $\vec{\theta}$ denotes the GW source parameters, $p(\vec{d}_{\mathrm{GW}} | \vec{\theta})$ 
and  $p(\vec{d}_{\mathrm{EM}} | z_s)$ are the likelihoods of 
the GW and EM data respectively, and $z_s$ is the source redshift. 
$p(\vec{\theta} | z_s, \betaMGR, H_0)$ and $p(z_s | \betaMGR, H_0)$ 
are the priors on the GW source parameters and redshift. 

Since we assume that the host galaxy has been localized, the true source redshift $z_s$ is known. Recent studies have shown that the spectroscopic redshift of the source can be measured to a sub-percent accuracy \cite{Zhou:2021fmt}. Therefore, we neglect the error on the measurement of $z_s$ so 
the term $p(\vec{d}_{\mathrm{EM}} | z_s)$ becomes a Dirac delta function centered on $z_s$, 
reducing Eq.~\eqref{eq:likeli} to 
\begin{eqnarray}
&& p(\vec{d}_{\mathrm{GW}}, \vec{d}_{\mathrm{EM}} | \betaMGR, H_0) \nonumber\\
&&= \int d\vec{\theta}\, p(\vec{d}_{\mathrm{GW}} | \vec{\theta}) p(\vec{\theta} | z_s, \betaMGR, H_0) \, 
p(z_s| \betaMGR, H_0) \, . \nonumber\\
&&
\end{eqnarray}

To estimate the GW likelihood $p(\vec{d}_{\mathrm{GW}}|\vec{\theta})$, we perform Bayesian parameter 
inference using nested sampling~\cite{Skilling:2006gxv} for the first image.  
Subsequently we use \textsc{Golum}~\cite{Janquart:2021qov, Janquart:2023osz} 
to speed up Bayesian parameter inference for the other images. \textsc{Golum} 
can rapidly analyse lensed images by using the posterior samples of the first image as prior 
for the subsequent images, as the source parameters for each of the four images 
are expected to be the same, apart from  
relative magnifications, rigid phase offsets, and differences in time of arrival. 
%We repeat this step for each quadruplet in our catalogue. 

Once we have the GW likelihood, we perform the integration over the $\vec{\theta}$ for all parameters except the 
luminosity distance $\DGW$, yielding
\begin{eqnarray}
\label{eqn:step-likeli}
&&p(\vec{d}_{\mathrm{GW}}, \vec{d}_{\mathrm{EM}} | \betaMGR, H_0) \nonumber\\ 
&& = \int dD_L^{\mathrm{ GW}} \, p(\vec{d}_{\mathrm{GW}} | D_L^{\mathrm{GW}}) \,
 p(D_L^{\mathrm{GW}} | z_s, \betaMGR, H_0) \nonumber\\ 
&& \;\;\;\;\;\;\;\;\; \times \, p(z_s| \betaMGR, H_0) \,.
\end{eqnarray}


The prior $p(\DGW | z_s, \betaMGR, H_0)$ reduces to a Dirac delta function as we exactly  
know $\DGW$ given the values of $z_s$, $\betaMGR$, $H_0$ and the modified gravity model (Eqs.~\eqref{eqn:D-model}, \eqref{eqn:xi-model} and \eqref{eqn:cm-model}). 
Therefore, integrating with respect to $\DGW$ leads to
\begin{equation}
	\label{eq:redu-likeli}
	p(\vec{d}_{\mathrm{GW}}, \vec{d}_{\mathrm{EM}} | \betaMGR, H_0)  = p( \vec{d}_{\mathrm{GW}} | \DGW) p(z_s|\betaMGR, H_0).
\end{equation}
Substituting Eq.~\eqref{eq:redu-likeli} into Eq.~\eqref{eq:pe1}, we can obtain the posterior distributions for $\betaMGR$ and $H_0$. 
The above derivation is performed for one of the images of the quadruplet. 
To combine the information from multiple images and obtain the joint posterior on $\betaMGR$, 
we need to express the distance likelihood $p( \vec{d}_{\mathrm{GW}} | \DGW)$ in 
Eq.~(\ref{eq:redu-likeli}) as a joint likelihood:
\begin{equation}
\label{eq:joint-distance-likelihood}
p(\vec{d}_{\mathrm{GW}} | \DGW ) = \prod_{i=0}^{3}\, p(\vec{d}_{\mathrm{GW}, i} | \DGW),
\end{equation}
where $\vec{d}_{\mathrm{GW}, i}$ refers to the GW data from image $i$. 

In what follows, we assume binary black hole coalescences with component mass distributions drawn from the \textsc{PowerLaw+Peak} in~\cite{KAGRA:2021duu}.
Our GW waveform model is IMRPhenomXPHM~\cite{Pratten:2020ceb}, with black hole spin magnitudes distributed uniformly 
between $0$ and $1$, and spin directions uniformly on the sphere.
The distribution of the redshifts of the BBH and the galaxy lenses (modelled as singular power law isothermal ellipsoids with external shear) is obtained from~\cite{Wierda:2021upe}.
The fiducial values of $\betaMGR$ are equal to their GR values. The fiducial value of the Hubble constant is 
$H_0 = 67.4$ km $\text{s}^{-1}$ $\text{Mpc}^{-1}$, and $\Omega_m = 0.315$. 
The lensed GWs were analyzed using \textsc{Golum}~\cite{Janquart:2021qov, Janquart:2023osz} and \textsc{Dynesty}~\cite{Speagle2019Dynesty:Evidences} to produce the $\DGW$ posteriors along with other source parameters. 
Our detector network consists of two LIGO~\cite{LIGOScientific:2014pky}, the Virgo~\cite{VIRGO:2014yos}, the KAGRA~\cite{KAGRA:2020tym}, and the LIGO-India~\cite{Unnikrishnan:2013qwa} detectors where the detection threshold on the network SNR is 8.  
Results obtained using lensed events will be compared with what can be obtained from the GW observation of 
the BNS merger GW170817 together with its host galaxy identification~\cite{LIGOScientific:2017adf}.
For GW170817, we use the $\DGW$ posterior sample from the corresponding data release~\cite{LIGOScientific:GWTC1DR}. 
For this event we cannot construct the prior on $H_0$ for GW170817 using the method which we used for lensed events; 
therefore we use Planck 2018~\cite{Planck:2018vyg} results when analyzing it. 


\section{Results}

\begin{figure}
    \centering
    \includegraphics[width=\textwidth]{../figures/presentation_t_perc_deviation_parameter_column.pdf}
    \caption{Parameter estimation results}
    \label{fig:lensing-pe-results}
\end{figure}
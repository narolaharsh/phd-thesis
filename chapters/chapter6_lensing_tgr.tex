
This chapter contains results from our recent work on constraining modified propagation theories using gravitational lensing of GWs ~\cite{Narola:2023viz}.

General relativity predicts that gravitational waves are non-dispersive (all frequency components travel at the same speed), and travel at the speed of light. Their amplitude is inversely proportional to the distance to the source. The non-dispersive nature has been a part of a suite of testing GR analyses carried out by LVK. The multimessenger detection of GW170817 helped place stringent constraints on the difference between the speed of GWs and the speed of light. Specifically, it was measured that the difference between the speed of GWs and light is in between $-3\times10^{-15}$ and $6\times10^{-16}$ times the speed of light. The identification of the EM counterpart also enabled the constraint of any modifications to the distance-amplitude relation. The relation in GR is

\begin{equation}
\text{Amplitude} \prop \frac{1}{\text{Distance to the source}}
\end{equation}

However, the constraints remain relatively poor. In this chapter, we demonstrate how the detection of strongly lensed GW signals can significantly improve upon the general relativistic theories that allow for modified propagation of GWs, specifically modified distance-amplitude relations. When discussion modified propagation theoies in the remaining of the chapter, we restrict ourselves to the ones that allow for a modified distance-amplitude realtions.   


\section{Gravitational lensing and distance measurements}
First, I briefly recap the discussion on the gravitational lensing of gravitational waves. 


\subsubsection{Electromagnetic distance}
%A strongly lensed GW source will have an improved sky localization compared to a non-lensed source, as we can observe the former multiple times with different detector orientations~\cite{Janquart:2021qov, Lo:2021nae, %Hannuksela2020LocalizingLensing}. Especially with four detectable images, we may be able to localize the source within $\mathcal O(1) $ square degrees~\cite{Hannuksela:2020xor, Wempe:2022zlk}. %Janquart2021AEvents
%Various lensing rate estimates show that such a scenario where strong lensing produces four images (a quadruplet) of GW signal is fairly likely. More concretely, about $30\%$ of the strong lensing scenarios are %expected to produce quadruplets.

When a GW source is lensed, we can expect that the electromagnetic (EM) radiation coming from its host 
galaxy is also lensed, as is widely assumed in cosmography 
studies~\cite{Chen:2017rfc, LIGOScientific:2018gmd, Gray:2019ksv, LIGOScientific:2019zcs, DES:2019ccw}. 
A joint GW+EM analysis can help locate the source's host galaxy once its location is narrowed down to a 
few square degrees using only GW data.
In this step, one reconstructs all the lenses in the region provided by the GW data to find which lens 
could best produce a GW quadruplet with 
properties similar to the ones observed; the galaxy that is undergoing lensing by this particular lens is then likely to be the host galaxy of the GW event.
This methodology was first proposed in~\cite{Hannuksela:2020xor, Wempe:2022zlk}. For the scenarios where strong lensing produces four images (a quadruplet), we can narrow down the host of the GWs source to one or a few galaxies. Various lensing rate estimates show that $\approx 30\%$ of all strong lensing scenarios will produce a quadruplet, making this a fairly likely scenario~\cite{Li:2018prc, Wierda:2021upe}.

Once the host galaxy is known, a dedicated spectroscopic or photometric follow-up can lead us to the redshift of the source.
By combining the source's redshift with a cosmological model, we can estimate the source's luminosity distance in a way 
that is unaffected by the anomalous GW propagation~\cite{Hogg:1999ad}. Let us refer to this distance as the electromagnetic distance $\DEM$ to the source. 

\subsubsection{Gravitational wave distance}
Besides the electromagnetic distance ($\DEM$), we can have another independent measurement of the source's luminosity distance from the GW data as explained in chapter~\ref{ch:gwpe}. Let us refer to this distance as the gravitational wave distance $\DGW$ to the source. The $\DGW$ distance measurement could be affected by the anomalous propagation of gravitational waves while $\DEM$ may not. Therefore, by comparing the two distances, $\DGW$ and $\DEM$, the anomaly in the gravitational wave propagation can be discovered or bounded.

\subsubsection{Strongly lensed signals: A probe complementary to the BNS signals}
Studying modified propagation theories in the context of strongly lensed and localized GW events, especially from 
binary black hole (BBH) coalescences, is attractive, because such events can help probe higher redshift regimes compared to binary neutron star (BNS) events. 
In the past, modified propagation theories have been tested using 
GW170817 \cite{Mastrogiovanni:2020mvm, Lagos:2019kds, LIGOScientific:2018dkp, Pardo:2018ipy}, 
a signal from a BNS inspiral with an identifiable EM counterpart 
\cite{LIGOScientific:2017vwq, LIGOScientific:2017ync}. However, by cosmological standards, the GW170817 signal 
travelled only a small distance before it reached the detectors, and 
in modified propagation theories, the imprint of the deviation tends to accumulate with distance. 
Other methods have been proposed that exploit the population properties of BBH coalescences observed with GWs 
\cite{Ezquiaga:2021ayr,Leyde:2022orh,MaganaHernandez:2021zyc}; since BBHs can be detected out to larger 
distances, this enables considerably improved bounds over the ones from GW170817. 
 Due to magnification, GWs from \emph{lensed} BBH events can potentially be seen out to redshifts 
 $z \sim 6$~\cite{Wierda:2021upe}, so more stringent constraints can be expected also from this methodology. 
This paper aims to quantify the gain from GW lensing for the different anomalous propagation scenarios considered.

\subsubsection{Modified propagation theories}



\section{Simulation set up}


\section{Results and conclusions}

\begin{figure}
    \centering
    \includegraphics[width=\textwidth]{../figures/tgr_wf_effects.pdf}
    \caption{The effect on the time domain GW signal in each of the modified propagation models. For each model, the GR waveform is shown in black colour and deviation from it by different amounts is shown in different colours. 
 In these examples, the GW source is assumed to be at $\sim 5$ Gpc and the masses of the source are similar to GW150914~\cite{LIGOScientific:2016aoc}. For the running 
 Planck mass model, the deviation is absolute since one has $c_M = 0$ in GR.  
 For large extra dimensions and $\Xi$-parameterization, we consider percentage deviation in the 
 parameters $D$ and $\Xi_0$, taking the fiducial values to be $D = 4$ and $\Xi = 1$, respectively. 
 For the $\Xi$-parameterization, we arbitrarily choose $n = 1$ here, though in our subsequent analyses
 it will be a free parameter.}
    \label{fig:wf-effects}
\end{figure}

\begin{figure}
    \centering
    \includegraphics[width=\textwidth]{../figures/thesis-2D-deviation-plots.pdf}
    \caption{Qualitative estimates}
    \label{fig:qual-res}
\end{figure}


\begin{figure}
    \centering
    \includegraphics[width=\textwidth]{../figures/presentation_t_perc_deviation_parameter_column.pdf}
    \caption{Parameter estimation results}
    \label{fig:lensing-pe-results}
\end{figure}
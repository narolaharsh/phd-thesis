General relativity predicts
\begin{itemize}
    
    \item Gravitational waves are non-dispersive, i.e., all frequency components travel at the same speed.
    \item They travel at the speed of light.
    \item Their amplitude is inversely proportional to the distance to the source. 

\end{itemize}

The non-dispersive nature has been a part of a suite of testing GR analyses carried out by the LVK collaboration~\cite{LIGOScientific:2021sio}. The multimessenger detection of GW170817 helped place stringent constraints on the difference between the speed of GWs and the speed of light. Specifically, it was measured that the difference between the speed of GWs and light is between $-3\times10^{-15}$ and $6\times10^{-16}$ times the speed of light~\cite{LIGOScientific:2017zic}. Modifications to the distance-amplitude relation were investigated with the help of GW170817 and its electromagnetic counterpart~\cite{LIGOScientific:2018dkp}. According to GR, the relation is
\begin{equation}
\text{Amplitude} \propto \frac{1}{\text{Distance to the source}}.
\end{equation}

However, these constraints remain relatively poor, as will become clear later. When discussing modified propagation theories, we specifically refer to the ones that predict a different distance-amplitude relation than GR. In this chapter, we demonstrate how the detection of a strongly lensed GW signal can significantly improve upon the bounds on the beyond-GR theories which allow for a modified GW propagation. This chapter contains results from our recent work on this topic~\cite{Narola:2023viz}.

\section{Gravitational lensing and distance measurements}
\iffalse
\begin{figure}
    \centering
    \includegraphics[width=\textwidth]{./figures/lensing_graphics.png}
    \caption{A simplified illustration of strong lensing. A GW signal emitted by the merger of binary black holes encounters a massive object in its path and splits into multiple copies (also known as images). The number of images produced and their (de-)magnification depend on the spatial configuration of the source and lens mass. Each image may travel a different path; therefore, it may arrive at Earth-based detectors at different times. Observing the same source multiple times in this manner drastically improves our measurements of source parameters.}
    \label{fig:strong-lensing}
\end{figure}
\fi
First, I briefly recapitulate the discussion on the gravitational lensing of gravitational waves from chapter \ref{ch:relative_binning}. When GWs encounter a massive object in their path of propagation, they may deviate from their original path. This phenomenon is known as gravitational lensing of gravitational waves. The precise effect of the lensing on the GWs depends on the relative configurations of the lens mass, the GW source, and their length scales. 

Specifically, when the length scale of the lens is much larger than the wavelength of the GW signals, we are in the so-called geometric optics regime. Such a scenario could produce multiple copies of the GW signal (known as images), and each copy may be observable by the detectors as repeated GW signals from the same source. This phenomenon is referred to as the strong lensing of a gravitational wave signal. As the detectors can observe the same source multiple times, we can combine the information from multiple images to improve our measurements of the source parameters, especially, we can significantly improve our measurements of sky-localisation. 

\subsubsection{Electromagnetic distance}

When a GW source is lensed, we can expect that the electromagnetic (EM) radiation coming from its host galaxy is also lensed, as is widely assumed in cosmography studies~\cite{Chen:2017rfc, LIGOScientific:2018gmd, Gray:2019ksv, LIGOScientific:2019zcs, DES:2019ccw}. A joint GW+EM analysis can help locate the source's host galaxy once its location is narrowed down to a few square degrees using only GW data. In this step, one reconstructs all the lenses in the region provided by the GW data to find which lens could best produce an image with properties similar to the ones observed; the galaxy that is undergoing lensing by this particular lens is then likely to be the host galaxy of the GW event. This methodology was first proposed in~\cite{Hannuksela:2020xor, Wempe:2022zlk}. For the scenarios where strong lensing produces four images (a quadruplet), we can narrow down the host of the GW source to one or a few galaxies. Various lensing rate estimates find that we may observe $\approx 1$ strongly lensed event per year with a network of the LIGOs, Virgo, and KAGRA detector operting with 100\% duty cycle~\cite{Li:2018prc, Wierda:2021upe, Oguri:2018muv}. The rates can be subject to uncertainties introduced by the merger-rates, detection thresholds, and detector duty cycles. The rate estimates also find that $\sim 30\%$ of all strong lensing scenarios will produce a quadruplet, making this a fairly likely scenario.

Once the host galaxy is known, a dedicated spectroscopic or photometric follow-up can yield a precise estimate of the redshift of the source. By combining the source's redshift with a cosmological model, we can estimate the source's luminosity distance~\cite{Hogg:1999ad}. Let us refer to this distance as the electromagnetic distance $\DEM$ to the source. 

\subsubsection{Gravitational wave distance}
Besides the electromagnetic distance ($\DEM$), we can have another independent measurement of the source's luminosity distance from the GW data as explained in chapter~\ref{ch:gwpe}. Let us refer to this distance as the gravitational wave distance $\DGW$ to the source. The $\DGW$ distance measurement may carry an imprint of the anomalous propagation of gravitational GWs while $\DEM$ may not. Therefore, by comparing the two distances, $\DGW$ and $\DEM$, the anomaly in the gravitational wave propagation can be discovered or bounded.

\section{Modified propagation theories}
\label{sec:mpt}
When discussing modified propagation theories, we refer to those theories that predict a different GW distance-amplitude relation from GR. In the specific modified gravity models we consider -- large extra dimensions, $\Xi$-parametrisation, and varying Planck mass -- there is a non-trivial relationship between $\DGW$ and $\DEM$. The relationships depend on the parameter(s) related to the deviation from GR and on the cosmological parameters. Let us briefly discuss what these relationships look like for our three models.

\begin{figure}
    \centering
    \includegraphics[width=\textwidth]{../figures/tgr_wf_effects.pdf}
    \caption[The effect on the time domain GW signal in each of the modified propagation models]{The effect on the time domain GW signal in each of the modified propagation models. For each model, the GR waveform is shown in black colour and deviation from it, by different amounts, is shown in different colours. In these examples, the GW source is assumed to be at $\sim 5$ Gpc and the masses of the source are similar to GW150914~\cite{LIGOScientific:2016aoc}. For the running the Planck mass model, the deviation is absolute since one has $c_M = 0$ in GR. For large extra dimensions and $\Xi$-parametrisation, we consider the percentage deviation in the parameters $D$ and $\Xi_0$, taking the fiducial values to be $D = 4$ and $\Xi = 1$, respectively. For the $\Xi$-parametrisation, we arbitrarily choose $n = 1$ and keep it fixed in making the plot above.}
    \label{fig:wf-effects}
\end{figure}

\subsubsection{Large extra dimensions}
In theories of gravity with large extra dimensions, there is the possibility of some energy of the GWs leaking into them \cite{Dvali:2000hr, Calcagni:2019kzo, LIGOScientific:2018dkp}, while EM radiation is confined to the usual three spatial dimensions. This would make the detected signal appear weaker, leading to larger measured values for $\DGW$ than would otherwise be the case. For definiteness, we will work with the following simple phenomenological ansatz for the relation between $\DGW$ and $\DEM$, based on conservation of integrated flux \cite{Deffayet:2007kf}:
\begin{equation}
    \label{eqn:D-model}
 \DGW = \left(\frac{\DEM(z_s, H_0)}{\mathrm{Mpc}}\right)^{\frac{D-2}{2}},
\end{equation}
where $D$ is the number of spacetime dimensions and $z_s$ is the source redshift. We will allow $D$ to be a real number, with the GR value $D = 4$ as a fiducial value. An illustration of the effect of extra dimensions on a GW signal is given in the left plot of Figure~\ref{fig:wf-effects}.

\subsubsection{The $\Xi$-parametrisation}
Another parametrisation was proposed in~\cite{Belgacem:2018lbp}, where the link between $\DGW$ and $\DEM$ is expressed as
\begin{equation}
    \label{eqn:xi-model}
 \DGW = \DEM(z_s, H_0)\left[\Xi_0+\frac{1-\Xi_0}{(1+z_s)^n}\right].
\end{equation}
The free parameters of the model are $(\Xi_0, n)$. This parameterisation is phenomenological, but as shown in \cite{LISACosmologyWorkingGroup:2019mwx}, it can be related to a large class of modified gravity theories, including Horndeski~\cite{Horndeski:1974wa} theories, Degenerate Higher Order Scalar-Tensor theories (DHOST)~\cite{Langlois:2018dxi}, and theories with nonlocally modified gravity~\cite{Maggiore:2013mea, Maggiore:2014sia, Belgacem:2017cqo}. When $z \ll 1$, $\DGW \simeq \DEM$. Therefore, similar to the extra dimension theories, we expect to observe a departure from GR only at large distances ($z \gtrsim 1$). For GR, $\Xi_0 = 1$ and $n$ is degenerate. In Figure~\ref{fig:wf-effects}, middle plot, one can see an illustration of the effect of this modification on a GW signal.

\subsubsection{Time-varying Planck mass}
A time-varying Planck mass is another possible cause for modified GW propagation. Following~\cite{Lagos:2019kds}, the relation between $\DGW$ and $\DEM$ can be expressed as
\begin{equation}
    \label{eqn:cm-model}
 \DGW(z) = \DEM(z_s, H_0) \times \exp{\left(\frac{c_M}{2\Omega_{\Lambda}}\ln{\frac{1+z_s}{(\Omega_{m}(1+z_s)^3 + \Omega_{\Lambda})^{1/3}}}\right)} ,
\end{equation}
where $c_M$ is a constant that relates the rate of change of the Planck mass to the fractional dark energy density in the Universe. For GR, $c_M = 0$. The right plot of Figure~\ref{fig:wf-effects} illustrates the effect on a GW signal.

\begin{table}
    \centering
    \begin{tabular}{ c | c | c }
        \hline\hline
 Theory                 & Parameter & Priors  \\
        \hline
 Large extra dimension        & $D$         & Uniform(3, 5)          \\
        \hline
        $\Xi$-parametrisation & $\Xi_0$   & Log Uniform(0.01, 100) \\
                               & $n$       & Uniform(0, 10)         \\
        \hline
 Running Planck mass    & $c_M$     & Uniform(-150, 150)     \\
        \hline
        
    \end{tabular}
    \caption{Deviation parameter(s) for each theory and the corresponding prior probability distributions used in our analyses. The ranges of the prior are centered around the general relativistic value of the parameter and are chosen following the previous studies on modified propagation \cite{Mastrogiovanni:2020mvm, Lagos:2019kds, Pardo:2018ipy}.}
    \label{tab:priors}
\end{table}

\section{Strongly lensed signals: A probe complementary to the binary neutron star signals}

\begin{figure}[h]
    \centering
    \includegraphics[width=\textwidth]{../figures/thesis-2D-deviation-plots.pdf}
    \caption[Imprint on the observable quantity ($\Delta$) as a function of source redshift (x-axis) and scale of deviation (y-axis)]{Imprint on the observable quantity ($\Delta$) as a function of source redshift (x-axis) and scale of deviation (y-axis). For the large extra dimension model and $\Xi$-parametrisation, the y-axis shows percentage deviation with respect to the general relativistic value. The 0 on the y-axis corresponds to the GR value. For time-varying Planck mass, the y-axis shows the absolute deviation since the general relativistic value is 0 for the model parameter. The blue vertical line marks the redshift measurement of the GW170817 signal. For the $\Xi$-parametrisation and varying Planck mass, at the redshift of GW170817, the $\Delta$ is quite small even for large amounts of deviations from GR (grey region), suggesting that these models may be poorly constrained. Access to the high-redshift regime is likely to lead to better constraints on these deviations. On the other hand, the effect of extra dimensions is less sensitive to redshift, and measurements of $D$ are already quite stringent at the redshift of GW170817 and
 are not expected to improve as much as for the other two cases when going to a higher redshift.}
    \label{fig:qual-res}
\end{figure}

Studying modified propagation theories in the context of strongly lensed and localised BBH signals is attractive because such events can help probe higher redshift regimes compared to BNS signals. In the past, modified propagation theories have been tested using GW signals from a binary neutron star merger (GW170817) with an identifiable EM counterpart~\cite{Mastrogiovanni:2020mvm, Lagos:2019kds, LIGOScientific:2018dkp, Pardo:2018ipy}. However, by cosmological standards, the GW170817 signal travelled only a small distance before it reached the detectors, carrying only a negligible imprint from any deviation from GR that may be present. Therefore, the constraints obtained from the neutron star merger remain poor. In the context of modified propagation theories, the imprint of the deviation tends to accumulate with the distance travelled by the signals. %Other methods have been proposed that exploit the population properties of BBH coalescences observed with GWs \cite{Ezquiaga:2021ayr, Leyde:2022orh, MaganaHernandez:2021zyc}; since BBHs can be detected out to larger distances, this enables considerably improved bounds over the ones from GW170817. 
Due to magnification, GWs from \emph{lensed} BBH events can potentially be seen out to redshifts $z \sim 6$~\cite{Wierda:2021upe}; this feature leads to further increases in the imprints of any anomaly in the propagation, consequently improving our bounds on the anomaly. Therefore, the ability of a strongly lensed GW signal to probe the high-redshift regime makes it an invaluable probe to test modified propagation theories. We also note that modified propagation theories which predict a dephasing of GW signals compared to GR are routienly performed using BBH signal, however these are not the type of theories we are considering. We restrict our selves to the that predict a modified distance-amplitude relationship. 


\section{Measurements of model parameters}
We want to estimate the posterior probability distribution on deviation parameters given the GW data and EM data, i.e.,
\begin{equation}
p(\betaMGR, H_0|\vec{d}_{\mathrm{GW}}, \vec{d}_{\mathrm{EM}}),
\end{equation}
where we use $\betaMGR$ to denote the model parameters in all generality (see Table~\ref{tab:priors}), the notations $\vec{d}_{GW}$ and $\vec{d}_{EM}$ represent the GW and EM data of a quadruply lensed event whose host galaxy has been identified. The parameter $H_0$ denotes the Hubble constant. 

Using Bayes' theorem, we can write 
\begin{equation}
\label{eq:pe1}
p(\betaMGR, H_0 | \vec{d}_\mathrm{GW},  \vec{d}_{\mathrm{EM}}) = \frac{\pi(\betaMGR, H_0) ~ p(\vec{d}_{\mathrm{GW}}, \vec{d}_{\mathrm{EM}} | \betaMGR, H_0)}{Z},
\end{equation}
where $\pi(\betaMGR, H_0)$ is the prior probability distribution on $\betaMGR$ and $H_0$; the factor $p(\vec{d}_{\mathrm{GW}}, \vec{d}_{\mathrm{EM}} | \betaMGR, H_0)$ represents the likelihood function; and $Z$ the evidence, whose value follows from the requirement that the posterior probability distribution is normalised. The prior distributions for $\betaMGR$ are specified in Table~\ref{tab:priors}. We have chosen uninformative priors on each of the $\betaMGR$ parameters to not favour any specific value. 

\subsubsection{Prior choice $H_0$}
For $H_0$, we could in principle choose a relatively narrow prior range based on the Planck \cite{Planck:2006aa}, SHoES~\cite{Riess:2021jrx}, or other existing measurements~\cite{H0LiCOW:2019pvv}. Instead, we make a more conservative choice, i.e., a wider prior. Specifically, we use a measurement of $H_0$ that is obtained from the differences in time of arrival of the lensed images together with lens reconstruction through electromagnetic means. We use the measurement obtained in such a way as the prior on $H_0$. This prior is relatively wider than any of the previous measurements; therefore, we argue that our measurements on $\betaMGR$ are on the conservative side.

The Hubble constant measurement can be determined by observing the delays in the arrival time of the lensed images~\cite{Liao:2017ioi}. The theory of strong lensing provides the relationship between the time delay $(\Delta t)$ between two images and the difference in the Fermat potential $(\Delta \Phi)$,
\begin{equation}
    \Delta t = D_{\Delta t}(1+z_l){\Delta \Phi},
\end{equation}
where $D_{\Delta t}$ is called the time delay distance and $z_l$ is the redshift of the lens. We can simultaneously obtain the $D_{\Delta t}$ and $\Delta \Phi$ by solving the equation above for a quadruple lensed system~\cite{Hannuksela:2020xor}. The remaining quantity, time delay distance $D_{\Delta t}$, is related to the Hubble constant in the followig way:
\begin{equation}
\label{eqn:time-delay-dist}
D_{\Delta t}(z_l, z_s, H_0) = \frac{\int_0^{z_s} dz'/E(z')}{\int_{z_l}^{z_s} dz'/E(z')}\DEM(z_s, H_0),
\end{equation} 
where $z_l$ and $z_s$ are respectively the lens and the source redshift, and 
$E(z) \equiv \sqrt{\Omega_m (1+z)^3 + \Omega_\Lambda}$. Here, $\Omega_m$ is the matter density parameter and $\Omega_\Lambda$ is the dark energy density parameter. If $D_{\Delta t}$ 
is measured, we can estimate $\DEM$ since we assume that $z_l$ and $z_s$ are known from 
the EM follow-up observations. In this work, for definiteness, we assume a flat Friedmann-Lemaître-Robertson-Walker universe, in which case one has
\begin{equation}
    \label{eqn:dl-em}
 \DEM =\frac{(1+z_s)}{H_0}\int_0^{z_s} \frac{dz'}{E(z')}.
\end{equation}

Using the $\DEM$ measurement, $H_0$ can be estimated through Eq.~\eqref{eqn:dl-em}. $D_{\Delta t}$ can be measured after reconstructing the Fermat potential; However, owing to the computational complexity and cost, we skip the reconstruction step. We assume that the observed value of $D_{\Delta t}$ follows a Gaussian distribution with a 10\% standard deviation~\cite{Hannuksela:2020xor}. To allow for an offset in the observation, we pick the mean of this distribution from another Gaussian distribution with a 10\% standard deviation, which is centred at the true value. We translate the measurement of $D_{\Delta t}$ into the measurement of $\DEM$ using Eq.~\eqref{eqn:time-delay-dist}. Using the value of $\DEM$ together with Eq.~\eqref{eqn:dl-em}, we construct the prior for $H_0$. Throughout the analysis, we keep the other cosmological parameters, $\Omega_m$ and $\Omega_{\Lambda}$, fixed assuming that they are known to a few per cent accuracy~\cite{Lagos:2019kds}. 

\subsubsection{Computing the joint likelihood}
To calculate the likelihood $p(\vec{d}_{\mathrm{GW}}, \vec{d}_{\mathrm{EM}} | \betaMGR, H_0)$, we first express it as
\begin{equation}
\label{eq:likeli}
p(\vec{d}_{\mathrm{GW}}, \vec{d}_{\mathrm{EM}} | \betaMGR, H_0) = \int d\vec{\theta}~dz_s~p(\vec{d}_{\mathrm{GW}} | \vec{\theta}) ~ p(\vec{d}_{\mathrm{EM}} | z_s) ~ \pi(\vec{\theta} | z_s, \betaMGR, H_0) ~ \pi(z_s| \betaMGR, H_0),
\end{equation}
where $\vec{\theta}$ denotes the GW source parameters, the factors $p(\vec{d}_{\mathrm{GW}} | \vec{\theta})$ and  $p(\vec{d}_{\mathrm{EM}} | z_s)$ are the likelihoods of the GW and EM data, respectively, and $z_s$ is the source redshift. The factors $\pi(\vec{\theta} | z_s, \betaMGR, H_0)$ and $\pi(z_s | \betaMGR, H_0)$ are the priors on the GW source parameters and the redshift. 

Since we assume that the host galaxy has been localised, the true source redshift $z_s$ is known. Recent studies have shown that the spectroscopic redshift of the source can be measured to a sub-per cent accuracy \cite{Zhou:2021fmt}. Therefore, we neglect the error on the measurement of $z_s$ so the factor $p(\vec{d}_{\mathrm{EM}} | z_s)$ becomes a Dirac delta function centered on $z_s$, reducing Eq.~\eqref{eq:likeli} to 
\begin{equation}
p(\vec{d}_{\mathrm{GW}}, \vec{d}_{\mathrm{EM}} | \betaMGR, H_0) = \int d\vec{\theta} ~ p(\vec{d}_{\mathrm{GW}} | \vec{\theta}) ~ p(\vec{\theta} | z_s, \betaMGR, H_0)~ p(z_s| \betaMGR, H_0) .
\end{equation}

To estimate the GW likelihood $p(\vec{d}_{\mathrm{GW}}|\vec{\theta})$, we have to perform joint parameter estimation on strongly lensed signals as described in chapter~\ref{ch:relative_binning}. In principle, we could use the relative binning framework to do the analysis. However, it was developed after this analysis concluded. For the purpose of this work, we use the method which was specifically developed to rapidly analyse lensed images, called \textsc{GOLUM}~\cite{Janquart:2021qov}. \textsc{GOLUM} method uses a subset of posterior samples of the first image as prior for the subsequent image analyses. One can also integrate the relative binning framework into \textsc{GOLUM} to gain additional speed-up.
%We repeat this step for each quadruplet in our catalogue. 

Once the GW likelihood is estimated, we perform the integration over all parameters represented by $\vec{\theta}$ except the luminosity distance $\DGW$ to obtain
\begin{align}
\label{eqn:step-likeli}
p(\vec{d}_{\mathrm{GW}}, \vec{d}_{\mathrm{EM}} | \betaMGR, H_0) = \int d\DGW~p(\vec{d}_{\mathrm{GW}} | \DGW)~ p(\DGW | z_s, \betaMGR, H_0) ~ p(z_s| \betaMGR, H_0).
\end{align}

The prior $p(\DGW | z_s, \betaMGR, H_0)$ reduces to a Dirac delta function as we exactly know $\DGW$ given the values of $z_s$, $\betaMGR$, $H_0$ and the modified gravity model (Eqs.~\eqref{eqn:D-model}, \eqref{eqn:xi-model} and \eqref{eqn:cm-model}). Therefore, integrating with respect to $\DGW$ leads to
\begin{equation}
    \label{eq:redu-likeli}
 p(\vec{d}_{\mathrm{GW}}, \vec{d}_{\mathrm{EM}} | \betaMGR, H_0)  = p( \vec{d}_{\mathrm{GW}} | \DGW)~p(z_s|\betaMGR, H_0).
\end{equation}
Substituting Eq.~\eqref{eq:redu-likeli} into Eq.~\eqref{eq:pe1}, we can obtain the posterior distributions for $\betaMGR$ and $H_0$. This completes the estimation of the joint likelihood. 
\iffalse
The above derivation is performed for one of the images of the quadruplet. 
To combine the information from multiple images and obtain the joint posterior on $\betaMGR$, 
we need to express the distance likelihood $p( \vec{d}_{\mathrm{GW}} | \DGW)$ in 
Eq.~(\ref{eq:redu-likeli}) as a joint likelihood:
\begin{equation}
\label{eq:joint-distance-likelihood}
p(\vec{d}_{\mathrm{GW}} | \DGW ) = \prod_{i=0}^{3} p(\vec{d}_{\mathrm{GW}, i} | \DGW),
\end{equation}
where $\vec{d}_{\mathrm{GW}, i}$ refers to the GW data from image $i$. 
\fi

While simulating the lensed BBH signals, we sample the component masses from the \textsc{PowerLaw+Peak} distribution presented in~\cite{KAGRA:2021duu}. We use the IMRPhenomXPHM waveform model~\cite{Pratten:2020ceb}, with black hole spin magnitudes distributed uniformly between $0$ and $1$, and spin directions uniformly on the sphere. The distribution of the redshifts of the galaxy lenses, modelled as singular power law isothermal ellipsoids with external shear, follows the SDSS galaxy catalogue~\cite{Collett:2015roa}. The fiducial values of $\betaMGR$ are equal to their GR values. The fiducial value of the Hubble constant is $H_0 = 67.4$ km $\text{s}^{-1}$ $\text{Mpc}^{-1}$, and $\Omega_m = 0.315$. For the purpose of the simulation, we consider a network that consists of two LIGOs, Virgo, KAGRA, and the LIGO-India detectors, where the detection threshold on the network SNR is 8. Measurements of the deviation parameters made using the lensed GW signals will be compared to those made using the BNS merger GW170817, together with its host galaxy identification. For GW170817, we use the $\DGW$ posterior sample from the corresponding data release~\cite{LIGOScientific:GWTC1DR}. For this event, we cannot construct the prior on $H_0$ for GW170817 using the method which we used for lensed events; therefore, we use Planck 2018 results when analysing it~\cite{Planck:2018vyg}. 
% https://arxiv.org/pdf/1901.03321 ... few percent

\section{Results and conclusions}


\begin{figure}
    \centering
    \includegraphics[width=\textwidth]{../figures/presentation_t_perc_deviation_parameter_column.pdf}
    \caption[Bounds on the model parameters for the three models]{Bounds on the model parameters for the three models we considered. Each cross represents the bounds of the deviation parameter (y-axis) obtained by a quadruply lensed signal. The x-axis shows the corresponding redshift of the source, and the colourbar shows the quadrature sum of the SNR of the four images. The triangle shows the corresponding measurements obtained using the GW170817 signal and its electromagnetic counterpart. The y-axis shows relative error bars for the large extra dimensions ($\Delta D/D$) and $\Xi$-parametrisation $(\Delta\Xi/\Xi$). For time-varying Planck mass, we show the absolute error bars since the fiducial value is equal to zero. The error bars represent a 90\% confidence interval. A strong lensing observation could improve the bounds on the large extra-dimensional model by a factor of $\sim5$, for $\Xi$-parametrisation by a factor of $\sim 10$, and for time-varying Planck by a factor of $\sim 10^2$; consistent with inference drawn from Figure~\ref{fig:qual-res}.}
    \label{fig:comb-ngr-params}
\end{figure}

Figure~\ref{fig:comb-ngr-params} shows the bounds on the model parameters using strongly lensed BBH signals. We analyse a total of 55 signals. Note that we do not expect to see this many quadruply lensed events until the third-generation detector era, nor do we combine information from multiple simulated lensed events. Our aim here is to explore the diversity of scenarios one might encounter for a single quadruply lensed GW.

Each cross in Figure~\ref{fig:comb-ngr-params} corresponds to a simulated strongly lensed GW signal with four images. The x-axis shows the redshift of the source. The true values of model parameters are set equal to their GR values. 
The y-axis shows relative error bars for the large extra dimensions ($\Delta D/D$) and $\Xi$-parametrisation $(\Delta\Xi/\Xi$). For time-varying Planck mass, we show the absolute error bar since the fiducial value is equal to zero. The error bars represent a 90\% confidence interval. For the $\Xi$-parametrisation model, the parameter $n$ is unconstrained when $\Xi_0$ equals its fiducial value of 1; we do not show results for it here, though it was treated as a free parameter in our measurements. Finally, the colourbar shows the combined SNR from the four images, i.e., the quadrature sum of the SNRs of the individual images. Also included are the results from GW170817.

The results are in qualitative agreement with plots presented in Figure~\ref{fig:qual-res}. In particular, for $\Xi_0$ and $c_M$, the advantage of being able to access higher redshifts is evident, with bounds improving over those of GW170817 by factors of up to $\mathcal{O}(10)$ and $\mathcal{O}(10^2)$, respectively. By contrast, the bounds on $D$ improve by up to a factor of $\sim 5$. The differences in improvement can be explained by the qualitative predictions of Figure~\ref{fig:qual-res} where a given amount of deviation has a small imprint on the observable quantity ($\Delta$) for $\Xi$-parameterisation (center) and time-varying Planck mass (right) but a large one for the large extra dimension model (left). 

We note that for the strongly lensed events in our catalogue, the combined SNR from the four images tends to be higher than that of GW170817, which can also improve the measurement accuracy on $\DGW$ and subsequently $\betaMGR$. However, Figure~\ref{fig:comb-ngr-params} shows that, using lensed events with SNR similar to GW170817 (which was $\simeq 32.4$ \cite{LIGOScientific:2017vwq}), we can measure the deviation parameters more accurately compared to the latter, as the lensed events travel a longer distance. An increment in the distance made accessible by strong lensing is indeed the dominating factor in the improvement of measurement accuracies. 

We also note that we did not directly perform lens reconstruction, but instead assumed Gaussian probability distributions for image magnification measurements used in the reconstruction of $\DGW$ and reconstructed electromagnetic luminosity distances, with widths informed by current astrophysical expectations \cite{Hannuksela:2020xor, Wempe:2022zlk}. We aim to treat this aspect in more depth in a future study. Similarly, the relation between $\DGW$ and $\DEM$ involves cosmological parameters. In this work, we only let $H_0$ be a free parameter, but the effect of uncertainties in the other parameters is also worth investigating. On the other hand, in this study, we used as a prior on $H_0$ the posterior density distribution obtained from time delay measurements and lens reconstruction, which is typically considerably wider than the ranges for $H_0$ obtained from either Planck or SHoES. Because of the degeneracy between $H_0$ and the deviation parameters, bounds on the latter are to a large extent set by the prior range of $H_0$ \cite{Lagos:2019kds}, which pushes our constraints on alternative theories towards the conservative side. We find that when analysing the lensed events with a prior from Planck 2018, we obtain bounds that are a factor of $\sim 2$ tighter. 
    
Let us also make a comparison with existing bounds on the model parameters. For the $\Xi$-parametrisation, the bounds we obtain are consistent with the results of Finke \textit{et al.}~\cite{Finke:2021znb}. In Mastrogiovanni \textit{et al.}~\cite{Mastrogiovanni:2020mvm}, bounds were obtained for the same three models considered here, by combining information from GW170817, its EM counterpart, and the information from the BBH signal GW190521, assuming that a particular EM flare observed by the Zwicky Transient Factory was associated with the GW190521 merger~\cite{ZTF-BBH}. Since GW190521 originated at a redshift of $\simeq 0.8$ \cite{LIGOScientific:2020iuh}, adding this event brings the bounds on deviation parameters closer to what we find for lensed events. For example, they report $\delta\Xi_0/\Xi_0 \lesssim 3 - 10$, comparable to the bounds of our simulations. However, it should be noted that the association of GW190521 with the EM flare is by no means conclusive; see e.g.~\cite{Ashton:2020kyr}. Studies based on the cosmic microwave background and large structure formation can lead to bounds on $c_M$ that are similar to the ones for lensed events; see e.g.~\cite{Noller:2018wyv}. Finally, methods have been developed that exploit the observed population properties of binary black hole coalescences using gravitational wave data only, in terms of e.g., redshift and mass distributions~\cite{Ezquiaga:2021ayr, Leyde:2022orh, MaganaHernandez:2021zyc}. 

In summary, the previous gravitational wave-based measurements on anomalous propagation models have relied on GW170817 with its EM counterpart. Until third-generation detectors such as the Einstein Telescope and Cosmic Explorer are operational, gravitational waves from binary neutron star mergers will only be seen to a relatively small redshift. Moreover, a definitive identification of transient electromagnetic counterparts to a GW signal may remain elusive. What we have demonstrated here is that a single fortuitous discovery of a lensed signal in conjunction with a dedicated electromagnetic follow-up may help us probe the high-redshift regime, enabling significantly stronger bounds on models of anomalous gravitational wave propagation.




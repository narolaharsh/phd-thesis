\documentclass[a4paper,11pt, twoside]{book}
%change bto twoside for printing
%\documentclass[a5paper,10pt]{book}


%\usepackage[left=1.5cm,right=1.5cm, bottom=1.5cm]{geometry}
\usepackage[utf8]{inputenc}  
\usepackage[T1]{fontenc}
\usepackage{fancyhdr}
\usepackage{import}
\usepackage{makeidx}
\usepackage[table]{xcolor}
\usepackage{amsmath,amssymb}
\usepackage{mathtools, amssymb}
\usepackage{xspace}
\usepackage{xstring}
%\usepackage{titlesec} 
\usepackage{fix-cm}
%\usepackage[T1]{fontenc}
\usepackage{hyperref}
\usepackage{moresize}
\usepackage{anysize}
\usepackage{caption}
\usepackage[raggedright]{titlesec}
\usepackage{multicol}
\usepackage{makecell}
\usepackage{rotating}
\usepackage{pgfornament}
\usepackage{cite}
\usepackage[normalem]{ulem}
\usepackage{enumitem}
\usepackage{titling}
\usepackage{chngcntr}

\usepackage{afterpage}

\newcommand\blankpage{%
	\null
	\thispagestyle{empty}%
	\addtocounter{page}{-1}%
	\newpage}

\renewcommand{\baselinestretch}{1.2} 
\captionsetup{font=small}

\makeindex

%\titleformat{\chapter}[hang]{\bfseries\Large}{\filright\MakeUppercase{\chaptertitlename} \Huge\thechapter}{1ex}{\vspace{1ex} \filleft}[\vspace{1ex}\titlerule]
\definecolor{gray75}{gray}{0.7}
\newcommand{\hsp}{\hspace{0pt}}
\titleformat{\chapter}[display]
{\bfseries\huge}
%{\fontseries{b}\fontsize{100}{130}\selectfont \textcolor{gray75}\thechapter\hsp}
{\itshape \filright \textcolor{gray}{\chaptertitlename} {\fontsize{60pt}{70pt}\selectfont\textcolor{gray75}\thechapter}}
{2ex}
%{\titlerule\vspace{1.5ex}\centering}
{\hrulefill\, \protect{\pgfornament[width=0.08\textwidth]{72}} \protect{\pgfornament[width=0.08\textwidth]{73}} \,\hrulefill \vspace{1.5ex} \newline \centering}
%{\hrulefill\quad \protect{\pgfornament[width=0.08\textwidth]{43}}\protect{\pgfornament[width=0.08\textwidth, symmetry=v]{43}} \quad\hrulefill \vspace{1.5ex} \newline \centering}
%{\textcolor{gray}{\hrulefill\; \protect{\pgfornament[width=3cm]{85}}\; \hrulefill\nobreak} \vspace{1.5ex} \newline \centering} %87
[\vspace{1.5ex}{\titlerule}] %{\titlerule}]

\titleformat{\section}
{\normalfont\Large\bfseries}{\thesection}{1em}{}[{\color{gray}\titlerule[0.02pt]}]

\newcommand{\comment}[1]{}
\newcommand{\red}[1]{{\color{red}{[#1]}}}
\newcommand{\lambdabar}{{\mkern0.75mu\mathchar '26\mkern -9.75mu\lambda}}
\mathchardef\mhyphen="2D
\def\bin{{\rm b}}
\newcommand\abs[1]{\ensuremath{\lvert#1\rvert}}


\usepackage[left=2cm,right=2cm, bottom=2.5cm, headheight=1cm]{geometry}

\def\seob{\texttt{SEOBNRv4PHM}\xspace}
\def\nrsur{\texttt{NRSur7dq4}\xspace}
\def\tphm{\texttt{IMRPhenomTPHM}\xspace}
\def\xphm{\texttt{IMRPhenomXPHM}\xspace}

\def \qupm {\textsc{qu-pm}}
\def \nopm {\textsc{no-pm}}
\def \free {\textsc{free-pm}}
\def \stwo {$\mathrm{Source2}_{\mathrm{[\textsc{qu-pm}]}}$}
\def \sth {$\mathrm{Source3}_{\mathrm{[\textsc{qu-pm}]}}$}
\def \sone {$\mathrm{Source1}_{\mathrm{[\textsc{qu-pm}]}}$}

%% events full names
\DeclareRobustCommand{\N}[1]{\IfEqCase{#1}{{GW150914}{GW150914\xspace}{GW190620}{GW190620\_030421\xspace}{GW191222}{GW191222\_033537\xspace}{GW190519}{GW190519\_153544\xspace}{GW190521a}{GW190521\xspace}{GW190521b}{GW190521\_074359\xspace}{GW190620}{GW190620\_030421\xspace}{GW190630}{GW190630\_185205\xspace}{GW190910}{GW190910\_112807\xspace}{GW191109}{GW191109\_010717\xspace}{GW191222}{GW191222\_033537\xspace}{GW200112}{GW200112\_155838\xspace}{GW200129}{GW200129\_065458\xspace}{GW200224}{GW200224\_222234\xspace}{GW200311}{GW200311\_115853\xspace}}}

\newcommand{\subtitle}[1]{%
	\posttitle{%
		\par \vspace{1cm}
\large#1
	\vskip0.5em}%
}


\begin{document}

\pagenumbering{roman}

\newcommand{\fl}{\hrulefill\,\protect{\pgfornament[width=0.08\textwidth]{72}} \, \protect{\pgfornament[width=0.08\textwidth]{73}}\,\hrulefill}

%\fl 

%\title{\fl \vspace{1.5ex} \newline \sc\Huge\bfseries Fundamental physics from gravitational waves \\ \vspace{1cm} \sc\Large\bfseries Testing general relativity and studying neutron stars with current and future detectors \centering \newline \vspace{1.5ex} \hrulefill}
%\subtitle{\sc\Large\bfseries Testing general relativity and studying neutron stars with current and future detectors \newline \vspace{1.5ex} \hrulefill}

\title{\sc\Huge\bfseries Fundamental physics with gravitational waves \\ \vspace{1cm} \sc\Large\bfseries Testing general relativity and studying neutron stars with current and future detectors}

\author{\large\bfseries Anna Puecher}
\date{}

%\maketitle



\begin{titlepage}
	\begin{center}
		\vspace*{5cm}
		\fl
		\newline
		\vspace{0.5cm}
		
		\Huge\textsc{\textbf{Fundamental physics with gravitational waves}}
		
		\vspace{0.5cm}
		\Large\textsc{\textbf{Testing general relativity and studying neutron stars with current and future detectors}}
		
		\vspace{1cm}
		\hrule
		\vspace{1.5cm}
		
		\large\textbf{Anna Puecher}
		

		
	\end{center}
\end{titlepage}

\thispagestyle{empty}
\vspace*{9cm}

%\vfill

\noindent ISBN: 978-90-393-7605-8 \\
\noindent \textsc{doi}: https://doi.org/10.33540/2119 \\
\noindent Printed by: Gildeprint -- www.gildeprint.nl\\
\noindent Copyright: \copyright \, 2023 Anna Puecher

\vspace{2cm}

\begin{figure}[htp]
	
	\centering
	\includegraphics[width=.35\textwidth]{figures/nikhef_logo.png}\hfill
	\includegraphics[width=.35\textwidth]{figures/UU_logo.png}\hfill
	\includegraphics[width=.2\textwidth]{figures/NWO_logo.png}
	
	
\end{figure}

\vspace{1cm}
\noindent This work originates as part of the research program of the Foundation for Fundamental
Research on Matter (FOM), and falls as of April 1, 2017 under the responsibility of the
Foundation for Nederlandse Wetenschappelijk Onderzoek Instituten (NWO-I), which is
part of the Dutch Research Council (NWO).

\begin{titlepage}
	\begin{center}
		\vspace*{1cm}
				
		\LARGE\textbf{Fundamental physics with gravitational waves}
		
		\vspace{0.5cm}
		\Large Testing general relativity and studying neutron stars \\ with current and future detectors
	
		\vspace{1.5cm}
		
		\LARGE\textbf{Fundamentele fysica met zwaartekrachtsgolven}
		
		\vspace{0.5cm}
		\Large Het testen van de algemene relativiteitstheorie en het bestuderen van neutronensterren met huidige en toekomstige detectoren \\ 
		\vspace{0.5cm} (met een samenvatting in het Nederlands)
		
		\vspace{3.5cm}
		
		\normalsize
		Proefschrift \\
		ter verkrijging van de graad van doctor aan de Universiteit Utrecht op gezag van de rector
		magnificus, prof.dr. H.R.B.M. Kummeling, ingevolge het besluit van het college voor promoties in
		het openbaar te verdedigen op woensdag 7 februari 2024 des middags te 4.15 uur
		
		\vspace{1cm}
		door \\
		\vspace{1cm}
		\textbf{Anna Puecher} \\
		geboren op 28 november 1994 \\
		te Trento, Itali\"{e}
		
		
	\end{center}
\end{titlepage}

\thispagestyle{empty}
\noindent Promotor: \\
\indent \indent Prof. dr. C.F.F. van den Broeck \\

\noindent Co-promotor: \\
\indent \indent Dr. S.E. Caudill \\

\noindent Assessment Committee: \\
\indent \indent  Prof. dr. J.F.J. van den Brand \\
\indent \indent Dr. N.E. Chisari \\
\indent \indent Prof. dr. S. Hild \\
\indent \indent Prof. dr. R.J.M. Snellings \\
\indent \indent Prof. dr. N. Stergioulas

\newpage

\thispagestyle{empty}
\vspace*{5cm}
\begin{flushright}

\textit{Lock up your libraries if you like; \\
but there is no gate, no lock, no bolt that \\ you can set upon the
freedom of my mind.}\\
\vspace{1ex}
---Virginia Woolf, A Room of One's Own.
\end{flushright}

\newpage

\thispagestyle{empty}

\pagestyle{fancy}

\fancyhf{}
\fancyfoot[C]{\thepage}
\fancyhead[RO]{\slshape\nouppercase{\rightmark}}
\fancyhead[LE]{\slshape\nouppercase{\leftmark}}

\tableofcontents
\listoffigures
\listoftables

\mainmatter

\pagenumbering{arabic}

\chapter*{Introduction}
\fancyhf{}
\fancyfoot[C]{\thepage}
\fancyhead[RO,LE]{\slshape\nouppercase{Introduction}}

\renewcommand\thefigure{I.{\arabic{figure}}}

\addcontentsline{toc}{chapter}{Introduction}
\import{chapters/}{introduction.tex}

\chapter{Gravitational waves in general relativity: from sources to detection}	
\label{ch:gws}
\fancyhf{}
\fancyfoot[C]{\thepage}
\fancyhead[RO]{\slshape\nouppercase{\rightmark}}
\fancyhead[LE]{\slshape\nouppercase{\leftmark}}

\renewcommand\thefigure{\arabic{chapter}.{\arabic{figure}}}
\import{chapters/}{chapter1_GRandGWs.tex}	
	
\chapter{Data analysis tools}
\label{ch:data_analysis}
\import{chapters/}{chapter2_data_analysis_tools.tex}

\chapter{Testing general relativity using higher-order modes of gravitational waves from binary black holes}
\label{ch:tgr_hom}
\import{chapters/}{chapter3_tgrhom.tex}

\chapter{Comparing gravitational waveform models for binary black hole mergers through a hypermodels approach}
\label{ch:hypermodels}
\import{chapters/}{chapter4_hypermodels.tex}


\chapter{Unraveling information about supranuclear-dense matter from the complete binary neutron star coalescence process using future gravitational-wave detector networks}
\label{ch:impm}
\import{chapters/}{chapter5_impm.tex}

\chapter{Measuring tidal effects with the Einstein Telescope: A design study}
\label{ch:coba}
\import{chapters/}{chapter6_coba.tex}

% Create the reference section using BibTeX:

\chapter{Conclusions}
\label{ch:conclusions}
\import{chapters/}{conclusions.tex}

\backmatter

\renewcommand\thefigure{S.{\arabic{figure}}}

\chapter*{Public summary}
\fancyhf{}
\fancyfoot[C]{\thepage}
\fancyhead[RO,LE]{\slshape\nouppercase{Public summary}}




\addcontentsline{toc}{chapter}{Public summary}
\import{chapters/}{public_summary.tex}

\chapter*{Openbare samenvatting}
\fancyhf{}
\fancyfoot[C]{\thepage}
\fancyhead[RO,LE]{\slshape\nouppercase{Openbare samenvatting}}

\setcounter{figure}{0}
\renewcommand\thefigure{Sn.{\arabic{figure}}}


\addcontentsline{toc}{chapter}{Openbare samenvatting}
\import{chapters/}{openbare_samenvatting.tex}

\chapter*{Sintesi per il pubblico}
\fancyhf{}
\fancyfoot[C]{\thepage}
\fancyhead[RO,LE]{\slshape\nouppercase{Sintesi per il pubblico}}

\setcounter{figure}{0}
\renewcommand\thefigure{Si.{\arabic{figure}}}


\addcontentsline{toc}{chapter}{Sintesi per il pubblico}
\import{chapters/}{riassunto_ita.tex}


\chapter*{Curriculum vitae}
\fancyhf{}
\fancyfoot[C]{\thepage}
\fancyhead[RO,LE]{\slshape\nouppercase{Curriculum vitae}}


\addcontentsline{toc}{chapter}{Curriculum vitae}
\import{chapters/}{cv.tex}

\chapter*{Acronyms}
\fancyhf{}
\fancyfoot[C]{\thepage}
\fancyhead[RO,LE]{\slshape\nouppercase{Acronyms}}

\addcontentsline{toc}{chapter}{List of acronyms}
\import{chapters/}{acronyms.tex}

\chapter*{Acknowledgments}
\fancyhf{}
\fancyfoot[C]{\thepage}
\fancyhead[RO,LE]{\slshape\nouppercase{Acknowledgments}}


\addcontentsline{toc}{chapter}{Acknowledgments}
\import{chapters/}{acknowledgments.tex}

\clearpage
\phantomsection
\addcontentsline{toc}{chapter}{Bibliography}
\fancyhf{}
\fancyfoot[C]{\thepage}
\fancyhead[RO,LE]{\slshape\nouppercase{Bibliography}}
\bibliographystyle{unsrt}
\bibliography{refs}

\afterpage{\blankpage}

          

\end{document}